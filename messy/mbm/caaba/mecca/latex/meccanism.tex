\documentclass[landscape]{article}

\makeatletter
% \def\shownote#1{\expandafter\gdef\csname note#1\endcsname{}}%
% \def\note#1#2{\@ifundefined{note#1}{\message{WARNING, NOTE NOT USED: #1}}{#1: #2}}%
\def\note#1#2{\code{#1}: #2}%
\makeatother

\def\myhline{\ifvmode\hline\fi}
\newcommand\tophline{\hline\noalign{\vspace{1mm}}}
\newcommand\middlehline{\noalign{\vspace{1mm}}\hline\noalign{\vspace{1mm}}}
\newcommand\bottomhline{\noalign{\vspace{1mm}}\hline}

\usepackage{multicol}
\usepackage{longtable}
\usepackage{natbib}
\usepackage{chem}
\usepackage{rotating} % loads graphics
\usepackage{amsmath} % \text etc.
%\usepackage{color}
\usepackage{colortbl}
% http://tex.stackexchange.com/questions/43006/why-is-input-not-expandable
\makeatletter\def\expandableinput#1{\@@input #1}\makeatother%

\newif\ifdraft
%\drafttrue
\ifdraft
  \usepackage{mypdfdraftcopy}
  \draftstring{DRAFT}\draftfontsize{150}\draftangle{330}
\fi 

\textwidth26cm
\textheight16cm
\topmargin-10mm
\oddsidemargin-15mm
\parindent0mm
\parskip1.0ex plus0.5ex minus0.5ex

\setlength{\LTcapwidth}{\textwidth}

\expandableinput{mecca_info.tex}
% show aerosol phase number:
\def\ONE{1}\ifx\apn\ONE
  \def\FormatAq#1#2{#1(aq)} % add suffix "(aq)" to aqueous-phase species
\else
  \def\FormatAq#1#2{#1(a#2)} % add suffix "(a01)", "(a02)", etc. to aqueous-phase species
\fi

\begin{document}

\thispagestyle{empty}
\begin{rotate}{-90}
\begin{minipage}{15cm}
\vspace{-30cm}
\begin{center}
  \ifdraft{\Huge\bf\color{red} PRELIMINARY}\\[3mm]\fi
  \LARGE {\bf The Chemical Mechanism of MECCA}\\[3mm]
  \Large KPP version: {\kppversion}\\[2mm]
  \Large MECCA version: {\meccaversion}\\[2mm]
  \Large Date: \today\\[2mm]
  \Large Batch file: \batchfile\\[2mm]
  \Large Integrator: \integr\\[2mm]
  \Large Gas equation file: \gaseqnfile\\[2mm]
  \Large Replacement file: \rplfile\\[2mm]
  \Large Selected reactions:\\
  ``\wanted''\\[2mm]
  Number of aerosol phases: \apn\\[2mm]
  Number of species in selected mechanism:\\
  \begin{tabular}{lr}
  Gas phase:     & \gasspc\\
  Aqueous phase: & \aqspc\\
  All species:   & \allspc\\
  \end{tabular}\\[2mm]
  Number of reactions in selected mechanism:\\
  \begin{tabular}{lr}
    Gas phase (Gnnn):                 & \Geqns\\
    Aqueous phase (Annn):             & \Aeqns\\
    Henry (Hnnn):                     & \Heqns\\
    Photolysis (Jnnn):                & \Jeqns\\
    Aqueous phase photolysis (PHnnn): & \PHeqns\\
    Heterogeneous (HETnnn):           & \HETeqns\\
    Equilibria (EQnn):                & \EQeqns\\
    Isotope exchange (IEXnnn):        & \IEXeqns\\
    Tagging equations (TAGnnn):       & \TAGeqns\\
    Dummy (Dnn):                      & \Deqns\\
    All equations:                    & \alleqns
  \end{tabular}\\[30mm]
  % see also IfFileExists in caaba_mecca_manual.tex:
  \IfFileExists{/home/sander/papers/sander-caaba4-refxxxx/README}
  {This document is part of the electronic supplement to our article\\
  ``The atmospheric chemistry box model CAABA/MECCA-4.0''\\
  in Geosci.\ Model Dev.\ (2018), available at:\\
  \url{http://www.geosci-model-dev.net}} % if branch
  {} % else branch  
\end{center}
\end{minipage}
\end{rotate}
\newpage

% define latex names of kpp species
\expandableinput{mecca_spc.tex}

\begin{longtable}{llp{9cm}p{6cm}p{5cm}}
\caption{Gas phase reactions}\\
\hline
\# & labels & reaction & rate coefficient & reference\\
\hline
\endfirsthead
\caption{Gas phase reactions (... continued)}\\
\hline
\# & labels & reaction & rate coefficient & reference\\
\hline
\endhead
\hline
\endfoot
\expandableinput{mecca_eqn_g.tex}
\end{longtable}

\newpage\begin{multicols}{3}
\section*{General notes}

\subsection*{Three-body reactions}

Rate coefficients for three-body reactions are defined via the function
\code{k_3rd}($T$, $M$, $k_0^{300}$, $n$, $k_{\rm inf}^{300}$, $m$,
$f_{\rm c}$). In the code, the temperature $T$ is called \code{temp} and
the concentration of ``air molecules'' $M$ is called \code{cair}. Using
the auxiliary variables $k_0(T)$, $k_{\rm inf}(T)$, and $k_{\rm ratio}$,
\code{k_3rd} is defined as:
\begin{eqnarray}
  k_0(T)              & = & k_0^{300} \times \left( \frac{300
                            \unit{K}}{T} \right)^n\\
  k_{\rm inf}(T)      & = & k_{\rm inf}^{300} \times \left( \frac{300
                            \unit{K}}{T} \right)^m\\
  k_{\rm ratio}       & = & \frac{k_0(T) M}{k_{\rm inf}(T)}\\
  \mbox{\code{k_3rd}} & = & \frac{k_0(T) M}{1+k_{\rm ratio}} \times f_{\rm
                            c}^{\left( \frac{1}{1+(\log_{10}(k_{\rm
                            ratio}))^2} \right)}
\end{eqnarray}

A similar function, called \code{k_3rd_iupac} here, is used by
\citet{1745} for three-body reactions. It has the same function
parameters as \code{k_3rd} and it is defined as:
\begin{eqnarray}
  k_0(T)                    & = & k_0^{300} \times \left( \frac{300
                                  \unit{K}}{T} \right)^n\\
  k_{\rm inf}(T)            & = & k_{\rm inf}^{300} \times \left( \frac{300
                                  \unit{K}}{T} \right)^m\\
  k_{\rm ratio}             & = & \frac{k_0(T) M}{k_{\rm inf}(T)}\\
  N                         & = & 0.75 - 1.27 \times \log_{10}(f_{\rm c})\\
  \mbox{\code{k_3rd_iupac}} & = & \frac{k_0(T) M}{1+k_{\rm ratio}} \times f_{\rm
                                  c}^{\left( \frac{1}{1+(\log_{10}(k_{\rm
                                  ratio})/N)^2} \right)}
\end{eqnarray}

\subsection*{Structure-Activity Relationships (SAR)}

Some unmeasured rate coefficients are estimated with structure-activity
relationships, using the following parameters and substituent factors:

{\renewcommand{\arraystretch}{1.25}
\begin{tabular}{ll}
  \tophline
  \multicolumn{2}{c}{$k$ for H-abstraction by \chem{OH} in \unit{cm^{-3} s^{-1}}}\\
  \middlehline
  \verb|k_p|          & $4.49\E{-18} \times (T/{\rm K})^2 \exp(-320\,{\rm K}/T)$\\
  \verb|k_s|          & $4.50\E{-18} \times (T/{\rm K})^2 \exp(253\,{\rm K}/T)$\\
  \verb|k_t|          & $2.12\E{-18} \times (T/{\rm K})^2 \exp(696\,{\rm K}/T)$\\
  \verb|k_ROHRO|      & $2.1\E{-18}  \times (T/{\rm K})^2 \exp(-85\,{\rm K}/T)$\\
  \verb|k_CO2H|       & $0.7 \times k_{\chem{CH_3CO_2H+OH}}$\\
  \verb|k_ROOHRO|     & $0.6 \times k_{\chem{CH_3OOH+OH}}$\\
  \middlehline
  \verb|f_alk|        & 1.23\\
  \verb|f_sOH|        & 3.44\\
  \verb|f_tOH|        & 2.68\\
  \verb|f_sOOH|       & 8.\\
  \verb|f_tOOH|       & 8.\\
  \verb|f_ONO2|       & 0.04\\
  \verb|f_CH2ONO2|    & 0.20\\
  \verb|f_cpan|       & 0.25\\
  \verb|f_allyl|      & 3.6\\
  \verb|f_CHO|        & 0.55\\
  \verb|f_CO2H|       & 1.67\\
  \verb|f_CO|         & 0.73\\
  \verb|f_O|          & 8.15\\
  \verb|f_pCH2OH|     & 1.29\\
  \verb|f_tCH2OH|     & 0.53\\
  \bottomhline
\end{tabular}}

{\renewcommand{\arraystretch}{1.25}
\begin{tabular}{ll}
  \tophline
  \multicolumn{2}{c}{$k$ for OH-addition to double bonds in \unit{cm^{-3} s^{-1}}}\\
  \middlehline
  \verb|k_adp|        & $4.5\E{-12} \times (T/300\,{\rm K})^{-0.85}$\\
  \verb|k_ads|        & $1/4 \times (1.1\E{-11} \times \exp(485\,{\rm K}/T)+$\\
                      & $1.0\E{-11} \times \exp(553\,{\rm K}/T))$\\
  \verb|k_adt|        & $1.922\E{-11} \times \exp(450\,{\rm K}/T) - k_{\text{ads}}$\\
  \verb|k_adsecprim|  & $3.0\E{-11}$\\
  \verb|k_adtertprim| & $5.7\E{-11}$\\
  \middlehline
  \verb|a_PAN|        & 0.56\\
  \verb|a_CHO|        & 0.31\\
  \verb|a_COCH3|      & 0.76\\
  \verb|a_CH2OH|      & 1.7\\
  \verb|a_CH2OOH|     & 1.7\\
  \verb|a_COH|        & 2.2\\
  \verb|a_COOH|       & 2.2\\
  \verb|a_CO2H|       & 0.25\\
  \verb|a_CH2ONO2|    & 0.64\\
  \bottomhline
\end{tabular}}

\subsection*{\chem{RO_2} self and cross reactions}

The self and cross reactions of organic peroxy radicals are treated
according to the permutation reaction formalism as implemented in the
MCM \citep{2419}, as decribed by \citet{1618}. Every organic peroxy
radical reacts in a pseudo-first-order reaction with a rate constant
that is expressed as $k^{\rm 1st} = 2 \times \sqrt{k_{\rm self} \times
  {\tt k\_CH3O2}} \times [\chem{RO_2}]$ where $k_{\rm self}$ =
second-order rate coefficient of the self reaction of the organic peroxy
radical, \code{k_CH3O2} = second-order rate coefficient of the self
reaction of \chem{CH_3O_2}, and $[\chem{RO_2}]$ = sum of the
concentrations of all organic peroxy radicals.

\section*{Specific notes}

\def\onlythischannel{Only this channel considered as the intermediate
  radical is likely more stable than \chem{CHCH(OH)_2}.}

\def\alkylnitrateneglected{Alkyl nitrate formation neglected.}

\expandableinput{mecca_eqn_g_notes.tex}

\end{multicols}

\clearpage

\begin{longtable}{llp{10cm}p{6cm}p{4cm}}
\caption{Photolysis reactions}\\
\hline
\# & labels & reaction & rate coefficient & reference\\
\hline
\endfirsthead
\caption{Photolysis reactions (... continued)}\\
\hline
\# & labels & reaction & rate coefficient & reference\\
\hline
\endhead
\hline
\endfoot
J (gas) & & & & \\
\expandableinput{mecca_eqn_j.tex}
PH (aqueous) & & & & \\
\myhline
\expandableinput{mecca_eqn_ph.tex}
\end{longtable}

\begin{multicols}{3}
\section*{General notes}

$j$-values are calculated with an external module (e.g., JVAL) and then
supplied to the MECCA chemistry.

Values that originate from the Master Chemical Mechanism (MCM) by
\citet{2419} are translated according in the following way:\\
j(11)             $\rightarrow$ \code{jx(ip_COH2)}     \\
j(12)             $\rightarrow$ \code{jx(ip_CHOH)}     \\
j(15)             $\rightarrow$ \code{jx(ip_HOCH2CHO)} \\
j(18)             $\rightarrow$ \code{jx(ip_MACR)}     \\
j(22)             $\rightarrow$ \code{jx(ip_ACETOL)}   \\
j(23)+j(24)       $\rightarrow$ \code{jx(ip_MVK)}      \\
j(31)+j(32)+j(33) $\rightarrow$ \code{jx(ip_GLYOX)}    \\
j(34)             $\rightarrow$ \code{jx(ip_MGLYOX)}   \\
j(41)             $\rightarrow$ \code{jx(ip_CH3OOH)}   \\
j(53)             $\rightarrow$ j(isopropyl nitrate)   \\
j(54)             $\rightarrow$ j(isopropyl nitrate)   \\
j(55)             $\rightarrow$ j(isopropyl nitrate)   \\
j(56)+j(57)       $\rightarrow$ \code{jx(ip_NOA)}

\section*{Specific notes}

\expandableinput{mecca_eqn_j_notes.tex}
\expandableinput{mecca_eqn_ph_notes.tex}

\end{multicols}

\clearpage

\begin{longtable}{llp{7cm}p{6cm}p{55mm}}
\caption{Reversible (Henry's law) equilibria and irreversible
  (``heterogenous'') uptake}\\
\hline
\# & labels & reaction & rate coefficient & reference\\
\hline
\endhead
\hline
\endfoot
\expandableinput{mecca_eqn_h.tex}
\end{longtable}

\begin{multicols}{3}
\section*{General notes}

The forward (\verb|k_exf|) and backward (\verb|k_exb|) rate coefficients
are calculated in subroutine \verb|mecca_aero_calc_k_ex| in the file
\verb|messy_mecca_aero.f90| using accommodation coefficients and Henry's
law constants from chemprop (see \verb|chemprop.pdf|).

For uptake of X (X = \chem{N_2O_5}, \chem{ClNO_3}, or \chem{BrNO_3}) and
subsequent reaction with \chem{H_2O}, \chem{Cl^-}, and \chem{Br^-} in
H3201, H6300, H6301, H6302, H7300, H7301, H7302, H7601, and H7602, we
define:
$$k_{\rm exf}(\chem{X}) = \frac{k_{\rm mt}(\chem{X})\times {\rm LWC}}
{[\chem{H_2O}] + 5\E2 [\chem{Cl^-}] + 3\E5 [\chem{Br^-}]}$$
Here, $k_{\rm mt}$ = mass transfer coefficient, and $\rm LWC$ = liquid
water content of the aerosol. The total uptake rate of X is only
determined by $k_{\rm mt}$. The factors only affect the branching
between hydrolysis and the halide reactions. The factor 5\E2 was chosen
such that the chloride reaction dominates over hydrolysis at about
[\chem{Cl^-}] $>$~0.1~\unit{M} (see Fig.~3 in \citet{536}), i.e.\ when
the ratio [\chem{H_2O}]/[\chem{Cl^-}] is less than 5\E2. The ratio
5\E2/3\E5 was chosen such that the reactions with chloride and bromide
are roughly equal for sea water composition \citep{358}. These ratios
were measured for uptake of \chem{N_2O_5}. Here, they are also used for
\chem{ClNO_3} and \chem{BrNO_3}.

% \section*{Specific notes}

\expandableinput{mecca_eqn_h_notes.tex}

\end{multicols}

\clearpage

\begin{longtable}{llp{9cm}p{7cm}p{5cm}}
\caption{Heterogeneous reactions}\\
\hline
\# & labels & reaction & rate coefficient & reference\\
\hline
\endfirsthead
\caption{Heterogeneous reactions (... continued)}\\
\hline
\# & labels & reaction & rate coefficient & reference\\
\hline
\endhead
\hline
\endfoot
\expandableinput{mecca_eqn_het.tex}
\end{longtable}

\section*{General notes}

Heterogeneous reaction rates are calculated with an external module
(e.g., MECCA\_KHET) and then supplied to the MECCA chemistry (see
\url{www.messy-interface.org} for details)

% \section*{Specific notes}

\expandableinput{mecca_eqn_het_notes.tex}

\clearpage

% from here on, do not show aerosol phase number anymore:
\def\FormatAq#1#2{#1}

\begin{longtable}{llp{7cm}p{3cm}p{25mm}p{6cm}}
\caption{Acid-base and other equilibria}\\
\hline
\# & labels & reaction & $K_0[M^{m-n}]$ & -$\Delta H / R [K]$ & reference\\
\hline
\endhead
\hline
\endfoot
\expandableinput{mecca_eqn_eq.tex}
\end{longtable}

\begin{multicols}{2}
% \section*{General notes}

\section*{Specific notes}

\expandableinput{mecca_eqn_eq_notes.tex}

\end{multicols}

\clearpage

\begin{longtable}{llp{8cm}p{3cm}p{25mm}p{5cm}}
\caption{Aqueous phase reactions}\\
\hline
\# & labels & reaction & $k_0~[M^{1-n}s^{-1}]$ & $-E_a / R [K] $& reference\\
\hline
\endfirsthead
\caption{Aqueous phase reactions (...continued)}\\
\hline
\# & labels & reaction & $k_0~[M^{1-n}s^{-1}]$ & $-E_a / R [K] $& reference\\
\hline
\endhead
\hline
\endfoot
\expandableinput{mecca_eqn_a.tex}
\end{longtable}

\begin{multicols}{3}
% \section*{General notes}

\section*{Specific notes}

\expandableinput{mecca_eqn_a_notes.tex}

\end{multicols}

\clearpage

\begin{multicols}{3}
\bibliographystyle{egu} % bst file
\bibliography{meccalit} % bib files
\end{multicols}

\end{document}
