\def\myfiledate{\today}

\documentclass[a4paper,12pt,twoside]{article}
\usepackage{ucs}\usepackage[utf8x]{inputenc}

\ifx\pdfoutput\undefined %
  %\usepackage{draftcopy}
  \usepackage[dvips]{graphicx} % LaTeX
  \DeclareGraphicsExtensions{.eps}
\else %
  \usepackage[pdftex]{graphicx} % pdfLaTeX
  %\DeclareGraphicsExtensions{.pdf,.png,.jpg}
  \DeclareGraphicsExtensions{.pdf}
\fi

%\usepackage[square]{natbib}
\usepackage{chem}
%\usepackage{afterpage}
%\usepackage{url}
\usepackage{color}
%\usepackage{rotating} % loads graphicx
%\usepackage{longtable}
%\usepackage{eclclass}
%\usepackage{verbatim}
%\usepackage{html}
%\usepackage{multicol}

\def\nosep{\setlength\parsep{0mm}\setlength\topsep{0mm}\setlength\itemsep{0mm}}

\oddsidemargin-10mm
\evensidemargin-13mm
\topmargin-20mm
\textheight250mm
\textwidth170mm
\raggedbottom
\parindent0mm
\parskip1.0ex plus0.5ex minus0.5ex
\renewcommand{\arraystretch}{1}
\renewcommand{\topfraction}{0.95}
\renewcommand{\dbltopfraction}{0.95}
\renewcommand{\bottomfraction}{0.95}
\renewcommand{\floatpagefraction}{0.95}
\renewcommand{\dblfloatpagefraction}{0.95}
\renewcommand{\textfraction}{0.01}
\setcounter{topnumber}{3}
\makeatletter\def\captioncont{\@dblarg{\@caption\@captype}}\makeatother

\newcommand{\todo}[1]{{\uppercase{\bf ((TODO: #1))}}}

\def\mytitle{How to write a new MESSy submodel}

\begin{document}

\begin{center}
  {\LARGE\bf\mytitle}\\[3mm]
  Version from \myfiledate
\end{center}

Adapting your model code to the MESSy standard has several advantages:

\begin{itemize}\nosep
\item You have access to data from all other MESSy submodels during the
  model simulation (online-coupling).
\item Your code will become part of the MESSy community model.
\item You can easily upgrade to newer MESSy versions without the need to
  make any changes to your code.
\end{itemize}

Here is a quick receipe how to turn your code into a MESSy submodel. In
the following text, the name ``XYZ'' is used for the new submodel.

\section{How to write a MESSy box model}

First of all, you must produce a ``box-model'' version of your code.
This is easy when starting from scratch but may require some effort when
the model is spread over several files as part of a historically grown
code.

It should be noted, that the term ``box model'' is used loosely here: It
refers to the model in its smallest possible entity. For chemistry, this
is indeed a 0-dimensional box. For other processes (e.g. vertical
transport) it would be a 1-dimensional column.

\begin{enumerate}\nosep
\item Write down explicitly what input is needed by your submodel and
  what output is produced. THIS MUST BE A COMPLETE LIST! Do not forget
  any ``unimportant'' variables!
\item If necessary, convert your code from Fortran77 to Fortran90 or even more
  recent Fortran standards. The
  tool \code{freeform}
  (\url{http://marine.rutgers.edu/po/tools/perl/freeform}) can simplify
  that task.
\item Put your code into the file \code{messy_}{\em xyz}\code{.f90} and
  divide it into three phases:
  \begin{itemize}\nosep
  \item initialization
  \item integration (e.g.\ executed during time loop)
  \item finalization
  \end{itemize}
  This file is from now on called the ``core'' file, or more precisely
  the ``submodel core layer'' (SMCL) file.
\item Create a very simple base model called \code{messy_}{\em
    xyz}\code{_box.f90}, which formally serves as the ``submodel interface
  layer'' (SMIL) and the ``basemodel interface layer'' (BMIL), and in which
  you: 
  \begin{itemize}\nosep 
   \item Define ALL values of the parameters that
         the core file needs as input. These values can be either constant or
         time-dependent.  
   \item Call the initialization in the core file.  
   \item Call the integration in the core file (either once or repeatedly 
         within a time loop).  
   \item Call the finalization in the core file.  
  \end{itemize} 
  Each of these subroutines transfers data as subroutine parameters to and
  from the core file.
\item Your box model may use the generic files that MESSy provides in
  \begin{itemize}\nosep
  \item \code{messy/smcl/messy_main_constants_mem.f90}
  \item \code{messy/smcl/messy_main_tools.f90}
  \item \code{messy/smcl/messy_main_blather.f90}
  \end{itemize} 
\item Create a \code{Makefile} to compile the boxmodel.
\end{enumerate}

\section{How to integrate your code into MESSy}

If you have followed the above procedure strictly, your code is now
ready for integration into the MESSy framework:

\begin{enumerate}\nosep
\item Put the core file \code{messy_}{\em xyz}\code{.f90} into the
  directory \code{messy/smcl/}.
\item Create the directory \code{messy/mbm/}{\em xyz}\code{/} and put
  the box model file \code{messy_}{\em xyz}\code{_box.f90} and the
  \code{Makefile} into it.
\item Create a link of the core file into the box model directory:\\
  \code{ln} \code{-s} \code{messy/smcl/messy_}{\em xyz}\code{.f90}
  \code{messy/mbm/}{\em xyz}\code{/}\\
  This link ensures that exactly the same core file is used for the box
  model and for all other MESSy basemodels.
  If you have used any of the generic MESSy
  files \code{messy/smcl/messy_main_[constants_mem,tools,blather].f90},
  you must link them also into the boxmodel directory.
\item Create a new file \code{messy/smil/messy_}{\em
    xyz}\code{_si.f90}. This file is similar to the box model file
  \code{messy_}{\em xyz}\code{_box.f90}. However, instead of
  ``inventing'' typical input data for the submodel, it extracts the
  necessary data from the BMIL. Make sure that there is no
  basemodel-specific code in \code{messy/smcl/messy_}{\em xyz}\code{.f90}
  and no submodel-specific code in \code{messy/smil/messy_}{\em
    xyz}\code{_si.f90}! In this file you can utilize the MESSy infrastructure
  provided by the \code{messy_main_*.f90} files.
\item To achieve that a model simulation also executes your new
  submodel, you need to adapt the following MESSy files:
  \begin{itemize}
  \item \code{messy/bmil/messy_main_control_*.inc}: Insert calls
    to your subroutines in \code{messy_}{\em xyz}\code{_si.f90} here.
    There are several entry points in the initialization phase, the main time
    loop, and the finalization phase in this file. Be aware of the naming
    convention of the main entry points.
  \item \code{messy/bmil/messy_main_switch_bi.f90}: Add the
    submodel string (\code{modstr}), the submodel version
    (\code{modver}), and the logical \code{USE_}{\em XYZ} here.
  \item \code{messy/smcl/messy_main_switch.f90}: Add the logical
    \code{USE_}{\em XYZ} here.
  % \item Run \code{gmake} \code{messycheck} to confirm that your submodel
  %   is MESSy conform.
  % \item Add your new submodel to \code{messy/util/validate}, then run
  %   \code{gmake} \code{validate} to confirm that your submodel is MESSy
  %   conform.
  \item Add \code{USE_}{\em XYZ}\code{=.TRUE.} to the namelist file
    \code{switch.nml}.
  \item For custom-made output, add the new submodel to the namelist
    file \code{channel.nml}.
  \end{itemize}
\end{enumerate}

The easiest way to make these additions is to look how other submodels
are implemented in these files (e.g.\ \code{BUFLY}).

\end{document}
