% work areas are denoted by #####
%
%
%new intro chapter starting 31/05/01

%$Source: /server/cvs/mpiom1/mpi-om/doc/tecdoc_mpiom/Attic/c3.tex,v $\\
%$Revision: 1.1.2.1.4.2.2.2.2.3.2.1 $\\
%$Date: 2006/03/07 14:50:43 $\\
%$Name: mpiom_1_2_0 $\\


%\pagenumbering{arabic}
\thispagestyle{empty}
 
\chapter[Creating a New Setup]
{\Large{\bf Creating a New Setup}\label{ch:setup}}

\section[Creating the Grid]
{\Large{\bf Creating the Grid}\label{ch:setup:grid}}

\subsection{anta}

Step one is the creation of the latitudes and longitudes of the new Arakawa C-grid.
We will need this file during the rest of the grid generation procedure.
The job script to create a new bipolar grid for MPI-OM is called "mk\_anta-etopo2.job".
Users need to adjust position and size of the two poles as well as the grid resolution.
The program should be run at double precision.
The example below is for the GROB60 test-setup.

\begin{footnotesize}
\begin{verbatim}
#INPUT PARAMETERS TO BE EDIT BY THE USER
#####################################
#name of the grid
grid=GR60

#1st pole positions
rlat1=72.
rlon1=-40.

#phi determines size of 1st pole
#phi=pi/2 gives no hole
#phi smaller than pi/2 gives increasingly larger hole
#note: size of second pole adjusted with parameter je
phi=1.49

#2nd pole Eurasia
rlat2=-84.
rlon2=60.

#horizontal dimensions
ie=60
je=50
######################################
\end{verbatim}
\end{footnotesize}



INPUT/OUTPUT:
\begin{enumerate}

\item \textbf{INPUT: etopo2.ext} \newline
ETOPO-5 dataset (Digital relief of the
Surface of the Earth) in big endian extra format.

\item \textbf{OUTPUT: anta} \newline
Grid file with lat's  and lon's from the Arakawa C-grid. This file is
needed during the rest of the procedure, e.g. generation of initial and forcing data)

\item \textbf{OUTPUT: topo} \newline
Ocean topography in ASCII format. 
For coarse resolution setups this file usually needs to be "corrected"
by hand to get a proper representation of the ocean topography,
e.g. Bering Strait or Panama etc. All important sill depth should be
checked, e.g. Drake Passage or Denmark Strait.
\end{enumerate}


\subsection{arcgri}

Step two is the creation of the grid distances of the new grid.
The job script to create the second grid description file for MPI-OM 
is called "mk\_arcgri.job".
The file contains the grid distances dlxp, dlxp, dlxu, dlyu, dlyu and dlyv of
the scalar and vector points of the Arakawa C-grid as well as the
local coriolis parameter f.
Users need to change the dimensions ie and je accordingly. The 
program should be run at double precision.

INPUT/OUTPUT:
\begin{enumerate}

\item \textbf{INPUT: anta} \newline
The grid file with lat's  and lon's from the Arakawa C-grid
in big endian extra format.

\item \textbf{OUTPUT: arcgri} \newline
Grid file with distances dlxp, dlxp, dlxu, dlyu, dlyu and dlyv of
the scalar and vector points of the Arakawa C-grid as well as the
local coriolis parameter in big endian extra format.

\end{enumerate}


\subsection{BEK}

Step three is the creation of BEK file for the new grid. 
The job script to create the BEK file for MPI-OM 
is called "mk\_BEK.job".
The file stores the information on the ocean basins which is of importance for 
the diagnostic output. Users need to change the dimensions ie and je 
in the script accordingly. 

INPUT/OUTPUT:
\begin{enumerate}

\item \textbf{INPUT: anta} \newline
The grid file with lat's  and lon's from the Arakawa C-grid
in big endian extra format.

\item \textbf{INPUT: topo} \newline
Ocean topography in ASCII format.

\item \textbf{OUTPUT: BEK} \newline
Ocean basin information in ASCII format.

\item \textbf{OUTPUT: ibek.ext} \newline
Ocean basin information in EXTRA format.

\end{enumerate}

{\bf Attention:} The generated BEK file has to be modified by hand afterwards and
some diagnostic output requires modifications in the source code!


\section[Creating the Input]
{\Large{\bf Creating the Input}\label{ch:setup:input}}

\subsection{INISAL and INITEM}

In step four we generate the initial value data sets for temperature and salinity 
by interpolating a global ocean climatology onto the model grid.
Two global hydrographic climatologies are available. First, the "PHC" climatology
with improved arctic ocean information from \citet{Steele:2001} and second, 
the new global hydrography from \citet{Gouretski:2004}.
The job script to create the files from the \citet{Steele:2001} climatology is called
"mk\_phc.job", the script to create them from the \citet{Gouretski:2004} climatology is called
"mk\_sac.job". The user has to specify the dimensions ie and je and the vertical resolution.
The following example is from "mk\_phc.job":
\begin{footnotesize}
\begin{verbatim}
foreach code ( temp salt )
foreach time ( an )   # opton : mon

if $version == GR60 then
set me=60
set ne=50
set lev2=20
endif
\end{verbatim}
\end{footnotesize}

INPUT/OUTPUT:
\begin{enumerate}

\item \textbf{INPUT: anta} \newline
The grid file with lat's  and lon's from the Arakawa C-grid
in big endian extra format.

\item \textbf{INPUT:PAC or SAC file} \newline
Ocean climatology in EXTRA format.

\item \textbf{OUTPUT: INISAL, INITEM} \newline
Ocean initial values for temperature and salinity.

\end{enumerate}

\subsection{SURSAL and SURTEM}

The surface restoring maps, SURSAL and SURTEM, have to be extracted from the 3-D INISAL, INITEM
by selecting the first level (CDO or EXTRA tools, "sellevel"). A monthly climatology is only available 
for \citet{Steele:2001}.

\subsection{runoff\_obs and runoff\_pos}

Positions of river discharge are stored with latitudes and longitudes. Their referring grid points are 
assigned during runtime. Nevertheless, to make sure the positions fit to the grid runoff\_obs 
and runoff\_pos have to be specifically created for each setup. 
The job script to create the files is called "mk\_runoff.job".

INPUT/OUTPUT:
\begin{enumerate}

\item \textbf{INPUT: anta} \newline
The grid file with lat's  and lon's from the Arakawa C-grid
in big endian extra format.

\item \textbf{INPUT: runoff.nc} \newline
Global runoff in NetCDF format from the OMPI project.

\item \textbf{INPUT: land\_sea\_mask.ECMWF.ext4} \newline
The ECMWF land-sea mask in EXTRA format. 

\item \textbf{INPUT: ibek.ext} \newline
Ocean basin information in EXTRA format.

\item \textbf{OUTPUT: runoff\_obs} \newline
 Monthly mean river discharge data (currently for 53 positions) in EXTRA format.

\item \textbf{OUTPUT: runoff\_pos} \newline
 Longitudes and latitudes of the river discharge positions in EXTRA format.

\end{enumerate}

\section[Interpolate the Forcing]
{\Large{\bf Interpolate the Forcing}\label{ch:setup:forcing}}

Now that all grid information is available we need to provide the daily mean surface fields
of heat, freshwater and momentum fluxes (see also.\ref{sec:using:forcing}).
A popular choice is the climatological OMIP forcing as discussed in \ref{sec:numeric:omip}.
The job script to interpolate the forcing from the OMIP climatology 
to the choosen model grid is called
"forcing\_omip.job". The user has to specify the dimensions ie and je.

INPUT/OUTPUT:
\begin{enumerate}

\item \textbf{INPUT: anta} \newline
The grid file with lat's  and lon's from the Arakawa C-grid
in big endian extra format.

\item \textbf{INPUT:OMIP climatology data} \newline
2m\_dewpoint\_temperature.nc,
east\_west\_stress.nc,
north\_south\_stress.nc,
scalar\_wind.nc,
total\_precipitation.nc,
2m\_temperature.nc,
mean\_sea\_level\_pressure.nc,
total\_cloud\_cover.nc and
total\_solar\_radiation.nc 
in NetCDF format. 

\item \textbf{INPUT: land\_sea\_mask.ECMWF.ext4} \newline
The ECMWF land-sea mask in EXTRA format. 

\item \textbf{OUTPUT: surface forcing fields} \newline
Total cloud cover: GICLOUD \newline
Precipitation: GIPREC \newline
Solar radiation: GISWRAD \newline
Dew point temperature: GITDEW \newline
Surface temperature: GITEM \newline
10 m wind speed: GIU10 \newline
zonal wind stress: GIWIX \newline
meridional wind stress: GIWIY \newline
All files in EXTRA format.
\end{enumerate}

Also available is a forcing compiled from the NCEP/NCAR reanalysis (section \ref{sec:numeric:ncep}).
The script is called "forcing\_ncep.job". The NCEP/NCAR reanalysis provides the same input fields as the 
OMIP project. The only difference is that "forcing\_ncep.job" has to consider the leap years.

\section[Modifications in the Source]
{\Large{\bf Modifications in the Source}\label{ch:setup:source}}
The generation of a new grid requires some modifications in the source code,
encapsulated with an appropriate CPP switch (see \ref{ch:using:compiling:conditional}).
The new grid has to be defined in the following subroutines:
\begin{enumerate}

\item \textbf{mo\_param1.f90} \newline
Define new global model dimensions.

\item \textbf{diag\_ini.f90} \newline
Define new diagnostic locations for timeseries output.

\item \textbf{mo\_commodiag.f90} \newline
Define different variables for grid types with different orientation, 
because the through-flows are different.

\item \textbf{diagnosis.f90} \newline
Grid types are taken into account 
when calculating the diagnostic output.

\end{enumerate}

%%%%%%%%%%%%%%%%%%%%%%%%%%%%%%%%%%%%%%%%%%%%%%%%%%%%%%%%%%%%%%%%%%%%%%%%%%%%%%%%%
\clearpage



  
