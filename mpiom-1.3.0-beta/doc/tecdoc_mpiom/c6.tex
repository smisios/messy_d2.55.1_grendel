% work areas are denoted by #####
%
%
%new intro chapter starting 31/05/01

%$Source: /server/cvs/mpiom1/mpi-om/doc/tecdoc_mpiom/Attic/c6.tex,v $\\
%$Revision: 1.1.2.1.4.2.2.2.2.3.2.1 $\\
%$Date: 2006/03/07 14:50:43 $\\
%$Name: mpiom_1_2_0 $\\


%\pagenumbering{arabic}
\thispagestyle{empty}
 
\chapter[Sea Ice Model]
{\Large{\bf Sea Ice Model}}
\label{ch:ice}

The sea ice model in routine \texttt{ocice.f90} and its subroutines \texttt{growth.f90}, 
\texttt{budget.f90} and \texttt{obudget.f90} consists of three parts:
the dynamics of sea ice circulation, the thermodynamics of sea ice growth and melt and 
the thermohaline coupling to the ocean model (brine rejection).
The following description is mostly identical to \citet{Marsland:2003}
and very similar to the {HOPE} model description by \citet{wolff97}.


\section{Sea Ice Dynamics}
\label{ch:ice:dynamics}

Sea ice motion is determined by a two-dimensional momentum balance equation.
\begin{equation}
\label{eqn:icemomentum}
{d{\vec{v}_i} \over {dt}} + f(\vec{k}\times\vec{v}_i) =
- g\vec{\nabla}\zeta + {\vec{\tau_{a}}\over {\rho_i{h_i}}}
+ {\vec{\tau}_o \over {\rho_i h_i}} + \vec{\nabla}\cdot\sigma_{\mathit{mn}}
\end{equation}
Here $f$, $\vec{k}$, $\zeta$, $g$, and $t$ are as in Eqn~\ref{eqn:mom}.
Sea ice of thickness $h_i$ and density $\rho_i$
has a velocity $\vec{v}_i$ which responds to wind stress $\vec{\tau_{a}}$,
ocean current stress $\vec{\tau_{o}}$,
and an internal ice stress
represented by the two dimensional stress tensor $\sigma_{\mathit{mn}}$.
It is noted that inclusion of the nonlinear (advective) terms 
in Eqn~\ref{eqn:icemomentum}
considerably reduces the model time-step.
For the standard \mbox{MPI-OM} setup considered in Section~\ref{ch:numeric:curvilinear} the time-steps
with and without the advective terms are 15 and 36 minutes respectively.
To reduce computational expenditure the nonlinear terms were therefore neglected
in the simulations considered here.
In Eqn.~\ref{eqn:icemomentum} the stress terms are in units of \mbox{N m$^{-2}$}.
From above $\vec{\tau_a}$ is taken as prescribed forcing,
while from below $\vec{\tau_o}$ is parameterized as
\begin{equation}
\label{eqn:icestress}
{\vec{\tau_o}}  = \rho_w C_W 
|\vec{v}_1 - \vec{v}_i| (\vec{v}_1 - \vec{v}_i)
\end{equation}
where 
$\vec{v}_1$ is the upper ocean layer velocity
and
the constant coefficient of bulk momentum exchange is given by $C_W$.
Currently no turning angles are employed for the atmosphere and ocean/ice stress terms.

The choice of sea ice rheology $\sigma_{\mathit{mn}}$ determines 
the way in which ice flows, cracks, ridges, rafts and deforms.
Following Hibler (1979) internal sea ice stress is modeled
in analogy to  a nonlinear viscous compressible fluid obeying
the constitutive law
\begin{equation}
\label{eqn:constitutivelaw}
\sigma_{\mathit{mn}}=2\eta\dot\epsilon_{\mathit{mn}} +
\left\{(\xi -\eta)(\dot\epsilon_{11} + \dot\epsilon_{22}) -
{P_i\over{2}}\right\}
\delta_{\mathit{mn}}
\end{equation}
where $\dot\epsilon_{\mathit{mn}}$ is the strain rate tensor
and $\delta_{\mathit{mn}}$ 
($m,n \in \{1,2\}$) 
is the Kronecker delta.
The internal sea ice pressure $P_i$ 
is a function of 
sea ice thickness $h_i$
and subgridscale areal fractional sea ice compactness
\begin{equation}
\label{eqn:hibpress}
P_i = P^*h_i e^{-C(1-I)}
\end{equation}
where $P^*$ and $C$ are empirically derived constants.
% with values ($5000 Nm^{-1}$ and 20 respectively)
%taken from Hibler (1979).
The pressure is related to the nonlinear bulk $\xi$ and shear $\eta$ viscosities according to:
\begin{equation}\label{eqn:etazeta}
\xi = {P_i\over{2\Delta}} \hspace*{.1cm};\hspace*{.1cm} \eta = {\xi \over{e^2}}
\end{equation}
\begin{equation}\label{eqn:hibdelta}
\Delta = \left[ \left(\dot\epsilon_{11}^2 +
\dot\epsilon_{22}^2\right)\left(1+{1\over{e^2}}\right)
+4{\dot\epsilon_{12}^2\over{e^2}} +
2 \dot\epsilon_{11}\dot\epsilon_{22}\left(1-{1\over{e^2}}\right)
\right]^{\frac{1}{2}}
\end{equation}
Here $e$ is the ratio of the lengths
of the principal axes of the yield ellipse
(these correspond to the principal components in stress space,
i.e.\ $\sigma_{11}$ and $\sigma_{22}$ from Eqn~\ref{eqn:constitutivelaw}).
The yield ellipse discriminates between linear-viscous (internal)
and plastic (boundary) points in stress space, 
while exterior points cannot be reached.
Numerical problems arise when the strain rates are small.
Then the $\Delta$ in Eqn~\ref{eqn:hibdelta} approaches zero,
and the viscosities in Eqn~\ref{eqn:etazeta} approach infinity.
%rev%The problem is avoided by choosing the viscosities to be a maximum of their
%rev%function value given in Eqn~\ref{eqn:etazeta}, 
%rev%and an empirically chosen maximum value corresponding to
%rev%the function value when \mbox{$\Delta = \Delta_{\mathit{min}} = 2.0 \times 10^{-9}$}.
Following Hibler (1979), \nocite{hibler79}
the problem is avoided by choosing the viscosities to be a maximum of their
function value given in Eqn~\ref{eqn:etazeta},
and an empirically chosen maximum value corresponding to
the function value when 
\mbox{$\Delta = \Delta_{\mathit{min}} = 2.0 \times 10^{-9} s^{-1}$}.

\section{Sea Ice Thermodynamics}
\label{ch:ice:thermodynamics}

Thermodynamics of sea ice involves the determination
of the local growth or melt rate at the base of the sea ice and the local
melt rate at the surface. 
To allow for the prognostic treatment of the subgridscale fractional sea ice cover
the surface heat balance is solved separately for the ice covered and ice free areas.
That is,
the net atmospheric heat flux $Q_{a}$ is weighted according to the
open water heat flux $Q_{w}$ and heat
flux over sea ice (or sea ice and snow) $Q_{i}$.
\begin{equation}
\label{eqn:icethermo}
Q_{a} = (1 - I)Q_{w} + IQ_{i}
\end{equation}
 
%% iterative solution of sea ice/snow layer surface temperature
A thermodynamic equilibrium is sought at the interface between
the atmosphere and the sea ice/snow layer.
An initial solution $T_{\mathit{srf}}^{\ast}$
is found for the sea ice/snow layer
surface temperature
$T_{\mathit{srf}}$
from the energy balance equation
\begin{equation}
\label{eqn:icesurf}
Q_i + Q_{\mathit{cond}} = 0.
\end{equation}
The conductive heat flux $Q_{\mathit{cond}}$ within the sea ice/snow layer is assumed to be
directly proportional to the temperature gradient across
the sea ice/snow layer and inversely proportional to the
thickness of that layer (i.e.\ the so-called zero-layer formulation of Semtner, 1976).\nocite{semtner76}
\begin{equation}
\label{cond1}
Q_{\mathit{cond}} = k_i \frac{(T_{\mathit{freeze}} - T_{\mathit{srf}})}{\tilde{h}_i}
\end{equation}
Here $k_i$ is the thermal conductivity of sea ice,
$T_{\mathit{freeze}}$ the freezing temperature of sea water
and $\tilde{h}_i$ the effective thermodynamic sea ice thickness of
the sea ice/snow layer.
This effective thickness is defined to be
\begin{equation}
\label{cond2}
\tilde{h}_i = \frac{1}{I}\left(h_i + h_s \frac{k_i}{k_s}\right)
\end{equation}
where $h_s$ is the snow layer thickness
and $k_s$ is the thermal conductivity of the snow.
The ratio of the thermal conductivity of sea ice with respect to
that of snow
%,$\frac{k_i}{k_sn}$, 
is approximately 7.
This means that snow is seven times more effective as an insulator against
oceanic heat loss to the atmosphere than sea ice.
Hence, even a relatively thin snow cover will result
in a much increased effective sea ice thickness.
Atmospheric precipitation is converted to snow fall when
$T_{a}$ is below 0$^{\circ}$C\@.
Snow loading on the sea ice may result in the
submerging of the sea ice/snow interface.
In such cases the thickness of the snow draft is converted to sea ice.
Since the heat of fusion of snow is slightly greater than the heat of fusion
of sea ice this process results in a net heat gain to the sea ice/snow layer.
To close the heat balance of the conversion process a small additional
amount of snow is also melted.

When the initial solution $T_{\mathit{srf}}^{\ast}$ in Eqn~\ref{cond1}
is greater than 0$^{\circ}$C
the left-hand side of Eqn~\ref{eqn:icesurf} is recalculated with
$T_{\mathit{srf}}$ replaced by 0$^{\circ}$C and the resultant
energy  is used to melt snow and then sea ice from above.
In the case where the entire sea ice/snow layer is melted from
above any remaining heat is added to $Q_w$ in
Eqn~\ref{eqn:icethermo}.

%update of oceanic surface temperature
To complete the sea ice thermodynamic evolution a heat balance
equation must also be applied at
the ocean/sea ice and ocean/atmosphere interfaces.
The balance equation takes the form
\begin{equation}
\label{iceundersurf}
\rho_w c_w \Delta{z}_{1}^{\prime}{\partial{\hat{\theta}_1} \over{\partial t}} =
(1-I) Q_w + I(Q_{\mathit{cond}} - h_i \rho_i L_i)
\end{equation}
and is solved for an interim upper layer oceanic temperature $\hat{\theta}_1$.
Here $c_w$ is the specific heat capacity of sea water,
$L_i$ is the latent heat of fusion of sea ice
and $\Delta{z}_{1}^{\prime}$ is the thickness of the upper ocean layer,
given by
\begin{equation}
\label{eqn:draft}
\Delta{z}_{1}^{\prime} = \Delta{z}_{1} + \zeta - h_{\mathit{draft}}
\end{equation}
where $\Delta{z}_{1}$ is the defined constant thickness of the ocean model's upper layer.
The draft of the sea ice/snow layer $h_{\mathit{draft}}$ is given by
\begin{equation}
\label{eqn:draft2}
h_{\mathit{draft}} = \frac{1}{\rho_w}
\left( \rho_ih_i + \rho_sh_s \right).
\end{equation}
where $\rho_i$ and $\rho_s$ are the densities
of the sea ice and snow layers respectively.
Note that the treatment of a sea ice draft is purely for thermodynamic considerations,
and that the ocean momentum balance is not effected.
The embedding of sea ice into the upper ocean layer,
as opposed to allowing sea ice to exist in multiple ocean layers,
is for computational convenience.
However, such treatment introduces an upperbound to the sea thickness.
In \mbox{MPI-OM} the sea ice draft is not allowed to remain above a local
maximum sea ice draft specified as
\begin{equation}
\label{eqn:maxice}
h_{\mathit{maxdraft}} = 0.7 (\Delta{z}_{1} + \zeta).
\end{equation}
Any additional sea ice draft is converted to water
in a salt (but currently not heat) conserving way.
It is noted that this critical sea ice thickness is never reached in
simulations using the standard grid of \mbox{MPI-OM} (Section~\ref{ch:numeric:curvilinear})
forced with OMIP (Section~\ref{sec:numeric:omip}) or NCEP (Section~\ref{sec:numeric:ncep}) surface forcing.
For the sea ice
undersurface to be in thermal equilibrium with the upper ocean it is
required that $T_{\mathit{melt}} \le \theta_1 \le T_{\mathit{freeze}}$.
To maintain this inequality
sea ice/snow is melted when the solution for $\hat{\theta}_1$ from
Eqn~\ref{iceundersurf} is above $T_{\mathit{melt}}$
and new sea ice is formed
when $\hat{\theta}_1$ is below $T_{\mathit{freeze}}$.
For the purposes of this study the effect of salinity on the freezing
and melting temperatures is ignored
and constant values of $T_{\mathit{freeze}}$
and $T_{\mathit{melt}}$ are used.
The model upper layer ocean temperature $\theta_1$
is only allowed to rise above $T_{\mathit{melt}}$ when all of
the sea ice/snow layer has been melted within a grid cell.
Then the new upper ocean temperature $\theta_1$
and the change in sea ice thickness $\Delta h_i$
are given by 
\begin{equation}
\label{melt1}
\theta_1 = \hat{\theta}_1 -
\min\left\{\frac{h_i{\rho_i}L_f}{\rho_w{c_w}\Delta{z}^{\prime}_1},
\hat{\theta}_1 - T_{\mathit{freeze}}\right\}
\end{equation}
\begin{equation}
\label{melt2}
\Delta{h_i} = \max\left\{ (T_{\mathit{freeze}} - \hat{\theta}_1)
\frac{\rho_w{c_w}\Delta{z}^{\prime}_1}{{\rho_i}L_f},0 \right\}  .
\end{equation}
For freezing conditions, 
the upper ocean temperature and the sea ice thickness change are
\begin{equation}
\label{freeze0}
\theta_1 = T_{\mathit{freeze}}
\end{equation}
\begin{equation}
\label{freezing}
\Delta h_i = \frac{\hat{\theta}_1 - T_{\mathit{freeze}}}{\rho_i L_f}
{\rho_w}c_w\Delta{z}^{\prime}_1 .
\end{equation}


%thermodynamic lead opening/closing
Subgridscale thermodynamic processes of sea ice growth and melt are assumed
to effect the sea ice compactness within a grid cell in the
following ways.
When freezing occurs over open water areas the sea ice compactness
increases (i.e.\ leads concentration decreases) at a rate given by
\begin{equation}
\label{eqn:leads-freeze}
\Delta I^{\mathit{thin}} = \max\left\{\frac{\Delta
h_i^{\mathit{thin}}(1 - I)}{h_o \Delta t},0\right\}
\end{equation}
where $\Delta t$ is the model time-step,
$\Delta h_i^{\mathit{thin}} = \Delta t Q_w/(\rho_i L_f)$ is thermohaline coupling to the ocean model
the thickness of new sea ice formed and 
$h_o$ is an arbitrary demarcation thickness
(taken to be 0.5~m following Hibler, 1979).\nocite{hibler79}
When melting of thick sea ice occurs the sea ice compactness
decreases (i.e.\ leads concentration increases) at a rate given by
\begin{equation}
\label{eqn:leads-melt}
\Delta I^{\mathit{thick}} = \min \left\{\frac{\Delta h_i^{\mathit{thick}}
I}{2 h_i \Delta t},0\right\}
\end{equation}
where $\Delta h_i^{\mathit{thick}}$ is the change in sea ice
thickness due to the melting.
This formulation is based on the assumption that sea ice
thickness within a grid cell has a uniform distribution between $0$
and $2 h_i$.
The change in compactness of sea ice due to thermodynamic lead
opening and closing is then calculated as the sum of both these
terms.
\begin{equation}
{\partial I \over {\partial t}} =
\Delta I^{\mathit{thin}} + \Delta I^{\mathit{thick}}
\end{equation}


\section{Update of Salinity (Brine Rejection)}
\label{ch:ice:brine}
%update of salinity (brine rejection}
Completion of the sea ice thermohaline coupling to the ocean model
requires consideration of salt and fresh water exchanges during
sea ice growth and melt.
Sea ice is assumed to have a constant salinity independent of it's age
and denoted by $S_{\mathit{ice}}$.
While multi-year Arctic sea ice has a salinity of around 3~psu,
thinner ice in both hemispheres has a much higher salinity \citep{cox74,eicken92}.
For the simulations with the standard grid (Section~\ref{ch:numeric:curvilinear}) an intermediate value of 5~psu 
representing this global diversity has been chosen.
The ocean model's upper layer salinity S$_1$ is changed by an amount
$\Delta S$ due to the surface fresh water flux (modified by snow fall
which accumulates on top of the sea ice) and due to sea ice growth or melt,
according to:
\begin{equation}
\label{eqn:icesalt}
\left( S_1 + \Delta S \right) \Delta z^{\prime \mathit{old}}
+ \frac{\rho_i h_i^{\mathit{old}}}{\rho_w} S_{\mathit{ice}}
=
S_1 \Delta z^{\prime \mathit{new}}
+ \frac{\rho_i h_i^{\mathit{new}}}{\rho_w} S_{\mathit{ice}}
\end{equation}
Here $\Delta z^{\prime \mathit{old}}$ is the
upper ocean layer thickness accounting for sea surface
elevation and sea ice draft as in Eqn~\ref{eqn:draft},
and also for the atmospheric precipitation minus evaporation.
$\Delta z^{\prime \mathit{new}}$ is $\Delta z^{\prime \mathit{old}}$
modified by the new sea ice draft due to melt or growth,
and $h_i^{\mathit{new}} - h_i^{\mathit{old}}$ is the amount of
sea ice growth (if positive) or melt (if negative).

\section{Subroutines}
\label{ch:ice:subroutines}
\begin{itemize}
\item
Subroutine \texttt{growth.f90} calculates the ice thickness, compactness, snow depth, upper ocean
temperature and salinity due to atmospheric heat and freshwater fluxes.
\item
 Subroutine \texttt{budget.f90} calculates the growth rates for the ice covered part of a grid
cell with standard bulk formulas. First, the snow or ice skin temperature and the surface residual heat flux
at the interface to the atmosphere and at the bottom are computed. Second, surface melt of snow or ice 
and bottom ablation or aggregation are deduced.        
\item
 Subroutine \texttt{obudget.f90} calculates growth rates of new ice in the ice free part of a grid
cell with bulk formulas or the net atmospheric heat flux at the water-atmosphere interface.
      
\end{itemize}

%%%%%%%%%%%%%%%%%%%%%%%%%%%%%%%%%%%%%%%%%%%%%%%%%%%%%%%%%%%%%%%%%%%%%%%%%%%%%%%%%
\clearpage



  
