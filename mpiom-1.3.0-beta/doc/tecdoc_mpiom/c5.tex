% work areas are denoted by #####
%
%
%new intro chapter starting 31/05/01

%$Source: /server/cvs/mpiom1/mpi-om/doc/tecdoc_mpiom/Attic/c5.tex,v $\\
%$Revision: 1.1.2.1.4.2.2.2.2.3.2.1 $\\
%$Date: 2006/03/07 14:50:43 $\\
%$Name: mpiom_1_2_0 $\\


%\pagenumbering{arabic}
\thispagestyle{empty}
 
\chapter[Time Stepping]
{\Large{\bf Time Stepping}\label{ch:timestepping}}


\section{Time stepping method}
\label{ch:timestepping:method}


The motions associated with the vertically integrated velocity field (barotropic
part) are solved implicitly which  damps the external
gravity wave mode and thus allows the use of  a longer time-step in the integrations.
Time stepping in MPI-OM is based on the idea of operator splitting, which
is also called {\sl time splitting} or {\sl method of fractional steps},
as described by e.g.\ \citet{press88}. 
The method is illustrated with the following example (taken from \citet{press88}).

Suppose you have an initial value equation of the form
\begin{eqnarray}
{{\partial u}\over{\partial t}} = \mbox{\op L}u
\end{eqnarray}
where {\op L} is some operator. While {\op L} is not necessarily linear, suppose that it can
at least be written as a linear sum of $m$ pieces, which act additively on $u$,
\begin{eqnarray}
\mbox{\op L}u=\mbox{\op L}_1u+\mbox{\op L}_2u + \cdots \mbox{\op L}_m u
\end{eqnarray}
Finally, suppose that for each of the pieces, you already know a differencing scheme
for updating the variable $u$ from time-step $n$ to time-step $n+1$, valid if
that piece of the operator were the only one on the right-hand side. We will write
these updating symbolically as
\begin{eqnarray}
u^{n+1} & = & \mbox{\op U}_1(u^n,\Delta t)\nonumber \\
u^{n+1} & = & \mbox{\op U}_2(u^n,\Delta t)\nonumber \\
        & \cdots & \\
u^{n+1} & = & \mbox{\op U}_m(u^n,\Delta t)\nonumber
\end{eqnarray}
Now, one form of operator splitting would be to get from $n$ to $n+1$ by the
following sequence of updating:
\begin{eqnarray}
u^{n+(1/m)} & = & \mbox{\op U}_1(u^n,\Delta t) \nonumber \\
u^{n+(2/m)} & = & \mbox{\op U}_2(u^{n+(1/m)},\Delta t) \nonumber \\
            & \cdots & \\
u^{n+1} & = & \mbox{\op U}_m(u^{n+(m-1)/m},\Delta t) \nonumber
\end{eqnarray}

This operator splitting is used in {\sl MPI-OM}.

%%%%%%%%%%%%%%%%%%%%%%%%%%%%%%%%%%%%%%%%%%%%%%%%%%%%%%%%%%%%%%%%%%%%%%%%%%%%%%%%%
% Klar wie Klosbruehe
\section{Timestepping in the Model}
\label{ch:timestepping:model}

The timestepping proceeds by the method of operator splitting or fractional steps
as described in Section~\ref{ch:timestepping:method}. 
That is, prognostic variables are updated
successively in several subroutines.
Prescribed forcing is read in at the start of each time-step after which the 
sea ice dynamics equations are solved by means of 
%successive over-relaxation with Chebychev acceleration.
functional iteration with under relaxation.
Then the sea ice thermodynamics are implemented.
The ocean momentum equation  is first solved partially for the friction terms and 
then the advection terms.
This results in a partially updated momentum equation which is 
decomposed into baroclinic and barotropic subsystems.
These are solved separately as described below.
As in HOPE \citep{wolff97} the prognostic equation for the free surface is solved
implicitly, which allows for the model's barotropic time-step
to equal the baroclinic time-step.

All subroutine called during one time step are listed in table \ref{tb:timestepping:sbr}
and are described in the following.
All symbols are explained in section \ref{sec:appendix:los} ``list of variables''.


\begin{table}[ht]
\begin{footnotesize}
        \begin{tabular}[t]{l|p{8cm}|l}
        \hline
          SBR name &
          Action &
          Ref. \\
        \hline\hline
         \texttt{octher.f90} &	   
           -boundary forcing on salt and temperature
	   
           -baroclinic pressure
	   
           -convective adjustment
	   
           -vertical eddy viscosity and diffusivity coefficients &
          \ref{ch:timestepping:octher} \\	  
         \texttt{-ocice.f90} &
	   Sea ice model as described by \citet{hibler79}&
          ch. \ref{ch:ice} \\		  
         \texttt{--growth.f90} &
	   Update of ice thickness, compactness, snow depth, upper ocean
           temperature and salinity due to atmospheric heat and freshwater fluxes.
          \\	
         \texttt{ocwind.f90} &
	   Update ocean velocities with the surface stress (wind forcing).&
          \ref{ch:timestepping:ocwind} \\
         \texttt{octide.f90} &
	   Include forcing from tides (optional).&
          \\
         \texttt{ocmodmom.f90} &
	   Decomposition into barotropic and baroclinic field.&
          \ref{ch:timestepping:ocmodmom} \\
%         \texttt{ocbarp.f90} &
%	   &
%          \\
         \texttt{occlit.f90} &
	   Solution of the baroclinic system.&
          \ref{ch:timestepping:occlit} \\
         \texttt{bartim.f90} &
	   Calculate sea level hight for the barotropic subsystem with gauss elimination and back-substitution (default).&
          \ref{ch:timestepping:bartim}\\
         \texttt{troneu.f90} &
	   Calculate sea level hight for the barotropic subsystem with iterative solver (SOR, optional).&
          \ref{ch:timestepping:troneu}\\
         \texttt{ocvtro.f90} &
	   Calculate new barotropic velocities.&
          \ref{ch:timestepping:ocvtro}\\
         \texttt{ocvtot.f90} &
	   Calculate new total velocities $u, v$ and $w$.&
          \ref{ch:timestepping:ocvtot}\\	  
         \texttt{ocuad.f90} &
	   Advection of momentum analogous to \texttt{ocadpo.f90} in u direction.&
          \ref{ch:timestepping:ocuad}\\	  
         \texttt{ocvad.f90} &
	   Advection of momentum analogous to \texttt{ocadpo.f90} in v direction.&
          \\		  
         \texttt{slopetrans.f90} &
	   Calculate bottom bondary layer (BBL) transport for tracer advection (optional).&
          \ref{ch:timestepping:slopetrans}\\		  
         \texttt{ocadpo.f90} &
	   Computes advection with a second order total variation diminishing (TVD) scheme 
	   \citep{Sweby:1984}. Called for temperature and salinity (optional).&
          \ref{ch:timestepping:ocadpo}\\		  
         \texttt{ocadfs.f90} &
	   Compute advection with predictor-corrector scheme
           or the quick-scheme as proposed by \citet{Farrow:1995} (optional).&
          \ref{ch:timestepping:ocadfs}\\		  
         \texttt{ocjitr.f90} &
	   Parameterize sub grid-scale eddy-induced tracer transport
           following \citet{Gent:1995} (optional).&
          \ref{ch:timestepping:ocjitr}\\	  
         \texttt{octdiff\_base.f90} &
	   Compute tracer-independent matrices for horizontal, isopycnal diffusion.&
          \ref{ch:timestepping:octdiff-base}\\		  
         \texttt{octdiff\_trf.f90} &
	   Calculate vertical diffusion and 
	   harmonic and biharmonic horizontal diffusion 
	   (with matrices calculated in \texttt{octdiff\_base.f90})
	   for temperature and salinity.&
          \ref{ch:timestepping:octdiff-trf}\\	  	  
         \texttt{relax\_ts.f90} &
	   3-D restoring of temperature and salinity to initial conditions.&
          \ref{ch:timestepping:relax-ts}\\	
%         \texttt{octimf.f90} &
%	   Update u and v velocities with ... ???? &
%          \\		  
         \texttt{ocschep.f90} &
	   Harmonic horizontal diffusion of momentum. &
          \ref{ch:timestepping:ocschep}\\		  
         \texttt{ocvisc.f90} &
	   Vertical diffusion of momentum, 
	   bottom friction, biharmonic horizontal momentum diffusion &
%	   and shear dependent diffusion&
          \ref{ch:timestepping:ocvisc}\\			  
        \end{tabular}
\end{footnotesize}
\caption{List of subroutine calls during one time-step.}
\label{tb:timestepping:sbr}
\end{table}


\subsection{octher.f90}
\label{ch:timestepping:octher}
\subsubsection{boundary forcing}

TO DO: The source has to be cleaned up. Encapsulate different parameterizations. 
CPP flag "NURMISCH" is identical to "CDVOCON == 0".  "NURDIFF" has to be default.

In ice free or non--compact ice covered areas the salinity is changed according to
\begin{eqnarray}
S^{n+1/m}\bigg\vert_{k=1}& = &S^n\bigg\vert_{k=1} + \Delta t\lambda_{S}\left(
                 S_{Levitus} - S^n\bigg\vert_{k=1}\right)\cdot (1-A)
\end{eqnarray}
where $A$ is the sea ice compactness. The salinity is further modified
in the sea ice model due to freezing or melting processes.
In addition, the sea surface temperature can relaxed to a SST climatology \citep{Levitus:1998}
\begin{eqnarray}
\Theta^{n+1/m}=\Theta_1^n + \Delta t \: \lambda_T\cdot(\Theta_{Levitus}-\Theta^n\bigg\vert_{k=1})
\end{eqnarray}
Default for the time constants is $\lambda_{S} = 3.*10^{-7}$ and $\lambda_{T} = 0.$ 
$\lambda_{S}$ and $\lambda_{T}$ can be modified in the namelist (\ref{tb:using:namelist}).
Both are weighted with the actual thickness of the first layer with respect to a thickness of 20~m
($\lambda = \lambda_{namelist}*20~m/DZW_{(1)}$).
Salinity and temperature restoring does only work if MPI-OM is not coupled to ECHAM. 
 
\clearpage
 
Discharge from rivers is affecting salinity and sea surface elevation. 
\begin{eqnarray}
S^{n+1/m}\bigg\vert_{k=1}& = &S^n\bigg\vert_{k=1} + {{\Delta z_{w}}\over{\Delta z_{1} + \Delta R_{input}}}\\
ZO^{n+1/m}& = &ZO + \Delta R_{input}\\
\end{eqnarray}
Local river input is calculated from discharge data for the given position (see section \ref{sec:using:input}):
\begin{eqnarray}
\Delta R_{input}& = &{{discharge*\Delta t}\over{\Delta x*\Delta y}}
\end{eqnarray}
 
%DRIV=GIRIV(I,J)*DT/(DLXP(I,J)*DLYP(I,J))
%     SAO(I,J,1)=SAO(I,J,1)*ZZZDZ/(ZZZDZ+DRIV)
%      ZO(I,J)=ZO(I,J)+DRIV


\subsubsection{baroclinic pressure and stability}


Hydrostatic pressure and stratification is computed.
Potential temperatures $\Theta$ are converted to in--situ temperatures $T$ (subroutine \texttt{adisitj.f90})
by solving the \citet{Gill:1982} formula with a Newton's method.
The density is computed in subroutine \texttt{rho1j.f90} with the \citet{unesco83} formula.

\begin{eqnarray}
p_1 &=& g\Delta z_{w1}\: \rho\left(S_1^{n+2/m}, T_1^{n+2/m}, p_{1(ref)}\right)\\
p_k &=& p_{k-1} + g\Delta z_{wk}\: \rho\left(S_k^{n+2/m}, T_k^{n+2/m}, p_{k(ref)}\right)\\
{{\partial\rho_k}\over{\partial z}} &=& {1\over\Delta z_{uk}} (\rho_{k-1}-\rho_k)^{n+2/m}
\end{eqnarray}
where the density $\rho$ is calculated using a reference pressure
\begin{eqnarray}
p_{(ref)k}= g\rho_0 h_k
\end{eqnarray}
where $h_k$ is the  depth of layer $k$.

There are several choices for parameterization of convection currently available in the \mbox{MPI-OM} model.
\begin{itemize}
\item
Default in cases of unstable stratification is a combination of vertical diffusion and mixing.
Other mechanisms are activated with compile flags (table \ref{tb:using:cpp-flags-physical}). 
\item
Compile flag "NURDIFF" disables the default mixing.
\item
Compile flag "UMKLAP" activates the convective adjustment. 
Convective adjustment follows \citet{bryan69}.
Traditionally this technique involved the full mixing of vertically adjacent grid cells 
in the presence of static instability.
The \mbox{MPI-OM} formulation is similar but only mixes the upper grid cell with an equivalent thickness
of the lower grid cell.
This approach aims to increase the penetrative depth of convection.
It is done with only one sweep through the water column per timestep, i.e.
for $\rho_z > 0$
\begin{eqnarray}
\Theta_k^{n+3/m} = {{\displaystyle \Delta z_{wk-1}\Theta_{k-1}^{n+2/m}+\Delta z_{wk}\Theta_{k}^{n+2/m}}\over
               {\displaystyle \Delta z_{wk-1}+\Delta z_{wk}}}\\
S_k^{n+3/m} = {{\displaystyle \Delta z_{wk-1}S_{k-1}^{n+2/m}+\Delta z_{wk}S_{k}^{n+2/m}}\over
               {\displaystyle \Delta z_{wk-1}+\Delta z_{wk}}}
\end{eqnarray}
for stable stratification $\rho_z < 0$ (or {\tt ICONVA = 0})
\begin{eqnarray}
\Theta_k^{n+3/m} =\Theta_{k}^{n+2/m}\\
S_k^{n+3/m} =S_{k}^{n+2/m}
\end{eqnarray}
If convective adjustment was active the stratification array is adjusted.

\item
Compile flag "PLUME" activates the so-called "PLUME" convection scheme (subroutine \texttt{nlopps.f90}). 
It is based on an original routine by E. Skyllingstad and T. Paluszkiewicz.
It is a much more physically based parameterization based on the
penetrative plume convection scheme of \citet{paluszkiewicz97}.
Plume convection was found to significantly improve the deep water characteristics and
the simulation of Southern Ocean sea ice in the {HOPE} model \citep{kim2001}.
However, the penetrative plume convection scheme is computationally quite expensive.
%For the parameter setting {\tt IEDDY = 1}

\item
If the compile flag  "NURMISCH" is set, only the  Richardson number depending coefficients are used for the diffusion.
Richardson number depending  coefficients for eddy viscosity and eddy diffusivity 
are computed every timestep \citep{pacanowski81}.
%(see e.g.\ Pacanowski and Philander, 1981, Monin and Yaglom, 1971, Jones, 1973).
Additionally, a mixed layer turbulence contribution is included.

\begin{eqnarray}\label{av1}
A_V^{\bf n+1} &=& \lambda_V A_V^n + (1-\lambda_V)\left[ {A_{V0}\over{1+(C_{RA} Ri)^2}}
\right] + A_B+ \delta_{\Delta T} W_T \\ \label{dv1}
D_V^{\bf n+1} &=& \lambda_V D_V^n + (1-\lambda_V)\left[ {D_{V0}\over{1+(C_{RD} Ri)^2}}
\right] + \delta_{\Delta T} W_T
\end{eqnarray}

where $0 \leq \lambda \leq 1$, $A_{V0}$ and $D_{V0}$ are constant values,
$A_B$ is the background mixing (set to $10^{-6}\mbox{ m}^2/\mbox{s}^{-1}$),
 $W_T$ is a value for
a wind induced mixed layer turbulence (increased turbulent
viscosity and diffusivity), $\delta_{\Delta T}$ is a
switch which is 1 for a vertical temperature difference to the sea surface temperature smaller than a preset $\Delta T$
and 0 elsewhere, and $Ri$ is the local Richardson number
\begin{eqnarray}
Ri = - {{g\partial\rho/ \partial z}\over{(\partial  u/ \partial z)^2+(\partial v/ \partial z)^2}}
\end{eqnarray}
Eqs.\ (\ref{av1}) and (\ref{dv1}) are slightly modified with respect to
the original formulations by \citet{pacanowski81}.

% For {\tt IEDDY = 0}
% all eddy coefficients are set constant in time and space.
\item
If the compile flag  "NURMISCH" is not set,
in cases of unstable stratification the coefficient is replaced by the mixing term "CDVOCON" which is set in the 
namelist (table \ref{tb:using:namelist}), if "CDVOCON" is larger that the Richardson coefficient.
This leads to a greatly enhanced vertical diffusion in the presence of static instability
(e.g.\ Marotzke, 1991; Klinger at al., 1996). \nocite{marotzke91,klinger96}
Such an approach avoids the excessive intermediate mixing associated with the traditional adjustment scheme
by introducing a timescale associated with the choice of (constant) convective-diffusion coefficient.
\end{itemize}

\subsection{ocwind.f90}
\label{ch:timestepping:ocwind}

TO DO: Clean Up, get rid of "SICOMU" and "SICOMV".

Wind stress and ice stress (per unit density) are added to the ocean velocities.
\begin{eqnarray}
u^{n+(1/m)} = u^n + \Delta t {{\tau^{x}_{wind}}\over{\Delta z_{w1}}}\cdot(1-A) + \Delta t {{\tau^{x}_{ice}}\over{\Delta z_{w1}}}\cdot A \\
v^{n+(1/m)} = v^n + \Delta t {{\tau^{y}_{wind}}\over{\Delta z_{w1}}}\cdot(1-A) + \Delta t {{\tau^{y}_{ice}}\over{\Delta z_{w1}}}\cdot A 
\end{eqnarray}
$A$ is the sea ice compactness. The ice stress is computed in subroutine \texttt{ocice.f90}.


\subsection{ocice.f90}
\label{ch:timestepping:ocice}

The sea ice model is computed in routine \texttt{ocice.f90} and its subroutines \texttt{growth.f90}, 
\texttt{budget.f90} and \texttt{obudget.f90}. A detailed described is given in chapter "Sea Ice Model".

In addition, the penetration of solar radiation is described in \texttt{ocice.f90} by a simple vertical
profile, constant with latitude and longitude.

\begin{eqnarray}
{I\over{I_0}}=(1-R)e^{(z/D)}
\end{eqnarray}

The radiation profile is converted to an absorption profile and used to update the temperatures.
There is no heat-flux through the bottom of the ocean, so all remaining heat is absorbed in the bottom layer.
If the marine biogeochemical model HAMOCC is included, there is an option to calculate the absorption as a 
function of chlorophyll. 

\subsection{ocmodmom.f90}
\label{ch:timestepping:ocmodmom}

The velocity fields $({\bf v}=(u,v))$ are decomposed into barotropic
(vertically averaged) and baroclinic parts
\begin{eqnarray}
{\bf V}&=&\int_{-H}^0 {\bf v} \: dz \\
{\bf v}' &=& {\bf v} - {1\over H}\int_{-H}^0 {\bf v} \: dz
\end{eqnarray}
The definition of ${\bf V}=(U,V)$ was chosen to be consistent with the coding, i.\ e.\ it is
the barotropic transport mode not a velocity.

\subsection{ocbarp.f90}
\label{ch:timestepping:ocbarp}

TO DO: No real function. Fill common block VSE, VZE, USO, UZO. Move to "bartim.f90" ????

\subsection{occlit.f90}
\label{ch:timestepping:occlit}

Solution of the baroclinic system. See section \ref{ch:timestepping:subsystem}

\subsection{bartim.f90}
\label{ch:timestepping:bartim}

Calculate sea level hight for the barotropic subsystem with an implicit method.
The equations are solved with a Gaussian triangulisation method.
Requires the subroutines \texttt{ocbarp.f90} and \texttt{trian.f90}.
See section \ref{ch:timestepping:subsystem}.

\subsection{troneu.f90}
\label{ch:timestepping:troneu}

If the compile flag  "SOR" is set (table \ref{tb:using:cpp-flags-physical}), the equations for the barotropic subsystem are solved by iteration
which requires less memory, but considerably more cpu time.
Requires the subroutines \texttt{itprep.f90} and \texttt{trotest.f90}.
See section \ref{ch:timestepping:subsystem}.

\subsection{ocvtro.f90}
\label{ch:timestepping:ocvtro}

The ``new'' sea level values are used to calculate new barotropic velocities.
%using eqs.\ \ref{uvdir}.

\subsection{ocvtot.f90}
\label{ch:timestepping:ocvtot}


\begin{itemize}
\item
Baroclinic and barotropic velocities are added to give total velocity fields.
The vertical velocity component is calculated from the continuity equation ($h$ is
layer thickness)

\parbox{13cm}{\begin{eqnarray*}
{\partial w^{n+1}\over{\partial z}}&=& -\beta\left[ {\partial\over\partial x}(hu^{n+1})
+ {\partial\over\partial y}(hv^{n+1})\right] \\
 & &-(1-\beta)\left[ {\partial\over\partial x}(hu^{n})
+ {\partial\over\partial y}(hv^{n})\right]\end{eqnarray*}}\hfill
\parbox{1cm}{\begin{eqnarray}\end{eqnarray}}

 The total velocity field of the new time step is available
in this subroutine
for further use (e.\ g.\ for diagnostics and post--processing).
\item
A very powerful consistency test for  the barotropic implicit
system and/or the correct back-substitution
 is available at this point, i.e. the consistency
of sea level change with the vertical velocity at z=0
\begin{eqnarray}
{\partial\zeta\over\partial t} - w|_{z=0} = 0
\end{eqnarray}
The results of this test should be of  round--off--error precision
(modified by the effects of twofold differencing, empirically : $10^{-8}$).

\item
Update time levels of velocity and sea level fields.

\item
Time memory of viscous dissipation according to local rate of strain ${\tt TURB}_{E/O}$.

\end{itemize}

\subsection{ocuad.f90 and ocvad.f90}
\label{ch:timestepping:ocuad}
Advection of momentum analogous to \texttt{ocadpo.f90} in u and v direction.

\subsection{slopetrans.f90}
\label{ch:timestepping:slopetrans}
Calculate bottom bondary layer (BBL) transport for tracer advection. 
For more details see section \ref{sec:bbl}.

\subsection{ocadpo.f90}
\label{ch:timestepping:ocadpo}
TO DO: BBL and ocadpo are not default. Is it necessary to 
compute the new vertical velocities in ocadpo (20 times with BGC)? 

First, new vertical velocities are calculated, including the BBL transport velocities.
Second, advection of scalar traces is computed with a second order total variation 
diminishing (TVD) scheme \citep{Sweby:1984}.
The total variation of a solution is defined as:
\begin{eqnarray}
TV_{(u^{n+1})}&=&\sum{u^{n+1}_{k+1}-u^{n+1}_{k}}
\end{eqnarray}
A difference scheme is defined as being total variation 
diminishing (TVD) if:
\begin{eqnarray}
TV_{(u^{n+1})}&\le&TV_{(u^{n})}
\end{eqnarray}
Momentum advection of tracers is by a mixed scheme that
employs a weighted average of both central-difference and upstream methods.
The weights are chosen in a two step process. First, according to the 
ratio of the first minus the second spatial derivative
over the first spatial derivative of the advected quantity T,
\begin{eqnarray}
r=max\left\{0,{|T^{\prime}|-|T^{\prime\prime}| \over |T^{\prime}|}\right\}
\end{eqnarray}
with $T^{\prime}=T_{k-1}-T_{k+1}$ and $T^{\prime\prime}=T_{k-1}+T_{k+1}-2\cdot T_{k}$.
If the second derivative is small usage of central-differencing is save and therefor favorably.
In contrast, if the first derivative is small and the second derivative is large there
is an extrema in the middle and an upstream scheme is preferred.

In a second step the ratio is weighted with strength of the flow
(the time it takes to fully ventilate the grid-box). 
Water transport in and out of a grid-box is given as:
\begin{eqnarray}
U_{in}&=&0.5\cdot\delta t \cdot\delta x \cdot\delta y \cdot(w_k +|w_{k-1}|) \nonumber \\
U_{out}&=&0.5\cdot\delta t \cdot\delta x \cdot\delta y \cdot(|w_{k+1}|-w_{k+1})
\end{eqnarray}
The total weight $R$ is defined as:
\begin{eqnarray}
R=min\left\{1,{\delta x \cdot\delta y \cdot\delta z \over U_{in}+U_{out}}\cdot r\right\}
\end{eqnarray}
If the flow is weak and $r$ is small, the magnitude of this ratio 
is less than 1 and the weights favor usage of central-differencing.
With a stronger flow or a larger $r$ the upstream scheme is preferred.
The idea is to incorporate the benefit of positive-definiteness of the upstream scheme
(and thus limit numerically spurious tracer sources and sinks),
while avoiding large implicit numerical diffusion in regions where strong gradients
exist in the tracer field.

Advection is computed as follows. Tracer transport in and out of a grid-box is defined as:
\begin{eqnarray}
T_{in}&=& U_{in}\cdot(1-R)\cdot T_k + R\cdot 0.5 \cdot (T_k+T_{k-1}) \nonumber \\
T_{out}&=& U_{out}\cdot(1-R)\cdot T_k + R\cdot 0.5 \cdot (T_k+T_{k+1})
\end{eqnarray}

The new tracer concentration $T_k^{n+1}$ is given by the old concentration $T_k^n$ 
plus tracer in-, and out-fluxes,
normalized by the "new" volume of the grid-box (volume plus in-, and out-fluxes of water).
\begin{eqnarray}
T_k^{n+1}&=&  \left(T_k^n \cdot\delta x \cdot\delta y \cdot\delta z  
+ T_{in \, k+1} - T_{in \, k} - T_{out \, k} + T_{out \, k-1}\right)\cdot{1\over V_{new}}
\nonumber \\
V_{new}&=& \delta x \cdot\delta y \cdot\delta z 
+ U_{in \, k+1} - U_{in \, k} - U_{out \, k} + U_{out \, k-1}
\end{eqnarray}



\subsection{ocadfs.f90}
\label{ch:timestepping:ocadfs}
Compute advection with either a predictor-corrector scheme
or with the quick-scheme as proposed by \citet{Farrow:1995}.
This routine does not work with BBL transport (\ref{ch:timestepping:slopetrans}).
\subsection{ocjitr.f90}
\label{ch:timestepping:ocjitr}
Parameterize sub grid-scale eddy-induced tracer transport
following \citet{Gent:1995}. The advection at the end is done with an upwind scheme.
\subsection{octdiff\_base.f90}
\label{ch:timestepping:octdiff-base}
Tracer-independent matrices for horizontal, isopycnal diffusion are computed for use 
in \texttt{octdiff\_trf.f90}.
See also equation \ref{eqn:kiso} in chapter \ref{ch:numeric}.
\subsection{octdiff\_trf.f90}
\label{ch:timestepping:octdiff-trf}
Calculate vertical diffusion and harmonic and biharmonic horizontal diffusion 
(with matrices calculated in \texttt{octdiff\_base.f90}) for scalar tracers
(temperature and salinity).
\subsection{relax\_ts.f90}
\label{ch:timestepping:relax-ts}
3-D restoring of temperature and salinity to initial conditions \citep{levitus98}.
\subsection{ocschep.f90}
\label{ch:timestepping:ocschep}
Harmonic horizontal diffusion of momentum.
\subsection{ocvisc.f90}
\label{ch:timestepping:ocvisc}
TO DO: clean up, TRIDSY to TRIDSY(i,j,k) as in HAMOCC routine octdiff\_bgc.f90 ?

Horizontal velocities are modified due to bottom friction, vertical harmonic momentum diffusion 
and horizontal biharmonic momentum diffusion. 
The vertical diffusion uses the vertical friction "avo" calculated in \texttt{oocther.f90}.


%Bottom friction (linear)
%\begin{eqnarray}
%{{\partial {\bf v}_{-H}}\over{\partial t}}&=& -\epsilon \cdot{\bf v}_{-H}\\
%{\bf v}_{-H}^{n+5/m}&=&{\bf v}_{-H}^{n+4/m} -\Delta t\: \epsilon \cdot{\bf v}_{-H}^{n+4/m}
%\end{eqnarray}


%%%%%%%%%%%%%%%%%%%%%%%%%%%%%%%%%%%%%%%%%%%%%%%%%%%%%%%%%%%%%%%%%%%%%%%%%%%%%%%%%%%%%%%

\section{Baroclinic and Barotropic Subsystem}
\label{ch:timestepping:subsystem} 

Denoting the internal baroclinic pressure divided by reference density as $p^{\prime}$,
and the three dimensional baroclinic velocities as ($u^{\prime},v^{\prime},w^{\prime}$),
the partially updated baroclinic momentum equations can be expressed by
\begin{equation}
\label{eqn:baroclinic_mom_u}
{\partial u^{\prime} \over{\partial t}} - f v^{\prime}
= 
{1 \over{H}} \int_{-H}^{\zeta}{\partial p^{\prime} \over{\partial x}}dz
- {\partial p^{\prime} \over{\partial x}}
\end{equation}
\begin{equation}
\label{eqn:baroclinic_mom_v}
{\partial v^{\prime} \over{\partial t}} + f u^{\prime}
=
{1 \over{H}} \int_{-H}^{\zeta}{\partial p^{\prime} \over{\partial y}}dz
- {\partial p^{\prime} \over{\partial y}}
\end{equation}
where $x$, $y$, $z$ and $t$ indicate the curvilinear parallel, curvilinear meridional,
vertical and temporal dimensions respectively.
The local depth is given by $H$, 
$\zeta$ is the sea surface displacement from the z-coordinate uppermost surface,
and $f$ is the Coriolis parameter.
Assuming only disturbances of small amplitude (linearization) 
allows the vertical density advection
to be expressed as 
${\partial \rho \over{\partial t}}=-w{\partial \rho \over{\partial z}}$.
This can be combined with the time derivative of the hydrostatic approximation
(${\partial \rho^2 \over{\partial z \partial t}}=
{-g \over \rho_0 }{\partial \rho \over{\partial t}}$)
to give an equation for the time evolution of the linearized internal baroclinic pressure
\begin{equation}
\label{eqn:baroclinic_press}
{\partial^{2} p^{\prime} \over{\partial z \partial t}}
=
{w g \over \rho_0} {\partial \rho \over \partial z}.
\end{equation}
Then the baroclinic subsystem is closed with the baroclinic continuity equation.
\begin{equation}
\label{eqn:baroclinic_cont}
  {\partial u^{\prime} \over{\partial x}} 
+ {\partial v^{\prime} \over{\partial y}} 
+ {\partial w^{\prime} \over{\partial z}} 
= 0
\end{equation}

Introducing the superscripts $n$ and $n+1$ to denote 
old and new time levels respectively,
the time discretization of the partially updated 
linearized baroclinic subsystem can then be written as
\begin{eqnarray}
\label{eqn:baroclinic_mom_u_discr}
u^{\prime^{n+1}} - u^{\prime^{n}}
&=&
\alpha \Delta{t}
\left(
f v^{\prime^{n+1}}
+ {1 \over{H}} \int_{-H}^{\zeta}p_{x}^{\prime^{n+1}} dz
- p_{x}^{\prime^{n+1}}
\right)
\nonumber \\
&&
+ (1-\alpha) \Delta{t}
\left(
f v^{\prime^{n}}
+ {1 \over{H}} \int_{-H}^{\zeta}p_{x}^{\prime^{n}} dz
-p_{x}^{\prime^{n}}
\right)
\\
\label{eqn:baroclinic_mom_v_discr}
v^{\prime^{n+1}} - v^{\prime^{n}}
&=&
\alpha \Delta{t}
\left(
- f u^{\prime^{n+1}}
+ {1 \over{H}} \int_{-H}^{\zeta}p_{y}^{\prime^{n+1}} dz
- p_{y}^{\prime^{n+1}}
\right)
\nonumber \\
&&
+ (1-\alpha) \Delta{t}
\left(
- f u^{\prime^{n}}
+ {1 \over{H}} \int_{-H}^{\zeta}p_{y}^{\prime^{n}} dz
-p_{y}^{\prime^{n}}
\right)
\\
\label{eqn:baroclinic_press_discr}
p_{z}^{\prime^{n+1}} - p_{z}^{\prime^{n}}
&=&
{g \over \rho_0} \Delta{t} \rho_z
\left(
\beta w^{\prime^{n+1}} + (1-\beta)w^{\prime^{n}}
\right)
\end{eqnarray}
Here $\Delta t$ is the model's timestep,
and $0 \le \alpha, \beta \le 1$ 
are stability coefficients partially weighting 
the new velocities to the old velocities.
For stability reasons it is required that $\alpha \ge 1-\alpha$ and $\beta\ge 1-\beta$.
In the semi-implicit case where $\alpha = \beta = {1 \over 2}$ the system is neutrally
stable, but similar to the familiar leapfrog-scheme this tends to produce
a computational mode with \mbox{$2\Delta t$} oscillations. 
These are suppressed
by the choice $\alpha$ = 0.55 and $\beta$ = 0.5.
The system is solved iteratively with the old baroclinic velocities used as a first guess.

For the partially updated barotropic subsystem, the momentum equations are
\begin{equation}
\label{eqn:barotropic_mom_u}
{\partial U \over{\partial t}}
- f V
+ g H {\partial \zeta \over{\partial x}}
+ \int_{-H}^{\zeta} {\partial \over{\partial x}}p^{\prime}dz
= 0
\end{equation}
\begin{equation}
\label{eqn:barotropic_mom_v}
{\partial V \over{\partial t}}
+ f U
+ g H {\partial \zeta \over{\partial y}}
+ \int_{-H}^{\zeta} {\partial \over{\partial y}}p^{\prime}dz
= 0
\end{equation}
where $U$ and $V$ are the partially updated barotropic velocities 
on the model's curvilinear grid.
The barotropic subsystem is closed with a continuity equation accounting for the
time derivative of the sea level $\zeta$
\begin{equation}
\label{eqn:barotropic_cont}
  {\partial \zeta \over{\partial t}}
+ {\partial U     \over{\partial x}}
+ {\partial V     \over{\partial y}}
=
Q_{\zeta}
\end{equation}
where the forcing term $Q_{\zeta}$ represents the surface freshwater flux 
(see Eqn~\ref{eqn:sfwf1} below).
The sea level is partially updated according to $Q_{\zeta}$ before the
baroclinic subsystem is solved and so is ignored in the following time discretization.

Denoting the partially updated sea level by $\zeta^{\prime}$,
the discretized partially updated barotropic subsystem can then be written as 
\begin{eqnarray}
\label{eqn:barotropic_mom_u_discr}
U^{n+1} - U^{n} -f \Delta t
\left( \alpha V^{n+1} + (1-\alpha)V^{n} \right)
&&
\nonumber \\
+ g H \Delta t 
\left( \alpha \zeta_x^{\prime^{n+1}} + (1-\alpha) \zeta_x^{\prime^n}  \right)
+ \Delta t 
\int_{-H}^{\zeta}p_{x}^{\prime^{n+1}} dz
&=& 0
\\
%\end{eqnarray}
%\begin{eqnarray}
\label{eqn:barotropic_mom_v_discr}
V^{n+1} - V^{n} +f \Delta t
\left( \alpha U^{n+1} + (1-\alpha)U^{n} \right)
&&
\nonumber \\
+ g H \Delta t
\left( \alpha \zeta_y^{\prime^{n+1}} + (1-\alpha) \zeta_y^{\prime^n}  \right)
+ \Delta t 
\int_{-H}^{\zeta}p_{y}^{\prime^{n+1}} dz
&=& 0
\end{eqnarray}
\begin{equation}
\label{eqn:barotropic_cont_discr}
\zeta^{\prime^{n+1}} - \zeta^{\prime^n} 
+ \Delta t 
%\left( \beta U_x^{n+1} + (1-\beta)U_x^{n} 
%  +    \beta V_y^{n+1} + (1-\beta)V_y^{n} \right)
\left( \beta ( U_x^{n+1} + V_y^{n+1}) + (1-\beta) (U_x^{n} + V_y^{n}) \right)
= 0
\end{equation}
where the partial temporal relaxation weights $\alpha$ and $\beta$ are the same as
for the baroclinic subsystem.
The set of equations \ref{eqn:barotropic_mom_u_discr}, \ref{eqn:barotropic_mom_v_discr}
and \ref{eqn:barotropic_cont_discr} is rearranged into matrix form and solved by
Gaussian elimination with back substitution.
Alternatively, when the model dimensions exceed the availability of core computing memory,
the matrix can be solved iteratively using successive over-relaxation.
For a discussion of how the implicit free surface approach of \mbox{MPI-OM}
compares with explicit treatment of the free surface the reader is referred to \citet{griffies2000}.

Momentum advection of tracers is by a mixed scheme that
employs a weighted average of both central-difference and upstream methods.
The weights are chosen according to the ratio of the second spatial derivative
over the first spatial derivative of the advected quantity.
When the magnitude of this ratio 
is less than 1 the weights favor usage of central-differencing,
and when greater than 1 the upstream scheme is preferred.
The idea is to incorporate the benefit of positive-definiteness of the upstream scheme
(and thus limit numerically spurious tracer sources and sinks),
while avoiding large implicit numerical diffusion in regions where strong gradients
exist in the tracer field.


%%%%%%%%%%%%%%%%%%%%%%%%%%%%%%%%%%%%%%%%%%%%%%%%%%%%%%%%%%%%%%%%%%%%%%%%%%%%%%%%%
\clearpage



  

