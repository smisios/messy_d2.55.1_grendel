\documentclass[11pt,a4paper,fleqn,twoside]{article}
\usepackage{float}
\usepackage{amsmath}
\usepackage{makeidx}
\usepackage{typearea}
\areaset[1.5cm]{15cm}{24cm}
\usepackage{./els1-4}
\bibliographystyle{./iris_aut1}
\setlength{\parindent}{0cm}
\setlength{\mathindent}{0cm}
\pagestyle{plain}
\renewcommand{\textfraction}{0}
\renewcommand{\topfraction}{1}
\renewcommand{\bottomfraction}{1}
\renewcommand{\floatsep}{0.5ex}      

%\renewcommand{\baselinestretch}{2}
% replace *.F90
%(?i<[\w\\]*.f90>) by \L&
%
% model tracer names

% dissolved inorganic
\def\pho{\text{\textsc{po$_4$}}\,}
\def\nit{\text{\textsc{no$_3$}}\,}
\def\ntwo{\text{\textsc{n$_2$}}\,}
\def\ntwoo{\text{\textsc{n$_2$o}}\,}
\def\car{\text{\textsc{c$_T^{12}$}}\,}
\def\cariso{\text{\textsc{c$_T^{14}$}}\,}
\def\oxy{\text{\textsc{o$_2$}}\,}
\def\alk{\text{\textsc{a$_T$}}\,}
\def\sio{\text{\textsc{si(oh)$_4$}}\,}
\def\dms{\text{\textsc{dms}}\,}

%anthropogenic tracers
\def\Acar{\text{\textsc{c$_T^A$}}\,}
\def\Aalk{\text{\textsc{a$_T^A$}}\,}
\def\Acal{\text{\textsc{caco$_3^A$}}\,}

% dissolved organic
\def\dom{\text{\textsc{dom}}\,}

% particulate
\def\phy{\text{\textsc{phy}}\,}
\def\zoo{\text{\textsc{zoo}}\,}
\def\det{\text{\textsc{det}}\,}
\def\opal{\text{\textsc{opal}}\,}
\def\cal{\text{\textsc{caco$_3$}}\,}

% aggregation
\def\snow{\text{\textsc{snow}}\,}
\def\nos{\text{\textsc{nos}}\,}
\def\adust{\text{\textsc{adust}}\,}

%iron and dust
\def\fe{\text{\textsc{fe}}\,}
\def\fdust{\text{\textsc{fdust}}\,}
\def\dust{\text{\textsc{dust}}\,}
\def\clay{\text{\textsc{clay}}\,}


%others
\def\dms{\text{\textsc{dms}}\,}
\def\hi{\text{\textsc{H$^+$}}\,}
\def\cartwo{\text{\textsc{CO$_3^{2-}$}}\,}

%units
\newcommand{\concP}{kmol P m$^{-3}$\,}
\newcommand{\fluxP}{kmol P m$^{-2}$ d$^{-1}$\,}
\newcommand{\concN}{kmol N m$^{-3}$\,}
\newcommand{\fluxN}{kmol N m$^{-2}$ d$^{-1}$\,}
\newcommand{\concC}{kmol C m$^{-3}$\,}
\newcommand{\fluxC}{kmol C m$^{-2}$ d$^{-1}$\,}
\newcommand{\concO}{kmol O m$^{-3}$\,}
\newcommand{\fluxO}{kmol O m$^{-2}$ d$^{-1}$\,}
\newcommand{\concSi}{kmol Si m$^{-3}$\,}
\newcommand{\fluxSi}{kmol Si m$^{-2}$ d$^{-1}$\,}
\newcommand{\concn}{particles cm$^{-3}$\,}
\newcommand{\conch}{kmol H m$^{-3}$\,}
\newcommand{\concalk}{kmol eq m$^{-3}$\,}
\newcommand{\concdust}{kg m$^{-3}$\,}
\newcommand{\concFe}{kmol Fe m$^{-3}$\,}
\newcommand{\concDMS}{kmol m$^{-3}$\,}


%others
\def\d{{\mathrm{d}}}
\newcommand{\degs}{$^{\circ}$}
\newcommand{\etal}{{\em et~al\,}.\,}
\newcommand{\concCHL}{mg Chl {\em a} m$^{-3}$\,}
\newcommand{\sink}{m d$^{-1}$\,}
\newcommand{\ham}{HAMOCC5.1\,}

\makeindex

\begin{document}

% HEADER RELEVANT TO THE ELSEVIER TEXTSTYLE
%\title{Technical description of the {\bf HAM}burg {\bf O}cean {\bf C}arbon 
%{\bf C}ycle model, version 5.1  (\ham), and of its interface to {\em MPI-OM}}

\title{The {\bf HAM}burg {\bf O}cean {\bf C}arbon {\bf C}ycle model \ham - 
                 Technical Description Release 1.1}

\author{Ernst Maier-Reimer, Iris Kriest, Joachim Segschneider, Patrick Wetzel\\
Max-Planck-Institute for Meteorology\\Bundesstr. 53\\ D-20146 Hamburg\\ Germany}
\date{\today}
%\begin{abstract}
%blabla
%\end{abstract}

%\maketitle

% FOR USE WITH RCS

\newpage

\tableofcontents{}

\newpage

\section{Introduction}

This is a technical description for \ham, release 1.1.

\ham is a model that simulates biogeochemical tracers in the oceanic water
column and in the sediment. It can be interfaced to any Ocean General Circulation
Model (OGCM). In the version presented here, it is set up as a subroutine of the
Ocean Model of the Max-Planck-Institute for Meteorology, version 1 ({\em
MPI-OM}, Marsland \etal, 2003\nocite{marsland:2003}). \ham\ is run forward in
time with a time step of 0.1 days. All biogeochemical tracers are fully
advected and mixed by the OGCM. The biogeochemical model itself is driven by
the same radiation as the OGCM to compute photosynthesis. Temperature and salinity 
are used to calculate various transformation rates and constants e.g., for solubility 
of carbon dioxide. 

The flux of carbon dioxide between atmosphere and ocean is computed 
depending on the local concentrations and the rates for  air-sea gas exchange. 
With only few modifications any field of atmospheric trace gases or wind stress 
can be used to drive the fluxes.

The biogeochemistry of \ham is based on that of HAMOCC3.1 (Six and
Maier-Reimer, 1996\nocite{six:1996}; hereafter referred to as SMR96). 
Modifications have been made to account for tracers in addition to phosphorous,
namely nitrogen, nitrous oxide, DMS, dissolved iron and dust. 
 Additional simulated processes are denitrification and
N-fixation, formation of calcium carbonate and opaline shells by phytoplankton,
aggregation and size dependent sinking, DMS production, dissolved iron uptake
and release by biogenic particles, and dust deposition and sinking. For
some  specific analyses additional tracers ``anthropogenic DIC and alkalinity''
where DICANTHR is treated like an isotope of carbon and ALKANTHR monitors the
effects of carbonate dissolution. It is not recommended to use these options
routinely. The model
now also features a sediment module which is based on  Heinze and
Maier-Reimer (1999)\nocite{heinze:1999b} and Heinze \etal (1999)\nocite{heinze:1999a}. 
The sediment model basically calculates the same tracers as the 
water column model. The numerical code of \ham consists of 36 FORTRAN90
subroutines. Coupling of the biogeochemistry to the ocean circulation requires 3
subroutines/header files.  

This report is intended to introduce the reader to the model structure of \ham, and to
assist in setting-up and running \ham driven by {\em MPI-OM}.  First, an
overview of the simulated biogeochemical processes is given (section
\ref{description_of_ham}). Second, the modules that compute biogeochemistry are
described in detail (section \ref{sources}). Third, the interface to 
{\em MPI-OM} and the coupling between the ocean physics
and the biogeochemical tracer model are described (section \ref{coupling}). The
input and output files are described in section \ref{io}. 
An appendix provides some (hopefully) useful tables
and cross-references for the usage of the model code.  



\section{\label{description_of_ham} Description of \ham}

\ham simulates three different components of the carbon cycle
and the interactions between them: air, water, and sediment. 
This section describes the simulation of the
biogeochemical processes within each sub-component, and the interface
to the whole system. 

\subsection{\label{the_water_column_bgc_model}The biogeochemical model of the water column}

The biogeochemical model for the water column in the standard setup three-dimensionally
computes eighteen different tracers (see table \ref{tab_water_tracers}). 
Photosynthesis and zooplankton grazing are restricted to the euphotic zone i.e. the  upper few layers
(down to about 90-120m, depending on the value of {\tt kwrbioz} in module {\tt
mo\_param1\_bgc.f90}, and the vertical grid-spacing of the ocean model). Below this
depth, all organic matter ultimately remineralizes to nutrients. In the
upper layers only aerobic processes are simulated. Denitrification is computed in
deeper layers if oxygen falls below a certain level. The following
subsections describe the simulation of the above processes.

%table1
\begin{table}[hbt]
\vspace{.2cm}
\caption{\label{tab_water_tracers} Biogeochemical tracers in the water
column: symbol, meaning, index-name and index-number {\tt ix} in the model's tracer field
{\tt ocetra(ie,je,ke,ix)}. {\tt ie}, {\tt je} and {\tt ke} give the
number of grid points in $x$ (longitude), $y$ (latitude) and $z$ (depth),
respectively. They are set in the ocean model and passed to the biogeochemical
model as an argument in the call to the respective biogeochemical subroutine. 
The index {\tt ix} is defined in {\tt mo\_param1\_bgc.f90}, and the model's
tracer field is declared in {\tt mo\_carbch.f90}.} 
\vspace{.2cm}
\begin{center}
\begin{tabular}{lllll} \hline
symbol & meaning        & index-name   & {\tt ix } & units\\ \hline
\multicolumn{5}{l}{\rule{0mm}{4mm}{\it default: tracers for
biogeochemistry}}\\ \hline
\car & total dissolved inorganic $^{12}$C         & {\tt isco212} & 1     &\concC\\
\alk & total alkalinity &{\tt ialkali}  & 2     & \concalk\\
\pho & phosphate        & {\tt iphosph} & 3     &\concP\\
\oxy & oxygen           &{\tt ioxygen}  & 4     &\concO\\
\ntwo& dinitrogen       & {\tt igasnit} & 5     &\concN\\
\nit & nitrate          & {\tt iano3}   & 6     &\concN\\
\sio & silicate         & {\tt isilica} & 7     &\concSi\\
\dom & dissolved organic matter & {\tt idoc}& 8 &\concP\\
\phy & phytoplankton    & {\tt iphy}    & 9     &\concP\\
\zoo & zooplankton      & {\tt izoo}    &10     &\concP\\
\det & detritus         & {\tt idet}    &11     &\concP\\
\cal & calcium carbonate shells         & {\tt icalc}   &12     &\concC\\
\opal& opal shells            & {\tt iopal}   &13     &\concSi\\ 
\cariso& total dissolved inorganic $^{14}$C   & {\tt isco214}   &14 &\concC\\ 
\ntwoo& nitrous oxide  & {\tt ian2o}   &15     &\concN\\ 
\dms& dimethyl sulfide             & {\tt idms}   &16     &\concDMS\\ 
\fdust& free (non-aggregated) dust & {\tt ifdust}   &17     &\concdust\\ 
\fe  & dissolved iron & {\tt iiron}   &18     &\concFe\\ 

\multicolumn{5}{l}{\rule{0mm}{4mm}{\it optional (}{\tt -DAGG}{\it ): tracers for
aggregation and size-dependent sinking}}\\ \hline
\nos & number of snow aggregates& {\tt inos}    &19     &\concn\\ 
\adust & aggregated dust& {\tt iadust}    &20     &\concdust\\ 

\multicolumn{5}{l}{\rule{0mm}{4mm}{\it optional (}{\tt -DPANTHROPOCO2}{\it
): tracers of anthropogenic origin}}\\ \hline
\Acar & dissolved inorganic C& {\tt isco212\_ant}    & 19   &\concC\\ 
\Aalk & alkalinity& {\tt ialk\_ant}    &20     &\concalk\\ 
\Acal& calcium carbonate & {\tt icalc\_ant}   &21     &\concC\\ 

\multicolumn{5}{l}{\rule{0mm}{4mm}{\it optional (}{\tt -DAGG and -DPANTHROPOCO2}
{\it ):}}\\ \hline
\Acar & anthropogenic DIC& {\tt isco212\_ant}    & 21   &\concC\\ 
\Aalk & anthropogenic alkalinity& {\tt ialk\_ant}    &22     &\concalk\\ 
\Acal& anthropogenic CaCO$_3$ & {\tt icalc\_ant}   &23     &\concC\\ \hline

\end{tabular}
\end{center}
\end{table}

\subsubsection{\label{euphotic_zone}Euphotic zone/upper layers biogeochemistry 
({\tt ocprod.f90})} 

The computation of the biogeochemistry in the euphotic zone is based on colimitation of  phosphorous, nitrate and iron. It is basically the same as in HAMOCC3.1 (see SMR96).
In this report only the modifications in \ham will be described in detail. 

HAMOCC3.1 calculates phosphorous concentrations and the associated carbon
biogeochemistry. \ham also takes into account nitrate, silicate and opal
production from diatom growth (eqns. \ref{eq_sio_sur} and \ref{eq_opal_sur}),
and calcium carbonate formation due to coccolithophorides (eq. \ref{eq_cal_sur}
and associated changes in $C_T$ and $A_T$).  Atmospheric dust input and transport
in the ocean, release of dissolved iron from dust and its influence on
photosynthesis, as well as DMS (dimethyl sulfide) dynamics are also modelled. The
interactions between the different tracers are simulated using stoichiometric
constants $R_{X:Y}$ (where $X$ and $Y$ stand for the respective tracers, see
table \ref{tab_bgc_params}). \footnote{The nomenclature for the parameters in
the text basically follows the suggestions made by Evans and Gar\c{c}on
(1997)\nocite{jgofs:1997}, where $\mu$ denotes maximum growth/grazing rates,
$\lambda$ denotes linear loss rates of biomass, and $\kappa$ denotes 
quadratic loss of biomass, or a loss rate based on the concentration of some
constituent. $\epsilon$ denotes fractions of fluxes that are gained, $\omega$
denotes the partioning of fluxes and $K$ denotes half saturation constants.
Indices (especially for the $\lambda$ terms) describe the direction of fluxes,
or the compartment to which the parameter is assigned.}

%table2
\begin{table}[htb]
{\caption{\label{tab_bgc_params}  Symbols in this text and names in the model code
of various parameters of \ham. They are declared in {\tt beleg\_bgc.f90} and communicated
between the modules in {\tt mo\_biomod.f90}.}}
\vspace{-.2cm}
\begin{center}
\renewcommand{\baselinestretch}{1}
\footnotesize
\begin{tabular}{llp{7cm}l} \hline 
symbol & name & variable & units \\ \hline
\multicolumn{4}{l}{\rule{0mm}{4mm}{\bf upper boundary and vertical delimiters}}\\ \hline
$I_0$ &{\tt strahl}& net solar radiation at surface& W m${-2}$\\ 
$k_{eu}$ &{\tt kwrbioz}  &index of deepest layer of euphotic zone & \\
$kb$ &{\tt kbo}& index field of bottom layer & \\
$\Delta z_{kb}$ &{\tt bolay}&  local thickness of bottom layer & m\\
\multicolumn{4}{l}{\rule{0mm}{4mm}\bf stoichiometry}\\ \hline
$R_{-O2:P}$ &{\tt ro2ut}& -O$_2$:P  ratio & mol O$_2$ mol P$^{-1}$\\
$R_{C:P}$&{\tt rcar}& C:P  ratio &mol C mol P$^{-1}$\\
$R_{N:P}$&{\tt rnit}& N:P  ratio &mol N mol P$^{-1}$\\
$R_{Fe:P}$&{\tt riron}& Fe:P  ratio &mol Fe mol P$^{-1}$\\
\multicolumn{4}{l}{\rule{0mm}{4mm}\bf upper ocean biogeochemistry: layers
1 to $k_{eu}$}\\
 \hline
\multicolumn{4}{l}{\rule{0mm}{4mm}\it phytoplankton}\\
$\alpha$  & - & initial slope of P-vs-I curve & d$^{-1}$ (W m${-2})^{-1}$\\
$k_w$&{\tt atten$_w$ }& light attenuation coeff. of water & m$^{-1}$ (\concP)$^{-1}$\\
$k_c$&{\tt atten$_c$ }& light attenuation coeff. of chlorophyll & m$^{-1}$ (\concP)$^{-1}$\\
$R_{C:Chl}$&{\tt ctochl }& C:Chl ratio of phytoplankton  & g C g Chl $^{-1}$\\
$K_{\phy}^{\pho}$&{\tt bkphy}& half-sat. constant for PO$_4$ uptake & \concP \\
$K_{\phy}^{\sio}$&{\tt bkopal}& half-sat. constant for Si(OH)$_4$ uptake & \concSi \\
$R_{Si:P}$&{\tt ropal}& Opal:P uptake ratio & mol Si mol P$^{-1}$\\
$R_{Ca:P}$&{\tt rcalc}& CaCO$_3$:P uptake ratio  & mol C mol P$^{-1}$\\
$M_{\cal}$&{\tt calmax}& max. fraction of CaCO$_3$ production & \\
$\phy_{0}$&{\tt phytomi}& min. concentration of phytoplankton  & \concP \\
$\lambda_{\phy,\det}^{surf}$&{\tt dyphy}& mortality rate & d$^{-1}$\\
$\lambda_{\phy,\dom}$&{\tt gammap}& exudation rate & d$^{-1}$\\
\multicolumn{4}{l}{\rule{0mm}{4mm}\it zooplankton}\\
$\mu_\zoo$ &{\tt grazra}& max. grazing rate & d$^{-1}$\\
$K_{\zoo}$&{\tt bkzoo}& half-saturation constant for grazing & \concP \\
$\zoo_{0}$&{\tt grami}& min. concentration of zooplankton&\concP \\
$1-\epsilon_{\zoo}$&{\tt epsher}& fraction of grazing egested  & \\
$\omega_{graz,\zoo}$&{\tt zinges}& assimilation efficiency & \\
$\lambda_{\zoo,\dom}$&{\tt gammaz}& excretion rate & d$^{-1}$\\
$\lambda_{\zoo}^{surf}$&{\tt spemor}& mortality rate & d$^{-1}$\\
$\omega_{mort,\pho}$&{\tt ecan}& fraction of mortality as PO$_4$& \\
\multicolumn{4}{l}{\rule{0mm}{4mm}\it dissolved organic matter}\\
$\lambda_{\dom,\pho}^{surf}$&{\tt remido}& remineralization rate & d$^{-1}$\\ 
\multicolumn{4}{l}{\rule{0mm}{4mm}\it dimethyl silfid (DMS)}\\
$D_1$ & {\tt dmspar(1)}& temp.-dependent release by phytoplankton& \degs C\\
$D_2$ & {\tt dmspar(2)}& photo destruction& (W m$^{-2}$)$^{-1}$ d$^{-1}$\\
$D_3$ & {\tt dmspar(3)}&  temp.-dependent destruction & \degs C$^{-1}$ d$^{-1}$\\
$D_4$ & {\tt dmspar(4)}& production by coccolithophorides  & kmol DMS (kmol Si)$^{-1}$ d$^{-1}$\\
$D_5$ & {\tt dmspar(5)}& production by diatoms  & kmol DMS (kmol Ca)$^{-1}$d$^{-1}$\\
$D_6$ & {\tt dmspar(6)}& microbial half saturation & kmol DMS \\
\multicolumn{4}{l}{\rule{0mm}{4mm}\bf deep remineralization: layers $(k_{eu}+1)$
to $kb$}\\ \hline
$\lambda_{\det,\pho}^{deep}$&{\tt drempoc}& detritus remineralization rate& d$^{-1}$\\
$\lambda_{\dom,\pho}^{deep}$&{\tt dremdoc}& DOM remineralization rate& d$^{-1}$\\
$\lambda_{\phy,\det}^{deep}$&{\tt dphymor}& phytoplankton mortality rate& d$^{-1}$\\
$\lambda_{\zoo,\det}^{deep}$&{\tt dzoomor}& zooplankton mortality rate& d$^{-1}$\\
\multicolumn{4}{l}{\rule{0mm}{4mm}\bf dissolution of opal and CaCO$_3$}\\ \hline
$\lambda_{\opal,\sio}$ &{\tt dremopal}& opal dissolution rate& d$^{-1}$\\
$\lambda_{\cal,\car}$ &{\tt dremcalc}& calcium carbonate dissolution rate& d$^{-1}$\\ 
\multicolumn{4}{l}{\rule{0mm}{4mm}\bf others}\\ \hline
$\mu_{NFix}$&{\tt bluefix}& nitrogen fixation rate ({\tt cyano.f90})& d$^{-1}$\\
$\epsilon_{Fe}\times S_{Fe}$ &{\tt perc\_diron}& weight fraction of iron in dust times iron solubility & \\
$\fe_{0}$ & -  & maximum value for excess diss. iron & \concFe\\
$\lambda_{Fe}$ &{\tt relaxfe}& complexation rate of excess diss. iron &
d$^{-1}$\\ \hline
\end{tabular}
\end{center}
\renewcommand{\baselinestretch}{1.5}
\normalsize
\end{table}

\paragraph{Phytoplankton}  Phytoplankton (\phy) \index{phytoplankton} growth
depends on availability of light ($I$) and nutrients. The local light supply is
calculated from the temporally and spatially varying solar radiation at the sea surface, $I(0,t)$, 
as provided by the OGCM.  Below the surface light intensity is reduced due to attuation 
by sea water ($k_w$) and chlorophyll ($k_c$) using a
constant conversion factor for C:Chl, $R_{C:Chl}$:

\begin{equation}
I(z,t) = I(0,t) \, e^{-(k_w+k_c\, \phy \, 12 \, R_{C:P} / R_{C:Chl}) z}  
\end{equation}

Phytoplankton growth depends linearly on the availability
of light, without saturation of growth rates for stronger
irradiance ($I$). The growth rate $J(I(z,t))$,
is calculated  as $J(I) = \alpha_\phy \, I(z,t)$, where $\alpha_\phy$ is the
slope of the P-vs-I-curve (production vs. light intensity). 
The light limited phytoplankton growth rate is then multiplied by the
nutrient limitation factor, which is calculated from a simple Monod function, 
limited by the least available nutrient (either phosphate, nitrate, or iron,  see equation
\ref{eq_phosy}). It is assumed that phytoplankton takes up P, N, and Fe
in constant proportions determined by the stoichiometric ratios $R_{N:P}$ and
$R_{Fe:P}$.
\begin{eqnarray}
\label{eq_phosy}
photosynthesis & = & \frac{\phy \, J(I(z,t)) \, X}
{K_{\phy}^{\pho} + X} \\
& & \qquad \mbox{with}
\quad X = \min\left(\pho,\frac{\nit}{R_{N:P}},\frac{\fe}{R_{Fe:P}}\right)
\nonumber
\end{eqnarray}

Constant time stepping can cause instabilities (overshoots) in the
biogeochemical equations. In order to avoid this, nutrient uptake by
phytoplankton is calculated using a modified backward approach:
\begin{eqnarray}
X^{t+\Delta t} & = & X^t - \phy^t \, J(I(z,t)) \, \frac {X^{t+\Delta t}}
{K_{\phy}^{\pho} + X^t} \, \Delta t\\ \nonumber
& = & \frac{X^t}{1+\frac{\phy^t \, J(I(z,t)) \,\Delta t}
{K_{\phy}^{\pho} +X^t}} \nonumber 
\end{eqnarray}

Thus, photosynthesis per timestep in phosphorous units is given by

\begin{equation*}
phosy  =  \frac{X^t - X^{t+\Delta t}}{\Delta t}
\end{equation*}

Phytoplankton exudate dissolved organic matter (\dom) with a constant rate
$\lambda_{\phy,\dom}$, and dies with a constant rate
$\lambda_{\phy,\det}^{surf}$, both until a minimum phytoplankton concentration
of  $2\, \phy_0$ is reached. Dead phytoplankton forms detritus. Thus, the time derivative
for phytoplankton is 
\begin{eqnarray}
\label{eq_phy_sur}
\frac{\Delta\phy}{\Delta t} & = & phosy - grazing -
\left(\lambda_{\phy,\det}^{surf}+\lambda_{\phy,\dom}\right) 
\left( \phy - 2 \,\phy_0 \right)
\end{eqnarray}

\paragraph{Zooplankton} Phytoplankton is grazed on by zooplankton (\zoo).
The functional response follows a Monod function, reduced by a minimum
phytoplankton concentration $\phy_0$ for the onset of grazing:

\begin{equation}
\label{eq_grazing}
graze  =  \zoo \, \mu_{\zoo} \frac {\phy-\phy_0}{K_\zoo + \phy} 
\end{equation}

To avoid negative phytoplankton concentrations due to the constant
forward time stepping, zooplankton grazing is calculated semi-implicitely:
\begin{eqnarray}
\phy^{t+\Delta t} & = & \phy^t + phosy - \zoo^t \, \mu_\zoo \, 
\frac {(\phy^{t+\Delta t}-\phy_0)} 
{K_{\zoo} + \phy^t} \, \Delta t \nonumber \\
 & = & \frac{\phy^t + phosy + \frac{\phy_0 \, \zoo^t \, \mu_\zoo
\Delta t}{K_\zoo + \phy^t}}
{1+\frac{\zoo^t \, \mu_\zoo \,\Delta t}{K_{\zoo}+\phy^t}} 
\end{eqnarray}

Thus, the actual amount of phytoplankton consumed per day is given by 

\begin{equation}
grazing  =  \frac {\phy^t + phosy - \phy^{t+\Delta t}}{\Delta t}
\end{equation}


Only a fraction of the grazed phytoplankton, $\epsilon_\zoo$, is ingested by
zooplankton (equation \ref{eq_zoo_sur}), the remaining fraction is immediately egested as
fecal pellets. Of the fraction ingested, a further fraction,
$\omega_{graz,\zoo}$, leads to zooplankton growth; the remainder is excreted
as nutrients (grazing related metabolism). Zooplankton further has a
basal metabolism, given by the constant rate $\lambda_{\zoo,\dom}$, until a
mininimum zooplankton concentration, $2\,\zoo_0$ is reached. Zooplankton
predators (see SMR96 for further details) are parameterized as a constant death rate
of zooplankton, $\lambda_{\zoo}^{surf}$, down to a minimum concentration of $2\,\zoo_0$.
Dead zooplankton is either immediately remineralized to phosphate (fraction
given by $\omega_{mort,\pho}$), or forms detritus.
\begin{eqnarray}
\label{eq_zoo_sur}
\frac{\Delta\zoo}{\Delta t} & = & grazing \, \epsilon_{\zoo} \,\omega_{graz,\zoo}
- \left( \lambda_{\zoo,\dom}+\lambda_{\zoo}^{surf}\right) \left( \zoo - 2\,\zoo_0 \right)
\end{eqnarray}

\paragraph{Detritus} Detritus (\det) is formed from dead phytoplankton and
zooplankton, and zooplankton fecal pellets (equation \ref{eq_det_sur}). In
contrast to SMR96, this flux is not immediately exported to deeper layers, but
the sinking of detritus is simulated explicitely (see later section about sinking).
There is no remineralisation of detritus in the surface layers.
\begin{eqnarray}
\label{eq_det_sur}
\frac{\Delta \det}{\Delta t} & = & \lambda_{\phy,\det}^{surf} \left( \phy - 2 \,\phy_0 \right)
+grazing \, (1-\epsilon_{\zoo}) \\ \nonumber
& & + \lambda_{\zoo}^{surf} \left( \zoo - 2\,\zoo_0 \right)
(1-\omega_{mort,\pho})
\end{eqnarray}

\paragraph{Dissolved organic matter} Dissolved organic matter (\dom) is produced
by phytoplankton exudation and  zooplankton excretion. It remineralizes to 
phosphate at a constant rate, $\lambda_{\dom,\pho}^{surf}$.
\begin{eqnarray} \label{eq_dom_sur} \frac{\Delta \dom}{\Delta t} & = &
\lambda_{\phy,\dom} \left( \phy - 2 \,\phy_0 \right) +\lambda_{\zoo,\dom}
\left( \zoo - 2\,\zoo_0 \right) \\ \nonumber & & - \lambda_{\dom,\pho}^{surf}
\, \dom \end{eqnarray}

\paragraph{Phosphate} Photosynthesis reduces the phosphate concentration (\pho), 
zooplankton excretion and DOM remineralization increase it:
\begin{eqnarray}
\label{eq_pho_sur}
\frac{\Delta \pho}{\Delta t} & = & - phosy + grazing \, \epsilon_{\zoo} 
(1-\omega_{graz,\zoo}) \\ \nonumber
& & + \lambda_{\zoo}^{surf} \left( \zoo - 2\,\zoo_0 \right) \omega_{mort,\pho}
+ \lambda_{\dom,\pho}^{surf} \, \dom
\end{eqnarray}

Nitrate (\nit) dynamics in the surface layer 
simply follow the phosphate dynamics, multiplied by a constant
stoichiometric ratio, $R_{N:P}$. There is no variation of the internal N:P
ratio of particulate or dissolved constituents from  surface layer processes
(but see later sections about denitrification and N$_2$ fixation).
\begin{eqnarray}
\label{eq_nit_sur}
\frac{\Delta \nit}{\Delta t} & = & \frac{\Delta \pho}{\Delta t} \, R_{N:P} 
\end{eqnarray}

\paragraph{Silicate and opal}  It is assumed that phytoplankton consists of
diatoms, coccolithophorids, and flagellates. It is further assumed, that
diatoms grow fastest, i.e., if silicate is available,  phytoplankton 
growth by diatoms ($delsil$) is first computed. 

In the default model set-up only the shells (opal and calcium carbonate) that are part
of detritus (i.e., that have been grazed and egested as fecal pellets or are
contained in dead zooplankton and phytoplankton) are
accounted for, whereas the living parts do not sink but sooner or
later undergo dissolution:
\begin{eqnarray}
\label{eq_delsilnoagg}
delsil & = & \min\left( \frac{\Delta \det}{\Delta t}\, R_{Si:P} \frac{\sio}
{K_{\phy}^{\sio} + \sio}
, 0.5 \, \sio \right)
\end{eqnarray}

$R_{Si:P}$ denotes the Si:P ratio required by diatoms, $K_{\phy}^{\sio}$ the
half-saturation constant for silicate uptake.   

If option {\tt -DAGG} is given during compilation, also the living cells
sink, and we have to consider them as part of the opal pool, as they 
might sink before dissolution.
Thus, opal production is determined by photosynthesis:
(eqn. 14)

\begin{eqnarray}
\label{eq_delsilagg}
delsil & = & \min\left(phosy \, R_{Si:P} \frac{\sio}{K_{\phy}^{\sio} + \sio}
, 0.5 \, \sio \right)
\end{eqnarray}

Opal (\opal) production by diatoms consumes silicate (\sio). Opal itself
dissolves with a constant ratio, $\lambda_{\opal,\sio}$. Thus, 
\begin{eqnarray}
\label{eq_sio_sur}
\frac{\Delta \sio}{\Delta t} & = & -delsil + \lambda_{\opal,\sio} \, \opal
\end{eqnarray}
\begin{eqnarray}
\label{eq_opal_sur}
\frac{\Delta \opal}{\Delta t} & = & + delsil - \lambda_{\opal,\sio} \, \opal
\end{eqnarray}

\paragraph{Dissolved inorganic carbon, calcium carbonate production and
alkalinity}  As mentioned above, it is assumed that diatoms grow fastest of all
groups, i.e., a certain fraction of photosynthesis or detritus production is
associated with opaline shell formation. The remaining fraction of photosynthesis
is by coccolithophorides. Again, as for opal in the default case
we only account for the sinking part of calcite production, i.e., the 
detritus production:
%(eqn. 17)

\begin{eqnarray}
\label{eq_delcarnoagg}
delcar & = & R_{Ca:P} \frac{\Delta \det}{\Delta t}\,\frac{K_{\phy}^{\sio}}
{K_{\phy}^{\sio}+\sio} 
\end{eqnarray}
%stimmt das?

If option {\tt -DAGG} is given during compilation alive coccolithophorid
are also subject to sinking.  Production of calcaerous shells is thus 
tied to photosynthesis, but
only up  to a limited fraction (M\_{CaCO3}) of total photosynthesis.
%(eqn 18)


\begin{equation}
\label{eq_delcaragg}
delcar = R_{Ca:P} \min\left(M_\cal \, phosy, phosy-\frac{delsil}{R_{Si:P}} \right)
\end{equation}

Thus, the production rate of calcium carbonate is:
\begin{eqnarray}
\label{eq_cal_sur}
\frac{\Delta \cal}{\Delta t} & = & + delcar
\end{eqnarray}


The formation of calcium carbonate (\cal) shells 
consumes dissolved inorganic carbon and, for each mole of CaCO$_3$ formed,
it decreases the alkalinity by two. Furthermore, the
formation and degradation of particulates (phosphorous basis) cause 
associated changes in dissolved inorganic carbon (\car), given by the constant
stoichiometric ratio, $R_{C:P}$. The associated changes in nitrate
concentration lead to changes in alkalinity (\alk; an increase of one for each
mole of nitrate produced). 
\begin{eqnarray}
\label{eq_car_sur}
\frac{\Delta \car}{\Delta t} & = & \frac{\Delta \pho}{\Delta t} \, R_{C:P} - delcar
\end{eqnarray}
\begin{eqnarray}
\label{eq_alk_sur}
\frac{\Delta \alk}{\Delta t} & = & - \frac{\Delta \pho}{\Delta t}\, R_{N:P} - 2 \, delcar  
\end{eqnarray}

\paragraph{Oxygen} Photosynthesis releases oxygen, respiration consumes it. The
changes of the oxygen concentration are related to the changes in phosphate, multiplied by the
stoichiometric ratio $R_{-O2:P}$.
\begin{eqnarray}
\label{eq_oxy_sur}
\frac{\Delta \oxy}{\Delta t} & = & - \frac{\Delta \pho}{\Delta t} \, R_{-O2:P} 
\end{eqnarray}

\paragraph{Dissolved iron}   The model includes iron limitation of
photosynthesis (identical for all phytoplankton groups), 
as well as iron release during remineralisation and iron
complexation by organic substances.  Dissolved, biologically available iron,
\fe,  released in the surface layer from the freshly deposited dust 
(see section \ref{dust_input}) is taken up during
photosynthesis, and released during remineralisation using a fixed Fe:P ratio
$R_{Fe:P}$.  Applying the model of Johnson \etal (1997)\nocite{johnson:1997},
there is a relaxation of iron to values of 0.6 nmol L$^{-1}$ ($\fe_0$) using a
relaxation time constant $\lambda_{\fe}$. This approach assumes that dissolved
iron beyond this limit is complexed by strong iron binding ligands, and 
therefor is lost for the biogeochemical cycle.

\begin{equation}
\frac{\Delta  \fe}{\Delta  t} = \frac{\Delta  \pho}{\Delta  t}\, R_{Fe:P} - \lambda_\fe \, \max \left(0,\fe-\fe_0 \right)
\end{equation}

\paragraph{DMS} The DMS production in the model depends on the growth of
diatoms and coccolithophorids, modified by temperature ($T$). DMS decrease
depends on temperature and irradiance ($I$). DMS is simulated using 6 parameters,
$D_1 - D_6$. DMS growth only takes place in the euphotic zone, i.e., in the upper few
layers of the water column. DMS decrease takes place in the total water column.
\begin{eqnarray}
\frac{\Delta  \dms}{\Delta  t} & = & \left( D_5 \, delsil + D_4 \, delcar \right)
            \, \left( 1+\frac{1}{(T+D_1)^2} \right) \\
        & &     - D_2 \, 8 \, I \,\dms \nonumber
             -  D_3\, |T+3| \,\dms \, \frac{\dms}{D_6 + \dms} \nonumber
\end{eqnarray}

\subsubsection{\label{deep_aerobic_remineralization}Deep water aerobic
remineralization  ({\tt ocprod.f90})} 

Below the maximum depth of primary production ($\approx$ 100 m, depth of layer $k_{eu}$) 
mortality of the living components of phytoplankton and zooplankton is simulated with constant rates
$\lambda_{\phy,\det}^{deep}$ and $\lambda_{\zoo,\det}^{deep}$, respectively. Detritus and DOM
are remineralized to phosphate with constant rates $\lambda_{\det,\pho}^{deep}$ and
$\lambda_{\dom,\pho}^{deep}$, respectively. These
processes take place only if sufficient oxygen is available
since remineralization of detritus and DOM
requires oxygen for the respiration of the associated carbon:
\begin{eqnarray}
remin^{aerob}_\det& = & \min\left(\lambda_{\det,\pho}^{deep} \det
, \frac{0.5\,\oxy}{R_{-O2:P}\,\Delta t}  \right) \nonumber \\
remin^{aerob}_\dom& = &  \min\left(\lambda_{\dom,\pho}^{deep} \dom
, \frac{0.5\,\oxy}{R_{-O2:P}\,\Delta t}  \right) \nonumber \\
\frac{\Delta \phy}{\Delta t} & = & -\lambda_{\phy,\det}^{deep} \max\left(0, \phy - \phy_0 \right)\\
\frac{\Delta \zoo}{\Delta t} & = & -\lambda_{\zoo,\det}^{deep} \max\left(0, \zoo - \zoo_0 \right)\\
\label{eq_det_aerob}\frac{\Delta \det}{\Delta t} & = & 
+ \lambda_{\phy,\det}^{deep} \max\left(0, \phy - \phy_0 \right)\\ \nonumber
& &+ \lambda_{\zoo,\det}^{deep} \max\left(0, \zoo - \zoo_0 \right) 
- remin^{aerob}_\det\\
\frac{\Delta \dom}{\Delta t} & = & - remin^{aerob}_\dom\\
\frac{\Delta \pho}{\Delta t} & = & remin^{aerob}_\det + remin^{aerob}_\dom
\end{eqnarray}

The changes of the other (non-phosphorous) components, again, are calculated from
the changes in phosphate multiplied by the respective stoichiometric ratio. 
\begin{eqnarray}
\frac{\Delta \nit}{\Delta t} & = & \frac{\Delta \pho}{\Delta t} \, R_{N:P}\\
\frac{\Delta \car}{\Delta t} & = & \frac{\Delta \pho}{\Delta t} \, R_{C:P}\\
\frac{\Delta \alk}{\Delta t} & = & - \frac{\Delta \pho}{\Delta t} \, R_{N:P}\\
\frac{\Delta \oxy}{\Delta t} & = & - \frac{\Delta \pho}{\Delta t} \, R_{-O2:P}
\end{eqnarray}

Opal dissolution increases deep ocean silicate concentrations as
\begin{eqnarray}
\frac{\Delta \opal}{\Delta  t} & = & -\lambda_{\opal,\sio} \, \opal\\
\frac{\Delta \sio}{\Delta  t} & = & +\lambda_{\opal,\sio} \, \opal
\end{eqnarray}

Calcium carbonate dissolution and its effect on alkalinity will be described
later together with the carbonate cycle.

If sufficient oxygen is available, a fraction of \ntwo is oxidized to
\ntwoo, depending on the oxygen undersaturation, $Sat_{O2} - \oxy$:
\begin{eqnarray}
oxidize & = & 0.0001 \, 
\left\{ \begin{array}{r@{\quad:\quad}l} 
1 & Sat_{O2} - \oxy < 1.97 \times 10^{-4}\\ 
4 & Sat_{O2} - \oxy \ge 1.97 \times 10^{-4}  
\end{array} \right. \nonumber \\
\frac{\Delta \ntwo}{\Delta  t} & = &  - 
\left(remin^{aerob}_\det+remin^{aerob}_\dom\right)\,R_{-O2:P}\,oxidize\\
\frac{\Delta \ntwoo}{\Delta  t} & = &  
\left(remin^{aerob}_\det+remin^{aerob}_\dom\right)\,R_{-O2:P}\,oxidize
\end{eqnarray}

Finally,  associated with the changes in deep phosphate, dissolved iron
is released during remineralisation, and, as for the surface layers, there is
complexation of dissolved iron:

\begin{equation}
\frac{\Delta  \fe}{\Delta  t} = \frac{\Delta  \pho}{\Delta  t}\, R_{Fe:P} - \lambda_\fe \, \max \left(0,\fe-\fe_0 \right)
\end{equation}


%1+3*1 for aou > 1.97e-4
%1+3*0 for aou < 1.97e-4

\clearpage

\subsubsection{\label{deep_anaerobic_remineralization}Deep water anaerobic
remineralization and denitrification  ({\tt ocprod.f90})} 

Even in the absence of sufficient oxygen there is still remineralization:
bacteria (which are not modelled explicitely) take  nitrate as
final electron acceptor, and nitrate is reduced to N$_2$. So there is a decrease in
nitrate, because of its reduction. On the other hand there is a gain in nitrate
because of remineralization of organic matter since
detritus contains nitrogen, which is released during remineralisation. The
intermediate step involving ammonium is neglected.  The oxygen from
two moles of nitrate corresponds to 3 moles of dissolved oxygen. Thus, the
loss of nitrate from oxidation is 2/3 times the remineralization under
aerobic conditions ($ 2/3 \, R_{-O2:P} \, remin$) and its gain is $
R_{N:P} \, remin$.  Again, the change in alkalinity is given by the changes in
the nitrate component, and the remineralisation of detritus is associated with
release of dissolved iron.
\begin{eqnarray}
remin^{anaerob1}_\det& = & 0.5 \, \lambda_{\det,\pho}^{deep} \,
\min\left(\det ,\frac{0.5\,\nit}{\frac{2}{3} R_{-O2:P}\, \Delta t}  \right)
\nonumber\\ 
\label{eq_det_anaerob1}\frac{\Delta \det}{\Delta  t} & = & - remin^{anaerob1}_\det\\
\frac{\Delta \nit}{\Delta  t} & = &  \left(R_{N:P} - \frac{2}{3} \, R_{-O2:P}\right) 
\, remin^{anaerob1}_\det \\
\frac{\Delta \ntwo}{\Delta  t} & = & \frac{1}{3} \,R_{-O2:P} \, remin^{anaerob1}_\det\\
\frac{\Delta \pho}{\Delta  t} & = & remin^{anaerob1}_\det\\
\frac{\Delta \car}{\Delta  t} & = & R_{C:P} \, remin^{anaerob1}_\det\\
\frac{\Delta \alk}{\Delta  t} & = & -R_{N:P} \, remin^{anaerob1}_\det\\
\frac{\Delta \fe}{\Delta  t} & = & +R_{Fe:P} \, remin^{anaerob1}_\det
\end{eqnarray}

Furthermore, detritus remineralizes using the oxygen contained
in \ntwoo:
\begin{eqnarray}
remin^{anaerob2}_\det& = & 0.01 
\min\left(\det ,\frac{0.003\,\ntwoo}{2\,R_{-O2:P}\, \Delta t}  \right)
\nonumber\\ 
\label{eq_det_anaerob2}\frac{\Delta \det}{\Delta  t} & = & 
- remin^{anaerob2}_\det\\
\frac{\Delta \nit}{\Delta  t} & = &  R_{N:P} \, remin^{anaerob2}_\det \\
\frac{\Delta \ntwoo}{\Delta  t} & = & - 2 \,R_{-O2:P} \, remin^{anaerob2}_\det\\
\frac{\Delta \ntwo}{\Delta  t} & = & 2 \,R_{-O2:P} \, remin^{anaerob2}_\det\\
\frac{\Delta \pho}{\Delta  t} & = & remin^{anaerob2}_\det\\
\frac{\Delta \car}{\Delta  t} & = & R_{C:P} \, remin^{anaerob2}_\det\\
\frac{\Delta \alk}{\Delta  t} & = & -R_{N:P} \, remin^{anaerob2}_\det\\
\frac{\Delta \fe}{\Delta  t} & = & +R_{Fe:P} \, remin^{anaerob2}_\det
\end{eqnarray}


\subsubsection{\label{dic_chemistry}Dissolved inorganic carbon chemistry and
calcium carbonate dissolution ({\tt carchm.f90})}

The treatment of carbon chemistry is similar to the one described in 
Maier-Reimer and Hasselmann  (1987\nocite{maier-reimer:1987}; see also Heinze
and Maier-Reimer, 1999b\nocite{heinze:1999b}), and is described only briefly here.
The model explicitely simulates total dissolved inorganic carbon ($C_T$),
and total alkalinity ($A_T$),  defined as 
\begin{eqnarray}
\label{eq_ct}
\left[C_T\right] & = & \left[H_2CO_3\right] + \left[HCO_3^-\right] + \left[CO_3^{2-}\right] \\
\label{eq_at}
\left[A_T\right] & = &  \left[HCO_3^-\right] + 2\,\left[CO_3^{2-}\right] + \left[B(OH)_4^-\right] + \left[OH^-\right] - \left[H^+\right] 
\end{eqnarray}

Changes in total carbon concentration and alkalinity due to biogeochemical
processes have been described above. Changes due to sea-air gas exchange (see
section \ref{air_sea_gas_exchange}) and calcium carbonate dissolution depend on
surface layer pCO$_2$ and carbonate ion concentration, [CO$_3^{2-}$], which are
determined numerically from $C_T$ and $A_T$ as follows. 
The carbonate system is defined by the two dissociation steps from $H_2CO_3$ to
carbonate, $CO_3^{2-}$, and the borate buffer:
\begin{eqnarray}
H_2CO_3 & \rightleftharpoons & HCO_3^- + H^+ \\
HCO_3^- & \rightleftharpoons & CO_3^{2-} + H^+  \\
H_2O + B(OH)_3 & \rightleftharpoons & B(OH)_4^- + H^+ \\ 
H_2O & \rightleftharpoons & (OH)^- + H^+ 
\end{eqnarray}

with $B_T = \left[B(OH)_3\right] + \left[B(OH)_4^-\right] \propto S$, ($S = $
salinity) computed at every time step. The constants (see Table \ref{tab_carb_params}
for definitions)
\begin{eqnarray}
K_1& = & \frac{\left[HCO_3^-\right]\left[H^+\right]}{\left[H_2CO_3 \right]} \\
K_2& = & \frac{\left[CO_3^{2-}\right]\left[H^+\right]}{\left[HCO_3^- \right]}  \\
K_B& = & \frac{\left[B(OH)_4^-\right]\left[H^+\right]}{\left[B(OH)_3\right]} \\ 
K_W& = & \left[(OH)^-\right] \left[H^+\right] 
\end{eqnarray}

are computed from temperature and salinity at the first day of  each
month. At the end of each simulated year the values are stored. In the
following year these values are interpolated linearly in time to the 
current time step. 
%new emr
With
\begin{eqnarray}
\left[CO_3^{2-}\right]& = &\frac{\left[C_T\right]}{(1+\left[H^+\right]/K_1+\left[H^+\right]^2/(K_1K_2))} \\
\left[B(OH)_4^-\right]& = &\frac{B_T}{1+\left[H^+\right]/K_B} \\
\left[(OH)^-\right]& = &\frac{K_W}{\left[H^+\right]}
\end{eqnarray}
$[A_T]$ is a function of the local inventories of B and C atoms and $[H^+]$
as single unknown which is easily solved by a Newton iteration.

The model first inserts equation \ref{eq_ct} into equation \ref{eq_at} to solve
numerically for [H$^+$]. Then either pCO$_2$ (for air-sea gas exchange of
CO$_2$) or CO$_3^{2-}$ (for dissolution of calcium carbonate) is computed from
equation \ref{eq_ct}.

The dissolution of calcium carbonate depends on the CO$_3^{2-}$ undersaturation
of sea-water and a dissolution rate constant $\lambda_{\cal,\car}$.
Undersaturation is calculated from Ca$^{2+}$ concentration in sea water and the
apparent solubility product, $S_{Ca}$, of calcite :
\begin{eqnarray}
dissol & = & \min\left(\frac{U_{CO3}}{\Delta t},\lambda_{\cal,\car} \,\cal \right) \quad \mbox{with} \nonumber \\ 
U_{CO3} & = & \max\left(0, [Ca^{2+}]\,S_{Ca} - CO_3^{2-}\right)  \nonumber \\ 
\frac{\Delta  \cal}{\Delta  t}& = &  - dissol\\
\frac{\Delta  \car}{\Delta  t}& = &  dissol\\
\frac{\Delta  \alk}{\Delta  t}& = &  2 \, dissol
\end{eqnarray}

%table3
\begin{table}[hbt]
\caption{\label{tab_carb_params} Variables and parameters for calcium carbonate
dissolution, defined in {\tt chemcon.f90} and  declared and communicated
between the modules in {\tt mo\_carbch.f90},  except: $^1$ defined in {\tt
beleg\_bgc.f90}, declared and passed on by module {\tt mo\_biomod.f90}. 
See description of module {\tt chemcon.f90} for their evolution in time.} 
\vspace{.2cm}
\begin{center}
\begin{tabular}{llll} \hline
symbol & variable        & name     & units\\ \hline
\multicolumn{4}{l}{\rule{0mm}{4mm}\bf diagnostic variables solved for
numerically}\\ \hline
\hi      & proton concentration   & {\tt hi(ie,je,ke)}  &\\
\cartwo  & CO$_3^{2-}$            & {\tt co3(ie,je,ke)} & \\
\multicolumn{4}{l}{\rule{0mm}{4mm}\bf constants for subsurface layers/calcium
carbonate dissolution}\\ \hline
$K_1$    &$K_1$ of carbonic acid  & {\tt ak13(ie,je,ke)} & \\
$K_2$    &$K_2$ of carbonic acid  & {\tt ak23(ie,je,ke)} & \\
$K_B$    &$K$ of boric acid       & {\tt akb3(ie,je,ke)} & \\
$K_W$    &ionic product of water  & {\tt akw3(ie,je,ke)} & \\
$B_T$    &total borat concentration$^1$  & {\tt rrrcl} & \\
$S_{Ca}$ &solubility product of calcite  & {\tt aksp(ie,je,ke)} & \\ 
\multicolumn{4}{l}{\rule{0mm}{4mm}\bf constants for $^{14}$C decay and air-sea gas
exhange}\\ \hline
$R^{atm}$ & $^{14}$C:$^{12}$C in the atmosphere & {\tt Ratm} & \\ 
$\lambda_{\cariso,\car}$ & decay constant for $^{14}$C to $^{12}$C & {\tt c14dec} & $d^{-1}$\\ \hline
\end{tabular}
\end{center}
\end{table}

In addition to the stable \car the model also simulates instable \cariso, which is
present at a  fixed ratio of \car in the atmosphere, and subject to decay in
the ocean (resulting in varying ratios of oceanic  $^{14}$C:$^{12}$C). \cariso
in the model does not interact with the other biogeochemical tracers, i.e. it
is not taken up during photosynthesis. Thus, the change of
\cariso in the water column due to local processes is:
\begin{equation}
\frac{\Delta  \cariso}{\Delta  t} =   \cariso \, \lambda_{\cariso,\car}
\end{equation}


\subsubsection{\label{anthropogenic_tracers}Anthropogenic carbon ({\tt
ocprod.f90},  {\tt carchm\_ant.f90})}

If key {\tt -DPANTHROPOCO2} is set before compilation, the model
computes the natural (preindustrial) plus any anthropogenic carbon (\Acar) 
and the associated tracers alkalinity (\Aalk)  and
calcium carbonate (\Acal). Changes by biogeochemical and physical
processes are computed as for natural carbon, alkalinity and calcium carbonate.

If key {\tt -DDIFFAT} is set before compilation, the model explicitely accounts for
the various CO$_2$ sources over the continents (see below in section
\ref{atmospheric_tracers}).

\subsection{\label{interactions_with_the_atmosphere}Interactions with the
atmosphere}

The water column model interacts with atmospheric components by air-sea
exchange of gaseous tracers (O$_2$, N$_2$, DMS and CO$_2$), dinitrogen (N$_2$)
fixation by diazotrophs at the sea-surface, and dust flux from the
atmosphere to the ocean. If option {\tt
-DDIFFAT} has been given during compilation, the model in addition accounts for
transport of atmospheric tracers parameterized as mixing.

\subsubsection{\label{n_fixation}N-fixation ({\tt cyano.f90})}

There is no explicit simulation of blue-green algae (diazotrophs). Instead,
nitrogen fixation is parameterized as the relaxation of surface layer
deviation of the N:P ratio of nutrients in the following way: It is assumed
that the water of the surface layer is always repleted with respect to
atmospheric nitrogen (N$_2$). If there is more phosphate than nitrate divided
by the stoichiometric ratio, algae take up atmospheric nitrogen which is
recycled immediately to nitrate. Thus, N-fixation is parameterized as
\begin{equation}
\frac{\Delta \nit}{\Delta  t}  =  \mu_{NFix} \, \max\left(0,\pho\,R_{N:P} - \nit\right)
\end{equation}

\subsubsection{\label{dust_input}Dust input and iron release ({\tt ocprod.f90})}

Monthly mean values of dust deposition ($Dep_d$) at the ocean surface are taken
from Timmreck \etal (Timmreck and Schulz, 2004)\nocite{timmreck:2004}. Alternatively,
other fields of atmospheric dust deposition may be used to drive the model. 
In \ham it is assumed that all dust particles (\fdust) have the same diameter ($l_d$), and sinking
speed ($w_\fdust$) which is computed from the diameter. This is a
simplification - in reality, the smallest dust particles would be transported much
further, and thus the size distribution and sinking rates of 
dust particles would change with distance from the source.

\begin{equation}
\frac{\Delta  \fdust}{\Delta  t} = \frac{Dep_d}{\Delta z_1^t}
\end{equation}

where $\Delta z_1^t$ denotes the depth of the surface layer. Dust is 
treated as chemically inert and loss occurs only by sinking and sedimentation
(see section 2.3).

Dissolved iron is released from the dust immediately after deposition in 
the surface layer assuming a fixed weight percentage of iron and a constant
solubility:
\begin{equation}
\frac{\Delta  \fe}{\Delta  t} = \frac{Dep_d}{\Delta z_1^t}\times \epsilon_{Fe}\times S_{Fe}
\end{equation}

\subsubsection{\label{air_sea_gas_exchange}Air-sea gas exchange of O$_2$,
N$_2$, DMS and CO$_2$  ({\tt carchm.f90})}

Air sea gas exchange of $O_2$, $CO_2$ and $N_2$ in the default case is computed
assuming constant atmospheric concentrations of the atmospheric tracers. If
option {\tt -DDIFFAT} is given during compilation,  the change of atmospheric
concentrations due to air-sea gas exchange is accounted for, and a (rather simple) 
atmospheric transport model (see below) computes the distribution of the atmospheric tracers
$O^{atm}_2$, $N^{atm}_2$, and $CO^{atm}_2$.  

%table5
\begin{table}[hbt]
\caption{\label{tab_air_sea_params} Variables and parameters for air-sea gas
exchange, defined in {\tt chemcon.f90} and  declared and communicated between
the modules in {\tt mo\_carbch.f90},  except: $^1$ defined in {\tt
beleg\_bgc.f90}, declared and communicated by module {\tt mo\_biomod.f90}. Note
that constants {\tt chemcm} for the surface layer are stored in fields over
latitude and longitude and for 12 months at the end of each year.
See description of module {\tt chemcon.f90} for their evolution with time.} 
\vspace{.2cm}
\begin{center}
\begin{tabular}{llll} \hline
symbol & meaning        & code     & units \\ \hline
\multicolumn{4}{l}{\rule{0mm}{4mm}\bf non-advected tracers solved for
numerically}\\ \hline
\hi      & proton concentration   & {\tt hi(ie,je,ke)}  &\\
\cartwo  & CO$_3^{2-}$            & {\tt co3(ie,je,ke)} & \\
\multicolumn{4}{l}{\rule{0mm}{4mm}\bf constants for air-sea gas exchange}\\ \hline
$K_1$    &$K_1$ of carbonic acid & {\tt chemcm(ie,je,4,12)} & \\
$K_2$    &$K_2$ of carbonic acid & {\tt chemcm(ie,je,3,12)} & \\
$K_B$    &$K$ of boric acid   & {\tt chemcm(ie,je,1,12)} & \\
$K_W$    &ionic product of water  & {\tt chemcm(ie,je,2,12)} & \\
$B_T$    &borat concentration$^1$  & {\tt rrrcl} & \\
$S_{CO2}$&solubility of $CO_2$ in seawater & {\tt chemcm(ie,je,5,12)} & \\
$S_{O2}$ &solubility of O$_2$ in seawater & {\tt chemcm(ie,je,7,12)} & \\
$S_{N2O}$ &solubility of N$_2$O in seawater & {\tt chemcm(ie,je,8,12)} & \\
$V_\oxy$ &relaxation constant for surface O$_2$ saturation & {\tt oxyex} & \\
$V_\ntwo$ &sea-air gas exchange rate for N$_2$ & {\tt an2ex} & \\
$V_\dms$ &sea-air gas exchange rate  for DMS & {\tt dmsex} & \\ \hline
\end{tabular}
\end{center}
\end{table}

All air-sea gas exchange rates, $V_{X}$,  (where $X$ stands for the respective
gas) are calculated using the Schmidt number and piston velocity according to
Wanninkhof (1992)\nocite{wanninkhof:1992}. Gas exchange is
divided by the actual thickness of the surface layer, $\Delta z^t_1$.

\paragraph{O$_2$} Oxygen solubility, $S_\oxy$,  is calculated from temperature
and salinity at the start of each month, according to Weiss
(1970)\nocite{weiss:1970} and interpolated linearly as explained for the
dissociation constants above. The oxygen Schmidt number is
calculated after Keeling \etal (1998)\nocite{keeling:1998}. 

\begin{equation}
\frac{\Delta  \oxy}{\Delta  t} = \frac{-V_\oxy}{\Delta z^t_1}\, 
\left(\oxy-S_\oxy\, \frac{pO^{atm}_2}{196800}\right)
\end{equation}

Note that in the default case $pO_2^{atm} = 196800$, i.e., the atmosphere is
considered to remain on the preindustrial content of oxygen. Changes of the 
oxygen cycle can be diagnosed form the oceanic inventory.

\paragraph{N$_2$ and N$_2$O ({\tt carchm.f90})} \ntwo solubility , $S_\ntwo$, 
is calculated from temperature and salinity at the beginning of each month,
according to Weiss (1970)\nocite{weiss:1970} and interpolated linearly as
explained above. Solubility of laughing gas, \ntwoo, is assumed to be
constant in time. It is calculated at the beginning of each year from temperature
and salinity  according to Weiss (1974)\nocite{weiss:1974}. The \ntwo and \ntwoo
Schmidt number and piston  velocity are assumed to be the same as for oxygen.
\begin{eqnarray}
\frac{\Delta  \ntwo}{\Delta  t} & = & \frac{-V_\oxy}{\Delta z^t_1}\, 
\left(\ntwo-S_\ntwo\,\frac{pN^{atm}_2}{802000}\right)\\
\frac{\Delta  \ntwoo}{\Delta  t} & = & \frac{ -V_\oxy}{\Delta z^t_1}\, 
\left(\ntwoo-S_\ntwoo\right)
\end{eqnarray}

\paragraph{DMS} Air-sea gas exchange of \dms is calculated from
its Schmidt number and piston velocity following Saltzman (1993)
\nocite{saltzman:1993} assuming zero concentrations of atmospheric \dms:
\begin{equation}
\frac{\Delta  \dms}{\Delta  t} = \frac{-V_\dms}{\Delta z^t_1}\, \dms
\end{equation}


\paragraph{CO$_2$} Schmidt number and piston velocity of CO$_2$ are calculated
according to Wanninkhof (1992)\nocite{wanninkhof:1992}. From the CO$_2$, computed
as explained in the previous section, pCO$_2^{water}$ is computed as
pCO$_2^{water} =$ CO$_2 / $S$_{CO2}$, where S$_{CO2}$ is the solubility of
carbon dioxide. Solubility is calculated monthly according to Weiss
(1974)\nocite{weiss:1974},  and interpolated in time as explained
above.  In the present version, atmospheric pCO$_2^{atm}$ is either set to a 
constant value of 278 ppmv in the default case, or is derived from explicit calculation of
atmospheric concentration (option {\tt -DDIFFAT}).  Air-sea CO$_2$ flux is then
computed from the difference in partial pressure between atmosphere and water,
multiplied by gas exchange rate and solubility, V$_{CO2} \times$ S$_{CO2}$, and
divided by the thickness of the surface layer, $\Delta z^t_1$:

\begin{equation}
\frac{\Delta  \car}{\Delta  t} = \frac{V_{CO2}}{\Delta z^t_1}\,
S_{CO2}\,\left(pCO_2^{atm}-pCO_2^{water}\right)
\end{equation}
 
where $\Delta z_1^t$ denotes the actual thickness of the surface layer.  Finally,
the changes of oceanic \cariso are calculated similar to the changes of \car,
with the concentration of atmospheric \cariso calculated as a fixed fraction of
\car ($R_{atm})$: 
\begin{eqnarray}
R^{water} & = & \frac{\cariso}{\car} \nonumber \\
\frac{\Delta  \cariso}{\Delta  t} & = & \frac{V_{CO2}}{\Delta z^t_1}\,
S_{CO2}\,\left(R^{atm}\, pCO_2^{atm}-R^{water}\,pCO_2^{water}\right)
\end{eqnarray}
 

\subsubsection{\label{atmospheric_tracers}Computation of atmospheric 
O$_2$, N$_2$ and CO$_2$ ({\tt atmotr.f90})}

If key {\tt -DDIFFAT} is set before compilation, the model explicitely computes
the change in atmospheric tracer concentrations of CO$_2$, O$_2$, and N$_2$.

%table5
\begin{table}[hbt]
\caption{\label{tab_atmospheric_tracers} Gaseous tracers in the atmosphere, their
names in the text and the model code, and index {\tt ix} in the model's tracer field {\tt
atm(ie,je,ix)}.  The index is defined in {\tt mo\_param1\_bgc.f90},  and the
model's tracer field is declared in {\tt mo\_carbch.f90}.} 
\vspace{.2cm}
\begin{center}
\begin{tabular}{lllll} \hline
symbol & meaning        & {\tt name }     & {\tt ix} & unit\\ \hline
\car$^{atm}$ & CO$_2$              & {\tt iatmco2} & 1     &kmol C m$^{-2}$\\
\oxy$^{atm} $& O$_2$           &{\tt iatmo2}  & 2     &kmol O$_2$ m$^{-2}$\\
\ntwo$^{atm}$& N$_2$            & {\tt iatmn2} & 3     &kmol N$_2$ m$^{-2}$\\
\multicolumn{5}{l}{\rule{0mm}{4mm}{\it optional (}{\tt -DPANTHROPOCO2}{\it )}}\\ \hline
\Acar$^{atm}$ & anthropogenic CO$_2$& {\tt iantco2}    & 4   &kmol C m$^{-2}$\\ \hline
\end{tabular}
\end{center}
\end{table}

Processes that affect these tracers are the air-sea gas exchange and horizontal
mixing. The model atmosphere is defined on the same grid as the ocean model but has no vertical resolution. 
The gain (loss) of tracers is simply the flux out of (into)  the ocean:
\begin{eqnarray}
\frac{\Delta  \car^{atm}}{\Delta  t} & = & - V_{CO2}\,
S_{CO2}\,\left(pCO_2^{atm}-pCO_2^{water}\right)\\
\frac{\Delta  \oxy^{atm}}{\Delta  t} & = & V_\oxy\, 
\left(\oxy-S_\oxy\, \frac{pO^{atm}_2}{196800}\right)\\\
\frac{\Delta  \ntwo^{atm}}{\Delta  t} & = & V_\oxy\, \left(
\left(\ntwo-S_\ntwo\,\frac{pN^{atm}_2}{802000}\right) + 
 \left(\ntwoo-S_\ntwoo\right) \right)
\end{eqnarray}

Note that both oceanic \ntwo and \ntwoo contribute to the atmospheric
\ntwo$^{atm}$. Diffusion in $x$ and $y$ direction in the atmosphere is
computed in analogy to the oceanic diffusion over both oceanic and land 
grid points.

If key {\tt -DPANTHROPOCO2} is set, total annual mean emissions of anthropogenic CO$_2$ 
are distributed over the different continents in fixed proportions 
and homogenous in time (see also section \ref{io}).


\subsection{\label{sinking}Sinking of particles ({\tt ocprod.f90})}

The flux of particles through the water column redistributes phosphorous and
associated tracers in the vertical. Fluxes from the bottom layer in each
grid cell provide the boundary condition for the sediment module.

Presently, the model provides two different ways to assign sinking speeds to
the particles. In the default case particles have constant sinking
speeds, $w_\det$, $w_\cal$, $w_\opal$ and $w_\dust$ for \det, \cal, \opal, and
\fdust, respectively. In the second case (key {\tt -DAGG}) the model computes the aggregation
of marine snow and the associated change in sinking speed with depth and time
following Kriest (2002)\nocite{kriest:2002}. The second method requires two
additional state variables: one, \nos, for the number of aggregates, and another one, \adust,
for dust in aggregates.  Key {\tt -DAGG} automatically sets up the model for the 
additional state variables. 

%table6
\begin{table}[htb]
{\caption{\label{tab_agg_params}  Parameters of \ham that determine sinking and aggregation 
They are defined in {\tt beleg\_bgc.f90} and  communicated between the modules in {\tt
mo\_biomod.f90}.}}
\renewcommand{\baselinestretch}{1}
\vspace{.2cm}
\footnotesize
\begin{center}
\begin{tabular}{llp{7cm}l} \hline 
symbol & name & meaning & units \\ \hline
$l_d$&{\tt dustd1}& diameter of a dust particle& cm \\
\multicolumn{4}{l}{\rule{0mm}{4mm}{\it constant sinking: layers 1 to $(kb-1)$}}\\ \hline
$w_\det$&{\tt wpoc}& constant sinking speed of detritus  &\sink\\
$w_\cal$&{\tt wcal}& constant sinking speed of calcium carbonate &\sink\\
$w_\opal$&{\tt wopal}& constant sinking speed of opal &\sink\\ 
$w_\fdust$&{\tt wdust}& sinking speed of dust, computed from $l_d$ &\sink\\ 
\multicolumn{4}{l}{\rule{0mm}{4mm}{\it variable sinking and aggregation (option
{\tt -DAGG}): layers 1 to $kb$}}\\ \hline
$\eta$&{\tt SinkExp}& exponent of sinking speed-vs-diameter relationship & \\
$\zeta$&{\tt FractDim}& exponent of P-vs-diameter relationship & \\
$Stick$&{\tt Stick}& maximum stickiness & \\
$m_l$&{\tt cellmass}& minimum mass (P) of a particle& nmol P \\
$w_l$&{\tt cellsink}& minimum sinking speed of a particle& \sink\\
$l$&{\tt alow1}& minimum diameter of a particle& cm \\
$L$&{\tt alar1}& max. diameter for size dependent sinking and aggregation & cm \\
$zdis$&{\tt  zdis}& size distribution coefficient of dead zooplankton & \\ \hline
\end{tabular}
\end{center}
\renewcommand{\baselinestretch}{1.5}
\normalsize
\end{table}

\subsubsection{\label{sinking_constant}Sedimentation using a constant sinking rate}

The loss of particles by sedimentation for the upper box to the deepest grid cell is computed by
\begin{eqnarray}
\frac{\Delta \det}{\Delta  t} & = & w_\det \, \frac{\Delta\det}{\Delta z}\\
\frac{\Delta \cal}{\Delta  t} & = & w_\cal \, \frac{\Delta\cal}{\Delta z}\\
\frac{\Delta \opal}{\Delta  t} & = & w_\opal \, \frac{\Delta\opal}{\Delta z}\\
\frac{\Delta \fdust}{\Delta  t} & = & w_\fdust \, \frac{\Delta\fdust}{\Delta z}
\end{eqnarray}

using a downward implicit scheme. 

\subsubsection{\label{sinking_variable}Aggregation and sedimentation using
variable sinking rates}

Using the option {\tt -DAGG}, the model explicitely calculates the particle
size distribution of marine snow (phytoplankton + detritus), and its variations
in time and with depth caused by aggregation, sinking, and zooplankton mortality. The
approach follows the one presented in Kriest and Evans (2000)\nocite{kriest:2000},
modified by Kriest (2002)\nocite{kriest:2002}, and details for the aggregation
model can be found there. Here, only the modifications specific to \ham are
presented.

For the computation of aggregation it is necessary to consider an additional 
state variable, the number of marine snow aggregates (\nos). \nos represents
the number of marine snow particles, which consist of alive phytoplankton (\phy)
and detritus (\det). The approach assumes that the relation between particle
diameter and phosphorous content follows a power law with parameters $\zeta$
for the slope on a logarithmic scale, and $m_l$ for the phosphorous content of
the minimum size particles (see table \ref{tab_agg_params}). It is also assumed that the 
sinking speed of individual particles depends on their diameter to the power of
$\eta$, and a minimum sinking speed, $w_l$. It is further assumed that the size
distribution of particles can be approximated by a power law function. The
parameters of the latter (in particular its exponent $\epsilon$) can then be
computed from the number and mass of the aggregates at every time step and
location. 

Number of particles and mass of particles change independently. 
The number of particles changes due to aggregation (i.e. collision
and subsequent adherence of particles, term $\xi$ in equation
\ref{eq_nos_surface} and \ref{eq_nos_deep}; for further details see Kriest,
2002, and citations therein), and sinking of particles, which especially
removes the larger particles. Thus the relationship between particle numbers and mass
(e.g., the average particle size) changes with time and location, and so does the
mean sinking speed of the particle population.

\paragraph{Effects of biogeochemical processes  on particle size} Except for
zooplankton mortality neither of the biogeochemical processes that shift mass
between the dissolved phase and either phytoplankton or detritus (see eqns.
\ref{eq_phy_sur} and \ref{eq_det_sur} for surface, and eqns.
\ref{eq_det_aerob}, \ref{eq_det_anaerob1} and \ref{eq_det_anaerob2} for deep
ocean processes)  is assumed to change the particle size distribution.  
In contrast to former implementations  of the aggregation
model, in \ham it is assumed that dead zooplankton corpses (or the fecal
pellets of the predators) always add large particles to marine snow. This is
parameterized by a constant flat slope of the size distribution of this flux
from zooplankton to detritus in the following way: if the slope $\epsilon^*$ of
the size distribution of any process with mass flux $P$ is given by

\begin{equation}
\epsilon^* = \frac {(\zeta + 1) \, P - m_l \, \nos_P}{ P - m_l \, \nos_P},
\end{equation}

solving for  $\nos_P$ then gives the appropriate number of marine snow aggregates for any
desired slope of size distribution. In particular, if the slope is 
$\zeta + 1 + \epsilon_0$, then
\begin{equation}
\nos_P = P \, \frac {\epsilon_0}{ (\zeta+\epsilon_0)\, m_l}
= P \, zdis
\end{equation}

By choosing a very small $\epsilon_0$, a flat size distribution (a large
average size for particles) for the flux $P$ is parameterized. Thus for surface
water the number of dead zooplankton / predator fecal pellets is:
\begin{equation}
zmort =  \lambda_{\zoo}^{surf} \, (1-\omega_{mort,\pho}) \, \zoo \, zdis
\end{equation}

And for deep water:
\begin{equation}
zmort =  \lambda_{\zoo,\det}^{deep} \, \zoo \, zdis
\end{equation}

\paragraph{Aggregation} Aggregation ($\xi$) depends on the particle abundance,
their size distribution, rate of turbulent shear and the difference in particle
sinking speeds, and the stickiness (the probability that two particles  stick
together after contact). The approach implemented in the \ham follows the one
described in Kriest (2002; see there for term $\xi$ and its computation).
Currently it is assumed that turbulent shear is high in the upper $k_{eu}$ layers, and
zero below. Summing-up the number of collisions due to turbulent shear and differential
settlement, $C_{sh}$ and $C_{se}$, respectively, the decrease of the number of particles due to
aggregation then is

\begin{equation}
\xi = Stick^* \, \left( C_{sh} + C_{se} \right)
\end{equation}

While previous implementations of the aggregation model assumed a constant
stickiness e.g. for a phytoplankton community dominated by diatoms, in \ham,
because it explicitely considers  diatoms, coccolithophorids, and flagellates,
stickiness is supposed to vary. Although little is known about the stickiness
of the non-diatom groups, it seems that diatoms are the main contributors to
formation of marine snow. Thus, in the approach applied here, the actual stickiness
($Stick^*$) depends on the abundance of \opal, i.e. living or dead diatoms. If
there is no opal, stickiness is zero; if all of the phytoplankton and detrital
mass  is as diatoms, it is half of its maximum value, $Stick$; if there is only
opal, but no organic matter (i.e. ``aged'' diatom detritus), its stickiness is
at its maximum value.

\begin{equation}
Stick^*=Stick\,\frac{\frac{\opal}{R_{Si:P}}}{\frac{\opal}{R_{Si:P}}+\phy+\det}
\end{equation} 

Thus, stickiness varies with the (present and past) phytoplankton community
composition, and with the ``age'' of sinking matter. 

\paragraph{Sinking rates of number and mass of aggregates} Sinking of particles with
regard to their number and mass is
simulated as described in Kriest and Evans (2000) and Kriest (2002) (see there
for term $\Phi$ for number sinking and  term $\Psi$ for mass sinking). In
particular, assuming that the relationship between diameter and mass of a
particle can be described by a power law defined by the parameters $\eta$ and
$w_l$ (see table \ref{tab_agg_params}), the {\em average} sinking rates of
numbers ($\bar{w}_{\nos}$) and mass ($\bar{w}_{\phy}=\bar{w}_{\det}$) can be 
computed  from
\begin{eqnarray}
\bar{w}_{\phy} & = & \bar{w}_{\det} = w_l \, \frac{\zeta+1-\epsilon
+\left(\frac{L}{l}\right)^{1+\eta+\zeta-\epsilon}\,\eta}{1+\zeta+\eta-\epsilon}\\
\bar{w}_{\nos} & = & w_l \, \frac{1-\epsilon
+\left(\frac{L}{l}\right)^{1+\eta-\epsilon}\,\eta}{1+\eta-\epsilon}
\end{eqnarray}

\paragraph{Equation for the number of aggregates}  Combining the effect of biogeochemical processes,
aggregation and sedimentation, the time derivative for the number of marine snow
aggregates in the euphotic zone is 
\begin{eqnarray}
\label{eq_nos_surface}
\frac{\Delta \nos}{\Delta  t} & = & 
 zmort - \xi + \frac{\Delta \left(\bar{w}_{\nos} \nos\right)}{\Delta z}\\
&&+ \left( phosy - \left(\phy-2\,\phy_0\right) \lambda_{\phy,\dom}
- grazing \, \epsilon_\zoo 
\right) \, \frac{\nos}{\phy+\det} \nonumber 
\end{eqnarray}

and for deep water 
\begin{eqnarray}
\label{eq_nos_deep}
\frac{\Delta \nos}{\Delta  t} & = & 
 zmort - \xi + \frac{\Delta \left(\bar{w}_{\nos} \nos\right)}{\Delta z}\\
&& - \left( remin^{aerob}_\det+remin^{anaerob}_\det \right) \, \frac{\nos}{\phy+\det} \nonumber 
\end{eqnarray}


\paragraph{Sedimentation of mass} No attempt has been made to change the
sinking speed with respect to the amount of calcium carbonate or opal in the
particles. Assuming that opal and calcium carbonate shells sink with the same
rate as organic matter ($\bar{w}_{\opal} = \bar{w}_{\cal} = \bar{w}_{\phy}$)
the sedimentation of organic matter then reads:
\begin{eqnarray}
\frac{\Delta \phy}{\Delta  t} & = & \frac{\Delta \left(\bar{w}_{\phy} \phy\right)}{\Delta z}\\
\frac{\Delta \det}{\Delta  t} & = & \frac{\Delta \left(\bar{w}_{\det} \det\right)}{\Delta z}\\
\frac{\Delta \opal}{\Delta  t} & = & \frac{\Delta \left(\bar{w}_{\opal} \opal\right)}{\Delta z}\\
\frac{\Delta \cal}{\Delta  t} & = & \frac{\Delta \left(\bar{w}_{\cal} \cal\right)}{\Delta z}
\end{eqnarray}

Sedimentation is calculated using an upward explicit scheme. At the beginning
of each model run the possible maximum sinking speed of particles is checked
against the minimum depth of the boxes and the model time step to avoid
overshoots during simulation.  However, during simulation the depth of the
surface ocean layer may decrease due to ice coverage, evaporation etc. To avoid
overshoots in this layer, the model first performs a check with respect to
sinking rate vs. layer thickness by time step; in case of possible overshoots, it
decreases the maximum length for size dependent sinking and aggregation, $L$,
such that the sinking speed will not exceed layer thickness / time step length.

\paragraph{Aggregation and sedimentation of dust} Free dust (\fdust) enters the oceanic
surface layer as single particles. It is assumed that single dust
particles in the water do not aggregate with each other, but that they are
subject to aggregation with marine snow particles (of biogenic origin). To
simulate aggregation of free dust particles with marine snow, we follow
basically the approach by Kriest and Evans (2000; see description of marine
snow  aggregation), i.e. the mass of those dust particles aggregating with
marine snow (i.e. those that become aggregated dust) is goverened by the
stickiness of marine snow, its number concentration, its size distribution,
and concentration of free dust particles:
\begin{eqnarray}
\label{eq_dustagg}
\xi_d & = & \nos \, \fdust \, Stick^* \, \left( C_{d,sh} + C_{d,se} \right) \qquad \mbox{with}\\
C_{d,sh} & = & 0.163 \, shear \,  \left(1-\epsilon \right) \, \left[ 
 \left(l_d^3+3\,l_d^2\,L+3\,l_d\,L^2+L^3 \right)\,\frac{F_L}{1-\epsilon} \right. \\
& & - \left(    \frac{F_L-1   }{1-\epsilon}\,l_d^3
            +3\,\frac{F_L\,L-l}{2-\epsilon}\,l_d^2  
    + \left. 3\,\frac{F_L\,L^2-l^2}{3-\epsilon}\,l_d
+            \frac{ F_L\,L^3-l^3}{4-\epsilon} \right) \right]\quad \nonumber  \\
C_{d,se} & = &  0.125\,\pi\,w_l\,l_d^2\,
\left( \frac{1-\epsilon+\eta\,F_L\,F_S}{1+\eta-\epsilon}-\frac{w_\fdust}{w_l}\right) \quad 
\nonumber \\
F_L & = &  \left( \frac{L}{l}\right)^{1-\epsilon} \quad \mbox{and} \qquad
    F_S =  \left( \frac{L}{l}\right)^{\eta}\nonumber 
\end{eqnarray}

where $l$ and $L$ are the lower and upper boundary of marine snow for size
dependent aggregation, $w_l$ is the minimum sinking speed of marine snow. The
aggregation of free dust particles with marine snow leads to a (mass)  loss of
free dust and to a gain of aggregated dust. Aggregated dust then sinks with the
sinking speed of marine snow ($\bar{w}_\phy$). Thus, in this approach biogenic
particles act as a trigger to transport aeolian dust downwards in aggregated
form, while the single dust particles still sink with their slow, constant
sinking speed:
\begin{eqnarray}
\frac{\Delta  \fdust}{\Delta  t} & = & - \xi_d + w_\fdust \frac{\Delta \fdust}{\Delta z} \\
\frac{\Delta  \adust}{\Delta  t} & = &  \xi_d +  \frac{\Delta (\bar{w}_\phy \adust)}{\Delta z}   
\end{eqnarray}


\subsection{\label{sediment}The sediment}

The simulation of the oceanic sediment  is done basically in the same way as in
Heinze and Maier Reimer (1999)\nocite{heinze:1999b}, with slight differences to
account for the porosity of the sediment, and for the relationships between the
different tracers (P, N, C, O; see table \ref{tab_sedi_tracers}). Its upper
boundary condition is given by the export from the bottom ocean layer (input) and by
the nutrient concentration in the bottom layer (for water - sediment diffusive
exchange of dissolved constitutents). Simulated processes are decomposition of
detritus -both under aerobic and anaerobic conditions-  and
dissolution of opal and calcium carbonate, carbon chemistry, and the vertical diffusion
of porewater. The model is closed at the lower boundary with
respect to porewater diffusion.

%table7
\begin{table}[htb]
{\caption{\label{tab_sedi_tracers}  Biogeochemical tracers in the sediment
model, their names in the text and their index {\tt ix} in the model  tracer
fields {\tt powtra(ie,je,ks,ix)} ({\underline po}re {\underline wa}ter {\underline tra}cer) and  {\tt
sedlay(ie,je,ks,ix)}. The index is defined in {\tt mo\_param1\_bgc.f90}, and the
model tracer field is declared in {\tt mo\_carbch.f90}. Subscript $s$ stands for sediment.} }
\vspace{.2cm}
\renewcommand{\baselinestretch}{1}
\footnotesize
\begin{center}
\begin{tabular}{lllll} \hline
symbol & meaning & name & {\tt ix} & units \\ \hline
\multicolumn{3}{l}{\it state variables of the pore water ({\tt
powtra(ie,je,ks,ix)})}\\
$\car_s$  & dissolved inorganic carbon &{\tt ipowaic}  & 1 & \concC \\
$\alk_s$  & alkalinity        &{\tt ipowaal}  &  2 & \concalk \\
$\pho_s$  & phosphate         &{\tt ipowaph}  &  3 & \concP \\
$\oxy_s$  & oxygen            &{\tt ipowaox}  &  4 & \concO \\
$\ntwo_s$ & N$_2$             &{\tt ipown2}   &  5 & \concN \\
$\nit_s$  & nitrate           &{\tt ipowno3}  &  6 & \concN \\
$\sio_s$  & silicate          &{\tt ipowasi}  &  7 & \concSi \\ \hline
\multicolumn{3}{l}{\it state variables of the solid fraction ({\tt
sedlay(ie,je,ks,ix})}\\
$\det_s$  & detritus          &{\tt issso12}  & 1 & \concP \\
$\opal_s$ & opal              &{\tt issssil}  & 2 & \concSi \\
$\cal_s$  & calcium carbonate &{\tt isssc12}  & 3 & \concC \\ 
$\dust_s$  & clay (from free and aggregated dust) &{\tt issster}  & 4 & \concdust \\ \hline
\end{tabular}
\end{center}
\renewcommand{\baselinestretch}{1.5}
\normalsize
\end{table}

The sediment is resolved by 12 layers, with increasing thickness
and  decreasing porosity from top to bottom. It is assumed that the
porosity of the sediment remains constant over the integration time.
In the real ocean there is a continuous uplifting of the sediment 
surface with a typical velocity of 19 x $10^{-5}$ m/a. To maintain
the porosity profile and the higher resolution at the sediment-water 
interface in the model, particulate matter is shifted downwards. Upward
shifting occurs if the sediment porosity would fall below the prescribed value.
This could occur, e.g., when sea-ice forms at the surface and production of detritus 
in the water column stops. In this case there is no flux of particulate 
matter to the sediment while the dissolution or 
decomposition of particulate components in the sediment continuous.

Below the biologically active 12 layers there is additionally a diagenetically
consolidated layer ({\tt burial}), containing all the particulate tracers that
have been shifted downward. This layer acts as a source for upward shifting. In
case this layer is empty of all biogenic solid components, it is assumed that
there is an unlimited supply of clay from below. 

%table8
\begin{table}[htb]
{\caption{\label{tab_sedi_params}  Parameters of the sediment model, most of 
them declared in {\tt beleg\_bgc.f90} and {\tt bodensed.f90} and communicated 
between the modules in {\tt mo\_biomod.f90} and {\tt mo\_sedmnt.f90}.}}
\vspace{.2cm}
\renewcommand{\baselinestretch}{1}
\footnotesize
\begin{center}
\begin{tabular}{llp{5cm}l} \hline 
symbol & name & variable & units \\ \hline
\multicolumn{4}{l}{\rule{0mm}{4mm}{\bf 2D fields over }{\tt (ie,je)} {\bf for boundary
exchanges}}\\ \hline
$ib$ &{\tt kbo}& k-index field of oceanic bottom layer & \\
$\Delta z_{ib}$ &{\tt bolay}&  local thickness of oceanic bottom layer & m\\
\multicolumn{4}{l}{\rule{0mm}{4mm}\bf stoichiometry}\\ \hline
$R_{-O2:P}$ &{\tt ro2ut}& -O$_2$:P  ratio & mol O$_2$ mol P$^{-1}$\\
$R_{C:P}$&{\tt rcar}& C:P  ratio &mol C mol P$^{-1}$\\
$R_{N:P}$&{\tt rnit}& N:P  ratio &mol N mol P$^{-1}$\\ \hline
\multicolumn{4}{l}{\rule{0mm}{4mm}{\bf sediment thickness and porosity}}\\
 \hline
$\Delta z_S$&{\tt seddw}& sediment layer thickness & m\\
$\phi$ & {\tt porsol} & volume fraction of solid sediment &  \\ 
$1-\phi$  & {\tt porwat} & volume fraction of porewater &  \\ 
\multicolumn{4}{l}{\rule{0mm}{4mm}{\bf dissolution and diffusion} {\tt powach.f90}}\\
 \hline
- & {\tt sedict } & diffusion coefficient for porewater & m$^2$ d$^{-1}$\\
$\kappa^{sedi}_{\opal,\sio}$& disso &dissolution coefficient for opal&(\concSi)$^{-1}$ d$^{-1}$\\
$\kappa^{sedi}_{\det,\pho}$& - &remineralisation coefficient for detritus&(\concO)$^{-1}$ d$^{-1}$\\
$\kappa^{sedi}_{\cal,\car}$& - &dissolution coefficient for calcium carbonate&(\concC)$^{-1}$ d$^{-1}$\\
$[Ca^{2+}]$&{\tt calcon}& [$Ca^{2+}$] concentration & kmol m$^{-3}$\\
$\lambda^{sedi}_{\det,\pho}$& denit &decomposition coefficient for denitrification& d$^{-1}$\\
\multicolumn{4}{l}{\rule{0mm}{4mm}{\bf Sediment shifting} {\tt sedshi.f90}}\\ \hline
$W_{\det}$ & {\tt orgwei} & weight of one mole detritus-P & kg (kmol P)$^{-1}$ \\ 
$W_{\opal}$ & {\tt opalwei} & weight of one mole opal & kg (kmol Si)$^{-1}$  \\ 
$W_{\cal}$ & {\tt calcwei} & weight of one mole CaCO$_3$ & kg (kmol C)$^{-1}$  \\ 
$\rho_{\det}$ & {\tt orgdens} & density of organic carbon & kg m$^{-3}$ \\
$\rho_{\opal}$ & {\tt opaldens} & density of opal & kg m$^{-3}$ \\
$\rho_{\cal}$ & {\tt calcdens} & density of CaCO$_3$  & kg m$^{-3}$ \\
$\rho_{\clay}$ & {\tt claydens} & density of clay & kg m$^{-3}$ \\ \hline
\end{tabular}
\end{center}
\renewcommand{\baselinestretch}{1.5}
\normalsize
\end{table}


\subsubsection{\label{sediment_biogeochemistry}Sediment biogeochemistry ({\tt
powach.f90})}

The model simulates the decomposition of particulate matter (particulate
organic phosphorous, opal, and calcium carbonate) simultaneously with the diffusion
of pore water constituents (oxygen, silicate and dissolved inorganic carbon),
using a backward approach (see Heinze and Maier-Reimer, 1999). It further
simulates concomitant changes in phosphate, nitrate, alkalinity and their
diffusion, and also computes denitrification in the sediment. This section
first describes the general procedure for the simulation of decomposition and
diffusion. Details for the different tracers and processes are
described later. The following assumptions and notations are made:

\begin{enumerate}
\item The fraction of solid sediment is denoted by $\phi$, the fraction of pore
water is then $1 - \phi$. 
\item $\phi$ varies with depth, but not with time.
\item The concentration of the solid components $S$ relates to the volume of the solid
fraction in the sediment, and the concentration of pore water constituents
relates to the volume of pore water.
\item Given a solid component, $S$, (e.g., opal) it is assumed that its 
decomposition rate $\kappa \, X$ depends on its dissolved counterpart $X$ ($X = $
silicate undersaturation  for opal dissolution; $X = $ oxygen concentration for
particulate organic matter decomposition;  $X = $ carbonate ion concentration
for calcium carbonate dissolution). Thus, $\kappa$ has units of  d$^{-1}$ (mmol X
m$^{-3}$)$^{-1}$.
\item The diffusion term is here simply given as $D^*X_{zz}$
 with $D^*~5 \mathrm{x} 10^{-10}m^2s^{-1}$
\item Input to the sediment from above is given by term $Q$, and provided by the
water column model. The term is $\ge 0$ for the top layer, and $ = 0$ for all
others. In the following, term $Q^*$ is used, which is input distributed over 
the total solid volume of the uppermost sediment layer: $Q^* = Q / (\phi\, \Delta
z_1)$.
\end{enumerate}

During remineralisation/dissolution the particulate phase $S$ 
enters the dissolved phase, $X$, i.e. the respective tracer shifts from a
volume given by $\phi$  into the volume given by $1-\phi$.  (The porewater that
exchanges with the overlying water is distributed over the volume of the
overlying layer). Thus, the equation for the system consisting of $S$ and $X$
is: 
\begin{eqnarray}
\frac{\Delta S}{\Delta t} & = & -\kappa \, X \, S  + Q^*\label{dgls}\\
\frac{\Delta X}{\Delta t} & = & D^*X_{zz} - \kappa \, X \, S \, \frac{\phi}{1-\phi}\label{dglx}
\end{eqnarray}

With $S^* = S / (\phi\, \Delta z_1) $, the implicit discretized scheme for 
equations \ref{dgls} and \ref{dglx} is then:
\begin{eqnarray}
\frac{S^*-S^{t}}{\Delta t} & = & -\kappa \, X^{t} \, 
S^*  + Q^*\label{diss}\\
\frac{X^{t+\Delta t}-X^{t}}{\Delta t} & = & D^*_{zz} .
\end{eqnarray}
The final equation for $S$ is then:
\begin{eqnarray}
S^{t+\Delta t} & = & \frac{S^t+Q^*\, \Delta t}
{1+\kappa \, \Delta t \, X^{t+\Delta t}} \label{sfinal1}\\ 
& = & S^t - \kappa \, \Delta t \, X^{t+\Delta t} \, 
\frac{ S^{t} + Q^* \, \Delta t} {1+ \kappa \,  \Delta t \,
X^{t + \Delta t}} + Q^* \, \Delta t \label{sfinal2}
\end{eqnarray}

\paragraph{Silicate and opal} Opal dissolves to silicate; the rate of
dissolution depends on a constant $\kappa_{\opal}$ and the undersaturation of
the pore water, $U_\sio = Sat_\sio - \sio_s$. The system for silicate and opal
is solved on the basis of the undersaturation of silicate:
\begin{eqnarray}
\frac{\Delta \opal_s}{\Delta t} & = & -\kappa^{sedi}_{\opal,\sio} \, U_\sio \, \opal_s  + Q^*_\opal\label{dglopal_sed}\\
\frac{\Delta U_\sio_s}{\Delta t} & = & D^* - \kappa^{sedi}_{\opal,\sio} \, U_\sio\, \opal_s \, \frac{\phi}{1-\phi}\label{dglsio_sed}
\end{eqnarray}

The system is solved for the new undersaturation $U^{t+\Delta t}_\sio$  as
explained in the previous section, and the new silicate concentration in the
pore water and new opal concentration is computed from it.
\begin{eqnarray}
\sio^{t+\Delta t}_s & = & Sat_\sio - U^{t+\Delta t}_\sio \\
\opal^{t+\Delta t}_s & = & \opal^t_s + Q^*_\opal \, \Delta t \\
&& -\kappa^{sedi}_{\opal,\sio} \, \Delta t \, U^{t+\Delta t}_\sio \, 
\frac{ \opal^{t} + Q^*_\opal \, \Delta t} {1+ \kappa^{sedi}_{\opal,\sio} \,  \Delta t \,
U^{t + \Delta t}_\sio} \nonumber 
\end{eqnarray}

\paragraph{Detritus decomposition/Aerobic conditions} It is assumed that
detritus ($\det_s$) degrades with a constant rate $\kappa^{sedi}_{\det,\pho}$
[(\concO)$^{-1}$ d$^{-1}$] to phosphate [\concP]. This degradation depends on 
the oxygen available (\oxy, \concO). The stoichiometric ratio $R_{-O2:P}$ sets the
moles of oxygen which are consumed per mole of phosphorous that is decomposed. 
First, neglecting the diffusion of dissolved tracers other than oxygen, and 
the effect of remineralisation on them,  the system is:
\begin{eqnarray}
\frac{\Delta \det_s}{\Delta t} & = & -\kappa^{sedi}_{\det,\pho} \, \oxy_s \, \det_s  + Q^*_\det\label{dglpop}\\
\frac{\Delta \oxy_s}{\Delta t} & = & D^* - \kappa^{sedi}_{\det,\pho} \, \oxy_s \, \det_s  \, R_{-O2:P} \, \frac{\phi}{1-\phi}\label{dglox}
\end{eqnarray}

 (Note, that if option {\tt -DAGG} has been given during compilation, $Q^*_\det$
includes the input of sinking detritus and phytoplankton to the sediment, as
both components sink in that model setup.)
The system is solved for the new oxygen as described above, however, 
the stoichiometric constant, $R_{-O2:P}$ is included in the computation of new
oxygen, i.e. the system to be solved for the new oxygen concentration is:
\begin{eqnarray}
\frac{\oxy^{t+\Delta t}_s-\oxy^{t}_s}{\Delta t} & = & 
D^* + \kappa^{sedi}_{\det,\pho} \, \oxy^{t+\Delta t}_s \, 
\frac{ \det^{t}_s + Q^*_\det \, \Delta t} {1+ \kappa^{sedi}_{\det,\pho} \,  \Delta t \,
\oxy^t_s } \, R_{-O2:P} \, \frac{\phi}{1-\phi} 
\end{eqnarray}

After solving for $\oxy^{t+\Delta t}_s$, then  
\begin{eqnarray}
\det^{t+\Delta t}_s & = & \det^t_s + Q^*_\det \, \Delta t \\
&& -\kappa^{sedi}_{\det,\pho} \, \Delta t \, \oxy^{t+\Delta t}_s \, 
\frac{ \det^{t} + Q^*_\det \, \Delta t} {1+ \kappa^{sedi}_{\det,\pho} \,  \Delta t \,
\oxy^{t + \Delta t}_s} \nonumber 
\end{eqnarray}


Further, for the calculation of the associated phosphate gain, the phosphorous
shifts from a volume given by $\phi$  (from detritus in solid fraction)
to the volume given by $1-\phi$ (the pore water fraction). In addition, the change of nitrate, dissolved inorganic carbon and alkalinity
has to be accounted for using stoichiometric ratios.
\begin{eqnarray}
\frac{\Delta \pho_s}{\Delta t} & = &  \kappa^{sedi}_{\det,\pho} \, \oxy^{t+\Delta t}_s \, 
\frac{ \det^{t} + Q^*_\det \, \Delta t} {1+ \kappa^{sedi}_{\det,\pho} \,  \Delta t \,
\oxy^{t + \Delta t}_s}\frac{\phi}{1-\phi}\label{dglpho}\\
\frac{\Delta \nit_s}{\Delta t} & = & R_{N:P} \frac{\Delta \pho_s} {\Delta t} \label{dglnit}\\
\frac{\Delta \car_s}{\Delta t} & = & R_{C:P} \frac{\Delta \pho_s} {\Delta t} \label{dglcar}\\
\frac{\Delta \alk_s}{\Delta t} & = & - R_{N:P} \frac{\Delta \pho_s} {\Delta t} \label{dglalk}
\end{eqnarray}

The diffusion of these pore water tracers is computed in a seperate step, together
with other constituents, at the end of all remineralisation-diffusion
computations.

\paragraph{Anaerobic conditions/denitrification} As for the water column,
denitrification in the sediment occurs where oxygen falls  below a certain
level, and where detritus and nitrate are present:
\begin{eqnarray}
\frac{\Delta \det_s}{\Delta t} & = & \lambda_{\det,\pho}^{sedi} \,
\min\left(\det_s ,\frac{0.5\,\nit}{\frac{2}{3} R_{-O2:P} \, \Delta t}  \right)
\end{eqnarray}

Phosphate from detritus decomposition moves from a volume fraction 
defined by $\phi$ (solid sediment) to a volume fraction defined
by $1-\phi$ (pore water). The associated losses and gains for the other tracers are determined from
the phosphate gain multiplied by the stoichiometric ratio:
\begin{eqnarray}
\frac{\Delta \pho_s}{\Delta t} & = & \frac{\Delta \det_s}{\Delta t} \, \frac{\phi}{1-\phi}\\
\frac{\Delta \nit_s}{\Delta t} & = &  \left(R_{N:P} - \frac{2}{3} \,
R_{-O2:P}\right)  \frac{\Delta \pho_s}{\Delta t} \\
\frac{\Delta \ntwo_s}{\Delta t} & = & \frac{1}{3} \,R_{-O2:P} \, \frac{\Delta
\pho_s}{\Delta t}\\
\frac{\Delta \car_s}{\Delta t} & = & R_{C:P} \, \frac{\Delta \pho_s}{\Delta t}\\
\frac{\Delta \alk_s}{\Delta t} & = & - R_{N:P} \, \frac{\Delta \pho_s}{\Delta t}
\end{eqnarray}


\paragraph{Calcium carbonate dissolution and inorganic carbon cycle} The
dissolution of calcium carbonate is simulated as in the water colum: first,
from the dissociation constants for carbon dioxide the system  is solved for
the carbonate ion concentration, [CO$_3^{2-}$]. The apparent solubility product
of calcite, $S_{Ca}$, and mean total [Ca$^{2+}$] concentration are then used to
compute the undersaturation of the pore water carbonate,  $U_{CO3}=\max
\left(0, [Ca^{2+}]\,S_{Ca} - [CO_3^{2-}]\right)$. Using a constant
dissolution rate $\kappa^{sedi}_{\cal,\car}$, the system for $U_{CO3}$ and
calcium carbonate 
\begin{eqnarray}
\frac{\Delta \cal_s}{\Delta t}& = &  -\kappa^{sedi}_{\det,\pho} \, U_{CO3} \, \cal_s  + Q^*_\cal \label{dglcal}\\
\frac{\Delta U_{CO3}}{\Delta t} & = & D^* - \kappa^{sedi}_{\det,\pho} \, U_{CO3}\, \cal_s \, \frac{\phi}{1-\phi} \label{dglco3}
\end{eqnarray}

is solved for the new undersaturation, $U_{CO3}^{t+\Delta t}$ as explained
above; and \cal$^{t+\Delta t}_s$ is computed in
\begin{eqnarray}
\cal^{t+\Delta t}_s &  = &  \cal^t_s + Q^*_\cal \, \Delta t \\
&& - \kappa^{sedi}_{\det,\pho} \, \Delta t \, \, U_{CO3}^{t+\Delta t}
\frac{ \cal^{t}_s  + Q^*_\cal\, \Delta t}
{1+ \kappa^{sedi}_{\det,\pho} \, \Delta t \, U_{CO3}^{t+\Delta t}} \nonumber
\end{eqnarray}

Then the fluxes from carbonate dissolution are computed for inorganic carbon
$\car_s$ and alkalinity $\alk_s$:
\begin{eqnarray}
\frac{\Delta \car_s}{\Delta t} & = & \kappa^{sedi}_{\det,\pho} \, \, U_{CO3}^{t+\Delta t}
\frac{ \cal^{t}_s  + Q^*_\cal\, \Delta t}
{1+ \kappa^{sedi}_{\det,\pho} \, \Delta t \, U_{CO3}^{t+\Delta t}}  \, \frac{\phi}{1-\phi}\\
\frac{\Delta \alk_s}{\Delta t} & = & 2 \frac{\Delta \car_s}{\Delta t} 
\end{eqnarray}

\subsubsection{\label{sediment_shifting}Sediment upward and downward advection
({\tt sedshi.f90})}

Solid matter is deposited onto the sediment surface. It is displaced in the
vertical (and, in reality, horizontally) by the action of burrowing organisms.
\ham mimics the way in which animals move the sediment, by shifting sediment
downwards and upwards, depending on the ``filling'' state of the sediment.
Basically, two cases might occur (see also Heinze \etal, 1999a,
\nocite{heinze:1999a} and Heinze and Maier-Reimer, 1999b\nocite{heinze:1999b}, 
who also discuss two cases with ``humped'' profiles)

\begin{enumerate}
\item The actual volume of the solid constituents of a layer (i.e. the weight
of  its solid constituents divided by their density) exceeds the volume
prescribed (i.e. $\phi \times$ layer volume). This case may occur when
deposition of solid matter onto the sediment is higher than its dissolution. 
\item The actual volume of the solid constituents of a layer is too low to fill
the prescribed volume. This case may occur when a region suddenly is covered by
ice, sedimentation stops, but dissolution of solid constituents continues.
\end{enumerate} 

In the first case, the excess matter in the model sediment will be successively shifted downwards such
that all layers maintain their porosity and volume. 
The deepest layer empties its excess solid matter into a final
storage layer, whichs contents are traced for the whole time of an integration.
In the second case the vertically integrated lack of solid matter is
calculated, and then solid matter from this deepest storage layer is distributed
over the sediment successively, such that  all layers maintain their porosity
and volume.

The loading of the sediment ($v$) and its resulting shifting rate ($w$) is
determined from the weight concentration ($W_X\, X$) and densities ($\rho_X$)
of all solid sediment components $X$. 
\begin{eqnarray}
v & = & \frac{W_{POC}\,\det_s\,R_{C:P}}{\rho_{POC}}
+\frac{W_{\cal}\,\cal_s}{\rho_{\cal}}+\frac{W_{\opal}\,\opal_s}{\rho_{\opal}}
+\frac{\clay_s}{\rho_{\clay}}\\
w & = & \max(0.,(v-1.)/(v))
\end{eqnarray}


The up- and downward movement of the different components depends on their
concentration. If there is the need for upward shifting of sediment, i.e. of
supply from the permanently buried layer, but there are no biogenic components in this
layer, it is assumed that there is an infinite supply of clay from below. Thus
the state equation for changes due to sediment shifting  for layers 1 to ke,
with $W(0) = 0$  is
\begin{eqnarray}
\frac{\Delta \det_s}{\Delta t} & = &  \frac{\Delta (w \, \det)}{\Delta (\phi\,z)}\\
\frac{\Delta \opal_s}{\Delta t} & = & \frac{\Delta (w \, \opal)}{\Delta (\phi\,z)}\\
\frac{\Delta \cal_s}{\Delta t} & = & \frac{\Delta (w \, \cal)}{\Delta (\phi\,z)}\\
\frac{\Delta \clay_S}{\Delta t} & = & \frac{\Delta (w \, \clay)}{\Delta (\phi\,z)}
\end{eqnarray}

\clearpage
\newpage
\section{\label{sources}The biogeochemical modules of \ham}

Some of the following functionality depends on conditional compile parameters.
Please see section \ref{preprocessor_options} for a list of the available preprocessor options.

\subsection{\label{initialization}Initialization  of biogeochemistry {\tt
ini\_bgc.f90}}

All of the initialization is carried out within subroutine {\tt ini\_bgc.f90}
(see figure \ref{inipic}). After the runtime (control) parameters are read and
time series are initialized, the {\tt mo\_carbch.f90} module allocates memory
for the biogeochemical tracers fields:

\begin{description} 
\item[water column:] {\tt ocetra(ie,je,ke,nocetra)}
\item[atmosphere:] {\tt atm(ie,je,natm)}
\item[sediment pore water:] {\tt powtra(ie,je,ks,npowtra)}, and
\item[solid sediment:] {\tt sedlay(ie,je,ks,nsedtra)}
\end{description}

plus auxiliary arrays for chemical constants. Routine {\tt bodensed.f90} sets
up sediment layer thickness and porosity, plus sediment parameters.  The
biogeochemical parameters are set in {\tt beleg\_bgc.f90}. Here  a first (bulk)
initialization of the tracers is done using global means. 
%If the preprocessor
%option {\tt BGCLEVI} is given during compilation, and the compilation is done
%for a year-0 simulation, the values for nitrate, silicate, phosphate and oxygen
%will be overwritten by routine {\tt bgc\_lev.f90}, which reads the initial
%values from annual averages of the World Ocean Atlas, 1994, 
%\cite{conkright:1994}.  
The initialization is overwritten by routine {\tt
aufr\_bgc.f90} for restart experiments using a netCDF restart file of the 
global fields created by the previous simulation.  
{\tt beleg\_bgc.f90} also calls routine {\tt get\_dust.f90}, which
reads monthly mean dust deposition fields.  The chemical constants are initialized in
routine {\tt chemcon.f90}, and finally checks and budgets for the tracers are
calculated in the following routines {\tt }.


\begin{figure}[htb]
\caption{\label{inipic} Flow chart for initilization of \ham. }
\setlength{\unitlength}{1cm}
\begin{picture}(13,16.1)

\put(0,1.1){\framebox(2.2,13.9){\tt ini\_bgc.f90}}

\put(2.2,14.6){\vector(1,0){1.3}}
\put(3.5,14.2){\framebox(4.0,0.8){\tt read\_namelist.f90}}

\put(2.2,13.5){\vector(1,0){1.3}}
\put(3.5,13.1){\framebox(4.0,0.8){\tt ini\_timeser\_bgc.f90}}

\put(2.2,12.4){\vector(1,0){1.3}}
\put(3.5,12.0){\framebox(4.0,0.8){\tt ALLOC\_MEM\_BGCMEAN}}

\put(2.2,11.3){\vector(1,0){1.3}}
\put(3.5,10.9){\framebox(4.0,0.8){\tt ALLOC\_MEM\_BIOMOD}}

\put(2.2,10.2){\vector(1,0){1.3}}
\put(3.5,9.8){\framebox(4.0,0.8){\tt ALLOC\_MEM\_SEDMNT}}

\put(2.2,9.1){\vector(1,0){1.3}}
\put(3.5,8.7){\framebox(4.0,0.8){\tt ALLOC\_MEM\_CARBCH}}

\put(2.2,8){\vector(1,0){1.3}}
\put(3.5,7.6){\framebox(4.0,0.8){\tt bodensed.f90}}

\put(2.2,6.9){\vector(1,0){1.3}}
\put(3.5,6.5){\framebox(4.0,0.8){\tt beleg\_bgc.f90}}
%\put(7.5,5.9){\line(1,0){0.6}}
%\put(8.1,5.35){\line(0,1){1.0}}
%\put(8.1,6.35){\vector(1,0){0.8}}
%\put(8.1,5.35){\vector(1,0){0.8}}

%\put(8.2,6.6){\parbox{1cm}{(1)}}
%\put(9.0,6.0){\framebox(3.5,0.8){\tt bgc\_lev.f90}}
\put(7.5,6.9){\vector(1,0){0.6}}
\put(8.2,6.5){\framebox(4.0,0.8){\tt get\_dust.f90}}

\put(2.2,5.8){\vector(1,0){1.3}}
\put(3.5,5.4){\framebox(4.0,0.8){\tt chemcon.f90}}

\put(2.2,4.7){\vector(1,0){1.3}}
\put(2.3,4.9){\parbox{1cm}{(1)}}
\put(3.5,4.3){\framebox(4.0,0.8){\tt aufr\_bgc.f90}}

\put(2.2,3.6){\vector(1,0){1.3}}
\put(3.5,3.2){\framebox(6.5,0.8){\tt open\_bgcmean-2d-3d-bioz-sed.f90}}

\put(2.2,2.5){\vector(1,0){1.3}}
\put(3.5,2.1){\framebox(4.0,0.8){\tt inventory\_bgc.f90}}

\put(2.2,1.4){\vector(1,0){1.3}}
\put(3.5,1.0){\framebox(4.0,0.8){\tt chck\_bgc.f90}}
\put(8,1.1){\parbox{6cm}{(1): {\tt if kpaufr = 1 (re-start)}}}
%\put(8,1.6){\parbox{4cm}{(1): {\tt \-DBGCLEVI}}}

\end{picture}
\end{figure}

\subsection{\label{computation_of_bgc}Computation of biogeochemistry {\tt
bgc.f90}}

Subroutine {\tt bgc.f90} computes all changes of pelagic biogeochemical tracers
due to local processes (e.g., photosynthesis, heterotrophic processes,
N-Fixation and denitrification, carbonate chemistry, dust deposition and
release of dissolved iron), the air-sea gas exchange of carbon dioxide, oxygen,
dinitrogen and DMS, and the benthic processes. It further computes the vertical
displacement of particles, either using a constant sinking rate for some of the
particulate components, or by computation of aggregation and resulting sinking
rate (if the preprocessor key {\tt -DAGG} is set). Vertical flux
in the bottom layer forms the boundary condition for the sediment. The different
processes are calculated by calls from {\tt bgc.f90} to the subroutines. In
particular, {\tt bgc.f90} does the following:



\begin{figure}[hb]
\caption{\label{bgcpic} Flow chart for simulation of biogeochemistry in \ham.
The numbers at the arrows denote the  conditional execution of a subroutine, in
case a preprocessor option  was active during compilation.}
\setlength{\unitlength}{1cm}
\begin{picture}(13,14.9)

\put(0,0.0){\framebox(2,13.9){\tt bgc.f90}}

\put(2,13.6){\vector(1,0){1.3}}
\put(3.5,13.1){\framebox(6.5,0.8){\tt avrg\_bgcmean-2d-3d-bioz-sed.f90}}

\put(2,12.5){\vector(1,0){1.3}}
\put(3.5,12.1){\framebox(6.5,0.8){\tt save\_bgcmean-2d-3d-bioz-sed.f90}}

\put(2,11.4){\vector(1,0){1.3}}
\put(2.3,11.6){\parbox{1cm}{(1)}}
\put(3.5,11.0){\framebox(3.5,0.8){\tt inventory\_bgc.f90}}

\put(2,10.3){\vector(1,0){1.3}}
\put(3.5,9.9){\framebox(3.5,0.8){\tt chemcon.f90}}

\put(2,9.2){\vector(1,0){1.3}}
\put(3.5,8.8){\framebox(3.5,0.8){\tt ocprod.f90}}

\put(2,8.1){\vector(1,0){1.3}}
\put(3.5,7.7){\framebox(3.5,0.8){\tt cyano.f90}}

\put(2,7.0){\vector(1,0){1.3}}
\put(3.5,6.6){\framebox(3.5,0.8){\tt carchm.f90}}

\put(2,5.9){\vector(1,0){1.3}}
\put(2.3,6.1){\parbox{1cm}{(2)}}
\put(3.5,5.5){\framebox(3.5,0.8){\tt carchm\_ant.f90}}

\put(2,4.8){\vector(1,0){1.3}}
\put(2.3,5.0){\parbox{1cm}{(3)}}
\put(3.5,4.4){\framebox(3.5,0.8){\tt atmotr.f90}}

\put(2,3.7){\vector(1,0){1.3}}
\put(3.5,3.3){\framebox(3.5,0.8){\tt powach.f90}}

\put(7,3.7){\line(1,0){0.3}}
\put(7.3,3.15){\line(0,1){1.1}}

\put(7.3,4.25){\vector(1,0){0.7}}
\put(8.2,3.85){\framebox(3.5,0.8){\tt powadi.f90}}
\put(7.3,3.15){\vector(1,0){0.7}}
\put(8.2,2.75){\framebox(3.5,0.8){\tt dipowa.f90}}

\put(2,2.6){\vector(1,0){1.3}}
\put(3.5,2.2){\framebox(3.5,0.8){\tt sedshi.f90}}

\put(2,1.5){\vector(1,0){1.3}}
\put(3.5,1.1){\framebox(4.5,0.8){\tt avrg\_timeser\_bgc.f90}}

\put(2,0.4){\vector(1,0){1.3}}
\put(2.3,0.6){\parbox{1cm}{(1)}}
\put(3.5,0.0){\framebox(3.5,0.8){\tt inventory\_bgc.f90}}

\put(9,1.0){\parbox{8cm}{(1): {\tt -DPBGC\_CK\_TIMESTEP}}}
\put(9,0.5){\parbox{8cm}{(2): {\tt -DPANTHROPOCO2}}}
\put(9,0.0){\parbox{8cm}{(3): {\tt -DDIFFAT}}}

\end{picture}
\end{figure}

\begin{enumerate}
\item Increase biogeochemical time step counters of run (year)  and total
integration length.
\item Compute net solar radiation and its reduction by ice cover.
\item {\em (option }{\tt -DPBGC\_CK\_TIMESTEP}{\em)}: Call to {\tt
inventory\_bgc.f90}: compute total inventory of tracers (for debugging
purposes). 
\item Increase the counter for calls to biogeochemical routine by one (for
later averaging of monthly mean fields).
\item If time step is the first of a month: compute chemical constants by call
to subroutine {\tt chemcon.f90}.
\item Call to biogeochemical subroutine {\tt ocprod.f90}:  Most of the
biogeochemical interactions take place here, in particular:
        \begin{itemize}
        \item {\em Layers 1 to $k_{eu}$:} Photosynthesis, phytoplankton exudation and
        mortality, zooplankton grazing, excretion, egestion and mortality,
        remineralization of dissolved organic matter (DOM), opal and calcium
        carbonate production, opal dissolution, DMS production, dust deposition
        and iron release. 
        \item {\em Layers $(k_{eu}+1)$ to $kb$:} Mortality of phyto- and zooplankton, aerobic
        remoneralisation of detritus and DOM, dissolution of opal.
        \item {\em Layers $(k_{eu}+1)$ to $kb$:} Anaerobic remineralization of detritus
        (denitrification).
        \item {\em Layers 1 to $kb$:} Sinking of detritus, opal, calcium carbonate
        and dust, each with its own, constant sinking speed (default), or
        computation of particle size change due to aggregation of phytoplankton
        and detritus, computation of sinking speed and finally sinking of
        phytoplankton, detritus, opal, calcium carbonate, free and aggregated
        dust (option {\tt -DAGG}). In the latter case it is assumed that opal,
        calcium carbonate and aggregated dust have the same (time and space
        varying) sinking speed as phytoplankton and detritus, while free dust
        sinks with its own, constant sinking speed.
        \end{itemize}
All of these processes are calculated on the basis of P. The associated
changes in N and C are calculated using constant stoichiometric
ratios.
\item Call to {\tt cyano.f90}: Uptake of atmospheric nitrogen and immediate
relase as nitrate by diazotrophs in the surface layer, calculated from a
deviation of the N:P ratio of nutrients.
\item Call to {\tt carchm.f90}: Sea-air gas exchange for oxygen, DMS, N$_2$
and CO$_2$. Dissolution of calcium carbonate. Computes associated changes in
DIC and alkalinity.
\item {\em (option }{\tt -DPANTHROPOCO2}{\em)}: Call to {\tt carchm\_ant.f90}: Sea-air
gas exchange, for anthropogenic tracers. Dissolution of calcium carbonate with
respect to anthropogenic tracers. Computes associated changes in anthropogenic
DIC and alkalinity.
\item {\em (option }{\tt -DDIFFAT}{\em)}: Call to {\tt atmotr.f90}: Diffusive mixing of
gaseous tracers in the atmosphere.
\item Call to {\tt powach.f90}: Compute diffusion of all pore water tracers, 
dissolution of opal and calcium carbonate in the sediment, and remineralisation
(aerobic and anaerobic) of detritus in the sediment. Upper boundary conditions
for the sediment are sedimentation out of the water column, and concentration
of nutrients in the overlaying water (for pore water exchange with water). 
\item Call to {\tt sedshi.f90}: Shift solid components of sediment downwards
and upwards, to account for sediment volumetric gain or loss and maintain the
porosity profile. This also accounts  for a layer of permanent burial, which
collects the particulate matter (P, Si, C) lost over the full time of
integration.
\end{enumerate}

\subsection{\label{parameters_and_variables}Parameters and variables}

Global parameters and variables are communicated between the subroutines by
the modules {\tt mo\_biomod.f90}, {\tt mo\_carbch.f90}, {\tt
mo\_control\_bgc.f90}, {\tt mo\_sedmnt.f90} and \\
{\tt mo\_timeser\_bgc.f90}. 

The state variables of \ham and some of the (important)
fields are explained in tables \ref{tab_water_tracers}, \ref{tab_carb_params},
and \ref{tab_sedi_tracers}. All state variables for the water column and
sediment are declared in module {\tt mo\_carbch.f90}, and their
index in the biogeochemical tracer field is set in {\tt mo\_param1\_bgc.f90}.

Most of the parameters are defined in routine {\tt beleg\_bgc.f90}. The
parameters for water column biogeochemistry are used in routine {\tt
ocprod.f90} and {\tt cyano.f90}, and handed over in module {\tt
mo\_biomod.f90} (see table \ref{tab_bgc_params}). These also include the thickness
and depth of the bottom layer in the water column, and the field for incident
solar radiation. The parameters for sediment biogeochmistry are used in
routines {\tt powach.f90}, {\tt powadi.f90} and {\tt dipowa.f90}, and handed
over with the module {\tt mo\_sedmnt.f90} - however these routines also need
the other modules, e.g. for stoichiometric ratios, variables, etc. Parameters
for carbonate chemistry are mostly defined in routine {\tt chemcon.f90}, which
computes the chemical constants once at the beginning of a simulation from
temperature and salinity, and interpolates between monthly mean values. The
surface constants may be overwritten by the ones stored in the restart file
from the simulation of a previous year, and used later for interpolation
between monthly vaues. They are handed over by module {\tt mo\_carbch.f90}. 


\subsection{\label{modules}Modules and subroutines} 

\subsubsection{atmotr.f90} 
%Calculates atmospheric diffusion for surface
%CO$_2$, O$_2$  and N$_2$.
Distributes anthropogenic CO$_2$-emissions over source regions
and calculates the atmospheric diffusion for atmospheric CO$_2$,
O$_2$, and N$_2$.
Called by {\tt bgc.f90}.

\subsubsection{aufr\_bgc.f90} Reads restart fields of biogeochemical tracers to
continue an integration. Uses either a netCDF file {\tt restart\_bgc.nc} ({\tt
-DPNETCDF}) or a binary file {\tt restart\_bgc} (default). Masks tracers with
land-sea mask, and restricts them to positive values, if necessary. Reads
chemical constants. Reads, masks and restricts to positive values H$^+$ and CO$^{3-}$
concentration, and permanently buried tracers in the sediment.  Called by {\tt
ini\_bgc.f90} .

\subsubsection{aufw\_bgc.f90} Writes restart data of biocheochmical tracers at
the end of an integration. Writes either to netCDF file {\tt restartw\_bgc.nc} ({\tt
-DPNETCDF}) or to binary file {\tt restart\_bgc} (default). Masks tracers 
with land-sea mask before writing. Writes, masks and restricts H$^+$ and
CO$^{3-}$ concentration. In case of netCDF output also writes the grid
definition (latitude, longitude, water depth, box depth) and chemical
constants. Called by {\tt mpiom.f90} ({\em MPI-OM}).

\subsubsection{avrg\_bgcmean\_2d-\_3d.f90} Average monthly mean fields of
biogeochemical tracers and fluxes. Called by {\tt bgc.f90} and {\tt end\_bgc.f90}.

\subsubsection{avrg\_timeser\_bgc.f90} Averages time series output every   {\tt
ldtrunbgc} time step. Increases sample counter of time series. Called by {\tt
bgc.f90}. 

\subsubsection{beleg\_bgc.f90} Initializes biogeochemical variables (with
globally homogeneous values for all variables) and water column parameters,
time step counters, variables for check-sums and budgets, fields for monthly
mean output. Called by {\tt ini\_bgc.f90}.

\subsubsection{bgc.f90} Main biogeochemical subroutine, called at each time
step by {\tt mpiom.f90} {\em (MPI-OM)}. See above for description.

%\subsubsection{bgc\_lev.f90} Initializes biogeochemical water column variables
%with World Ocean Atlas \cite{conkright:1994} phosphate, nitrate, silicate and
%oxygen. Uses files {\tt INIPHO}, {\tt ININIT}, {\tt INISIL} and {\tt INIOX}.
%Called by  {\tt beleg\_bgc.f90} after globally homogeneous initilization, if
%preprocessor option {\tt -DBGCLEVI} is given.

\subsubsection{bodensed.f90} Initializes sediment (layers, thickness, pore
water fraction). Defines thickness ({\tt bolay(ie,je)}) and index ({\tt
kbo(ie,je)}) field for sediment-water interaction. Sets sediment parameters.
Called by {\tt ini\_bgc.f90}.

\subsubsection{carchm.f90} Computes carbonate chemistry in the water column,
and sea-air gas exchange, for oxygen, DMS, O$_2$, N$_2$ and CO$_2$. Computes
dissolution of calcium carbonate. Called by {\tt bgc.f90}.

\subsubsection{carchm\_ant.f90}  Sea-air
gas exchange, for anthropogenic tracers. Dissolution of calcium carbonate with
respect to anthropogenic tracers. Computes associated changes in anthropogenic
DIC and alkalinity. Called by {\tt bgc.f90}.

\subsubsection{close\_bgcmean\_2d-\_3d-\_bioz-\_sed.f90} Close mean fields of
biogeochemical tracers and fluxes. Called by {\tt end\_bgc.f90}.

\subsubsection{chck\_bgc.f90} Checks biogeochemical fields for invalid
(negative) values in wet and dry fields (using ocean subroutine {\tt
extr.f90}). Checks for cyclicity of tracers. 

\subsubsection{chemcon.f90} Computes chemical constants in the surface layer ({\tt
chemcm(ie,je,1-8,12}) and the water column ({\tt ak13(ie,je,ke)}, {\tt
ak23(ie,je,ke)}, {\tt akb3(ie,je,ke)} and {\tt aksp(ie,je,ke)}) from
temperature and salinity. This routine is called once in the beginning, and
during the model run at the first time step of each month. Called by {\tt ini\_bgc.f90}
and {\tt bgc.f90}. 
Chemical constants for the surface layer will be computed only if the
first argument in the call to {\tt CHEMCON} is ($\le 0$). The default for the first
argument for a call from {\tt bgc.f90} is (-26). If the 
first argument is set to (-13, default for call from initialization {\tt
ini\_bgc.f90}), surface values for all months will be set to the current
(usually month 1) values. Values for the surface will be overwritten by the ones
from the restart file {\tt restartr\_bgc.nc}. The constants for the water column
will be computed in any case. 

\subsubsection{cyano.f90} Computes uptake of atmospheric nitrogen and its
immediate release as nitrate by diazotrophs in the surface layer, calculated
from a deviation of the N:P ratio of nutrients. Called by {\tt bgc.f90}.

\subsubsection{dipowa.f90} Computes upward diffusion of pore water tracers.
Called by {\tt powach.f90}.

\subsubsection{end\_bgc.f90} Finishes with marine biogeochemistry modules.
Calculates global inventories for water column and sediment tracers and writes
time series. Called by at the end of the simulation by {\tt mpiom.f90} ({\em
MPI-OM}). 

\subsubsection{get\_dust.f90} Gets monthly mean fields of dust deposition. Uses
file {\tt dust\_grob}. Called by {\tt beleg\_bgc.f90}

\subsubsection{ini\_bgc.f90} Initializes the biogeochemical module. See
detailed description above. Called by {\tt mpiom.f90} ({\em MPI-OM}).

\subsubsection{ini\_timeser\_bgc.f90} Assigns grid cell indices for prescribed (by
runtime namelist) latitude and longitude of time series. Additionally assigns 
grid cell indices for three prescribed depths of sedimentation at time series. 
Initializes field {\tt ts1(nvarts1,nelets1,lents1)}, where {\tt nvarts1} is the
number of tracers (e.g., phytoplankton, phosphate, ...), {\tt nelets1} is the
number of stations (note that the first one - i.e., index 1 - is currently used
for global inventories and fluxes), and {\tt lents1} is the number of time
steps. Dimension {\tt nvarts1} is defined in {\tt mo\_timeser\_bgc.f90}.
Dimension {\tt nelets1} is defined from {\tt nts}  as in {\tt
mo\_timeser\_bgc.f90} in this routine. Dimension {\tt lents1} is defined during
runtime from the sampling frequency given in the namelist ({\tt nfreqts1}) and
the total number of model time steps. Called by {\tt ini\_bgc.f90}.

\subsubsection{inventory\_bgc.f90} Calculates global inventories of tracers in
the water column and the sediment, and of additional tracers (H$^+$,
CO$^{2-}_3$).  Calculates global inventories of mass. Calculates global fluxes of
air-sea gas exchange (CO$_2$, O$_2$, N$_2$ and N$_2$O), and fluxes to to
sediment (P, CaCO$_3$-C, Si). Calculates global fluxes of antropogenic tracers
for option {\tt -DPANTHROPOCO2}. Calculates global total C, N, P, and Si. Called
by {\tt ini\_bgc.f90} and {\tt  bgc.f90}.

\subsubsection{ocprod.f90} Most of the biogeochemical interactions take place
here, see subsection \ref{computation_of_bgc} for details. Called by {\tt
bgc.f90}.

\subsubsection{mo\_param1\_bgc.f90} Header file that defines and declares the
indices for the arrays of water column tracers ({\tt nocetra}, for array {\tt
ocetra(ie,je,ke,nocetra)}), pore water tracer ({\tt npowtra}, for array {\tt
powtra(ie,je,ks,npowtra)}), solid sediment ({\tt nsedtra}, for array {\tt
sedlay(ie,je,ks,nsedtra)}) and atmospheric tracers ({\tt natm}, for array {\tt
atm(ie,je,natm)}). Also sets the dimension of the monthly mean fields {\tt
bgcm2d} and {\tt bgcm3d}). 

\subsubsection{open\_bgcmean\_2d-\_3d-\_bioz-\_sed.f90} Open the monthly mean fields of
biogeochemical tracers and fluxes. Called by {\tt ini\_bgc.f90}.

\subsubsection{powach.f90} Computes diffusion of porewater tracers, dissolution
of opal and calcium carbonate and remineralisation (aerobic and anaerobic) of
detritus in the sediment. Called by {\tt bgc.f90}.


\subsubsection{powadi.f90} Solves tridiagonal matrix for computation of
simultaneous dissolution and diffusion in the sediment. Called by {\tt
powach.f90}.

\subsubsection{read\_namelist.f90} Reads the parameters handed at run time to
the model from file {\tt NAMELIST\_BGC}. See section \ref{io} for  names and
meaning of parameters to be read. Called by  {\tt ini\_bgc.f90}.

\subsubsection{save\_timeser\_bgc.f90} Writes time series output to ASCII
file {\tt timeser\_bgc}. Writes inventories and fluxes, as well as globally
integrated sedimentation to file {\tt bgcout} for budgets. Called by {\tt
end\_bgc.f90}.

\subsubsection{sedshi.f90} Shifts solid components of sediment downwards and
upwards, to account for volumetric sediment gain or loss. This includes a layer
for permanent burial, which collects the particulate matter (P, Si, C, clay)
over the full time of integration. Called by {\tt bgc.f90}. 

%\subsubsection{TREDSY.h} Defines auxiliary field {\tt tredsy(ie,0:ke,3)} needed
%by {\tt powadi.f90} and {\tt dipowa.f90}.

\subsubsection{write\_bgcmean\_2d-\_3d-\_bioz-\_sed.f90} Write mean fields of
biogeochemical tracers and fluxes. Called by {\tt bgc.f90} and {\tt end\_bgc.f90}.

\subsubsection{mo\_biomod.f90} Declares biogeochemical parameters and auxiliary
fields {\tt bolay(ie,je)}, {\tt kbo(ie,je)} and {\tt strahl}.

\subsubsection{mo\_bgcmean.f90} Declares fields for monthly means of
biogeochemical tracers and flows.

\subsubsection{mo\_carbch.f90} Declares biogeochemical variables for tracers
in the water column ({\tt ocetra(ie,je,ke,nocetra)}) and atmosphere ({\tt
atm(ie,je,natm)}), fields for chemical constants, gas exchange coefficients,
dust deposition.

\subsubsection{mo\_control\_bgc.f90} Sets logical numbers of IO units, time
step constants and counters and mask values.

\subsubsection{mo\_sedmnt.f90} Declares biogeochemical variables for sediment
{\tt powtra(ie,je,ks,npowtra)} and \\
{\tt sedlay(ie,je,ks,nsedtra)}, fields for
chemical constants, and flux fields {\tt silpro(ie,je)}, {\tt prorca(ie,je)},
{\tt prcaca(ie,je)} and {\tt produs(ie,je)}. Declares sediment parameters.

\subsubsection{mo\_timeser\_bgc.f90} Declares logical IO unit for time series
fluxes, and time series auxiliary arrays.

\section{\label{coupling}Coupling \ham and {\em MPI-OM}}
\subsection{\label{model_setup} model setup}


\ham is compiled as a part of the ocean model {\em MPI-OM}. A wide variety of platforms are supported, including
NEC SX-6 (DKRZ hurrikan), LINUX, SUN and Windows (not documented). The models also support the parallelization options
OpenMP and MPI.
For a detailed documentation of the ocean model options, parallelization options and
the various runtime environments, please refer to the {\em MPI-OM} documentation which is currently under revision.  
The source code is available from CD or, if you are a registered user, from the ZMAW CVS server.

To retrieve the sources for release 1.1 from CVS you have to do the following:
\begin{footnotesize}
\begin{verbatim}
setenv CVSROOT :pserver:<user-name>@cvs.zmaw.de:/server/cvs/mpiom1
cvs login
cvs checkout -r release_1_1 mpi-om 
\end{verbatim}
\end{footnotesize}

In any way you will end up with a directory named mpi-om and five subdirectories:

\begin{enumerate}

\item \textbf{src}  \newline 
All sources required to compile the MPI-OM ocean standalone model. 
 
\item \textbf{src\_hamocc} \newline 
All sources required to compile the HAMOCC marine biogeochemistry together with the MPI-OM ocean model. 

\item \textbf{make} \newline 
Makefiles required to compile the MPI-OM ocean and the MPI-OM/HAMOCC model. 

\item \textbf{bin} \newline 
This is where the binaries are stored after compilation. 

\item \textbf{run} \newline 
Shell scripts to set up a runtime environment. 

\end{enumerate}

If you are working within the Max Planck Institute for Meteorology (MPI-MET) or the DKRZ,
just go to the make directory and type "make -f Makefile\_mpiom\_hamocc\_omip". the Makefile will 
automatically detect your machine-type and select an available compiler.
If you need to use a different compiler or if you are not working within the MPI-MET or the DKRZ,
you will have to provide and select the compiler and, if required, the MPI and the NetCDF libaries yourself.
The example below shows parts of the standard Makefile. 

\begin{footnotesize}
\begin{verbatim}
#-----------------------------------------------------------------------------
DEF =   -DZZNOMPI -DVERSIONGR30 -DZZLEVELS40 -DZZTIMECHECK \
        -DZZYEAR360 -DSOR -DZZRIVER_GIRIV \
        -DMEAN -DRESYEAR -DZZDEBUG_ONEDAY \
        -DQLOBERL -DBULK_KARA \
        -DEISREST -DREDWMICE -DALBOMIP \
        -DISOPYK -DGMBOLUS \
        -DADPO -DSLOPECON_ADPO  \
        -DNURDIF \
        -DDIAG -DZZGRIDINFO -DZZDIFFDIAG -DZZKONVDIAG \
        -DZZCONVDIAG -DZZAMLDDIAG -DTESTOUT_HFL -DZZRYEAR \ # end of the MPI-OM flags
        -DPBGC -DPNETCDF -DDIFFAT -DFB_BGC_OCE              # HAMOCC5 flags
#-----------------------------------------------------------------------------

PROG =	mpiom_hamocc.x

VPATH = ../src_hamocc  : ../src

SRCS =	absturz.f90 adisit.f90 adisit1.f90 adisitj.f90 amocpr.f90 aufr.f90 \
        ....

OBJS =	absturz.o adisit.o adisit1.o adisitj.o amocpr.o aufr.o \
        ....

# Set up system type
UNAMES := $(shell uname -s)
HOST   := $(shell hostname)

ifeq ($(UNAMES),SunOS)
NETCDFROOT = /pf/m/m214089/yin/local/SunOS64
NETCDF_LIB = -L${NETCDFROOT}/lib -lnetcdf
NETCDF_INCLUDE = -I${NETCDFROOT}/include

MPIROOT = /opt/SUNWhpc
MPI_LIB = -L${MPIROOT}/lib/sparcv9 -R${MPIROOT}/lib/sparcv9 -lmpi
MPI_INCLUDE = -I${MPIROOT}/include
endif

INCLUDES = $(NETCDF_INCLUDE) $(MPI_INCLUDE)
LIBS = $(NETCDF_LIB) $(MPI_LIB)

ifeq ($(UNAMES), SunOS)
#-----------------------------------------------------------------------------
#FOR SUN (SunStudio10 compiler)
F90 = f95
F90FLAGS = $(INCLUDES) -xtypemap=real:64,double:64,integer:32 -fast \
                       -g -xarch=v9b -xchip=ultra3cu  -fpp 
# OpenMP: -xopenmp
endif

LDFLAGS = $(F90FLAGS)

all: $(PROG)

$(PROG): $(OBJS)
	$(F90) $(LDFLAGS) -o $@ $(OBJS) $(LIBS)
	cp $(PROG) ../bin/.

clean:
	rm -f $(PROG) $(OBJS) *.mod i.*.L

.SUFFIXES: $(SUFFIXES) .f90

%.o: %.f90
	$(F90) $(F90FLAGS) -c $(DEF) $<	

#
#-----------------------------------------------------------------------------
# Dependencies
#
absturz.o: mo_commo1.o mo_commo2.o mo_param1.o mo_units.o mo_parallel.o
...
\end{verbatim}
\end{footnotesize}

In the run directory you will also find a shell-script called "prepare\_run\_mpiom\_hamocc\_omip" which helps to set up
a runtime environment on hurrikan. It will create a directory in the appropriate place and 
set links to the necessary input and forcing files. It will also create run-script 
which can serve as a simple example for your actual script. 
For more details please refer to the {\em MPI-OM} documentation.

\subsection{\label{preprocessor_options}\ham preprocessor options}

There are eight different preprocessor options, which can be used to include
different configurations of \ham (see Appendix B for a listing of subroutines
that are affected by these preprocessor options):

\paragraph{\tt -DAGG} Include aggregation of marine snow and variable sinking
speed. 

%ich glaube patrick hatte das rausgeworfen, 14.03.05
%\paragraph{\tt -DBGCLEVI} Start from annual fields of World Ocean Atlas 1994
%\cite{conkright:1994} phosphate, nitrate, silicate and oxygen. 
% Ich hab das nicht rausgeworfen, ich habs nur nie implementiert (keine Zeit).
% Finde die Idee aber gut. Patrick, 01.09.05

\paragraph{\tt -DDIFFAT} Use ``diffusive atmosphere'' for air-sea gas exchange.

\paragraph{\tt -DFB\_BGC\_OCE} Use Chl {\it a} attenuation to modify water leaving
radiance and heat budget for the ocean.

\paragraph{\tt -DPNETCDF} Output to netCDF instead of binary files. 

\paragraph{\tt -DPANTHROPOCO2} Include anthropogenic carbon tracers. 

\paragraph{\tt -DPANTHROPOCO2\_START} Use values for natural carbon tracers in
restart file to initialize anthropogenic carbon tracers. 

\paragraph{\tt -DPBGC\_CK\_TIMESTEP} Check inventory of tracers (by call to
{\tt inventory\_bgc.f90} at every time step before and after biogeochemical
routines). 

\paragraph{\tt -PDYNAMIC\_BGC} Special diagnosis, not further documented! 
Compute changes of tracers in the upper 4 layers
due to mixing, advection and biology. Output follows the bgcmean standards and is saved in {\tt dynamic\_bgc.nc}.


\section{\label{implementation}Implementation of \ham into {\em MPI-OM}}

The biogeochemical model is activated as a subroutine of the ocean circulation model 
if the preprocessor key {\tt \-DPBGC} is set. The main oceanic routine ({\tt
mpiom.f90}) first calls the routine {\tt ini\_bgc.f90} for initialization of 
the biogechemical tracer fields (see figure \ref{mainpic}). 
During the model integration computation of biogeochemistry 
is done as part of the ocean model's time loop by calling {\tt bgc.f90} at each time step. 

After the biogeochmistry time step, subroutine {\tt dilute\_bgc.f90} computes the change 
of tracer concentrations due to 
changes in the thickness of the surface layer (e.g.,  as a result of ice melting). 
Advection and diffusion of the biogeochemical tracer fields are then performed
calling {\tt OCADPO} of the {\em MPI-OM} code for each tracer. 

Afterwards fields are time averaged over specified
periods (see section \ref{sec:bgcmean}) by calling {\tt avrg\_bgcmean.f90}. 
At the end of each integration (normally a year, depending on model resolution), final tracer concentrations 
are written to a restart file called {\tt restartw\_bgc.nc} in the routine {\tt aufw\_bgc.f90}. 
The time averaged
fields of tracer concentrations are written to files {\tt bgcmean\_2d[\_3d][\_bioz].nc} in the
routines {\tt write\_bgcmean\_2d[\_3d][\_bioz].f90}. Inventories and time series are written
in routine {\tt end\_bgc.f90} to files {\tt bgcout} and {\tt time series\_bgc}. 

Biogeochemical tracer fields, their dimensions, auxiliary
arrays and other parameters are communicated between the different subroutines
and between the biogeochemical part and the circulation model in modules
{\tt mo\_biomod.f90}, {\tt mo\_carbch.f90} and {\tt mo\_control\_bgc.f90},
and further through the header file {\tt mo\_param1\_bgc.f90}.


\begin{figure}[htb]
\caption{\label{mainpic} Calling sequence of the biogeochemical subroutines from
the physical ocean model. $^*$: see detailed flow chart for these routines 
(Fig. \ref{inipic}, \ref{bgcpic}). $^1$:
Subroutines are part of {\em MPI-OM}. Subroutines that are commented out in the
current model version  and debugging routines are not included.}
\setlength{\unitlength}{1cm}
\begin{picture}(13,15.6)
\put(0,1.5){\framebox(2,12.5){\tt mpiom.f90}}

\put(2,13.6){\vector(1,0){1.3}}
\put(3.5,13.2){\framebox(3.5,0.8){\tt ini\_bgc.f90$^*$}}

\put(2,12.9){\line(1,0){10.2}}
\put(10,13.0){\parbox{2cm}{\it time loop}}

\put(12.2,7.0){\line(0,1){5.9}}

\put(2,11.9){\vector(1,0){1.3}}
\put(3.5,11.5){\framebox(3.5,0.8){\tt bgc.f90$^*$}}

\put(2,10.9){\vector(1,0){1.3}}
\put(3.5,10.5){\framebox(3.5,0.8){\tt dilute\_bgc.f90$^*$}}


%\put(2,10.8){\vector(1,0){1.3}}
%\put(3.5,10.4){\framebox(3.5,0.8){\tt ADVECTION\_BGC.h}}
%\put(7,10.8){\line(1,0){0.3}}
%\put(7.3,10.25){\line(0,1){1.1}}
%\put(7.3,10.25){\vector(1,0){0.7}}
%\put(8.2,9.85){\framebox(3.5,0.8){\tt ocadpo.f90$^1$}}
%\put(7.3,11.35){\vector(1,0){0.7}}
%\put(8.2,10.95){\framebox(3.5,0.8){\tt dilute\_bgc.f90}}

\put(2,9.7){\vector(1,0){1.3}}
\put(3.5,9.3){\framebox(3.5,0.8){\tt ocadpo.f90$^1$}}

\put(2,8.7){\vector(1,0){1.3}}
\put(3.5,8.3){\framebox(3.5,0.8){\tt ocjitr.f90.f90}}

%\put(7,8.6){\vector(1,0){1}}
%\put(8.2,8.2){\framebox(3.5,0.8){\tt octdiff\_bgc.f90}}

\put(2,7.7){\vector(1,0){1.3}}
\put(3.5,7.3){\framebox(3.5,0.8){\tt octdiff\_trf.f90}}


%\put(2,7.5){\vector(1,0){1.3}}
%\put(3.5,7.1){\framebox(3.5,0.8){}}

%\put(2,6.4){\vector(1,0){1.3}}
%\put(3.5,6){\framebox(6.5,0.8){\tt avrg\_bgcmean-2d-3d-bioz-sed.f90}}

\put(2,7.0){\line(1,0){10.2}}

\put(2,6.2){\vector(1,0){1.3}}
\put(3.5,5.8){\framebox(3.5,0.8){\tt aufw\_bgc.f90}}


\put(2,4.15){\vector(1,0){1.3}}
\put(3.5,3.75){\framebox(3.5,0.8){\tt end\_bgc.f90}}

\put(7,4.15){\line(1,0){0.3}}
\put(7.3,1.95){\line(0,1){4.4}}
\put(7.3,6.35){\vector(1,0){0.7}}
\put(8.2,5.85){\framebox(6.5,0.8){\tt avrg\_bgcmean-2d-3d-bioz-sed.f90}}
\put(7.3,5.25){\vector(1,0){0.7}}
\put(8.2,4.85){\framebox(6.5,0.8){\tt write\_bgcmean-2d-3d-bioz-sed.f90}}
\put(7.3,4.15){\vector(1,0){0.7}}
\put(8.2,3.75){\framebox(6.5,0.8){\tt close\_bgcmean-2d-3d-bioz-sed.f90}}
\put(7.3,3.05){\vector(1,0){0.7}}
\put(8.2,2.65){\framebox(4.5,0.8){\tt inventory\_bgc.f90}}
\put(7.3,1.95){\vector(1,0){0.7}}
\put(8.2,1.55){\framebox(4.5,0.8){\tt save\_timeser\_bgc.f90}}

\end{picture}
\end{figure}

\subsection{\label{temporal_resolution}Temporal resolution}

The temporal resolution of the biogeochemical model is determined by the ocean
model: {\tt mpiom.f90} reads the length of the time step ({\tt dt}, in seconds) from
the namelist and passes it to {\tt ini\_bgc.f90} as argument in the call. 
{\tt ini\_bgc.f90} then sets {\tt dtbgc} (time step length in seconds) and {\tt
dtb} (time step length in days) from it. Most of the rate parameters used in
the biogeochemical model are already multiplied with the time step length during
initialization of the model. The time step parameters are communicated between
the routines in module {\tt mo\_control\_bgc.f90}. 

\subsection{\label{spatial_resolution}Spatial resolution}

The ocean model's grid dimensions are given by {\tt ie}, {\tt je}, and
{\tt ke}, for the number of grid cells in $x$, $y$, and $z$ direction, respectively. 
The dimensions are passed to the biogeochemical modules by function call arguments.

\subsection{\label{transport}Transport and mixing of biogeochemical
tracers}

The physical processes that change biogeochemical tracers are 
advection, diffusion, dilution resulting from changes in the surface layer thickness,
and mixing by subgrid scale eddies. The ocean model computes advection and diffusion of the biogeochemical
tracers as it does for salinity or temperature, except for the omission of isopycnal
mixing for biogeochemical tracers.

The main ocean model routine first defines a field {\tt bgcddpo(ie,je,ke)}  that contains
the actual thickness of the surface layer (i.e., corrected for changes due to ice
melting, evaporation, freshwater fluxes, etc.), and the standard depths for all
layers. A subset of this field, {\tt layer1\_bgc(ie,je)} contains only
the thickness of the surface layer at the beginning of each time step. 

After the computation of changes in surface layer thickness,
the updated depths are stored in field {\tt layer1\_new(ie,je)}. 
Subroutine {\tt dilute\_bgc.f90} then computes the dilution of all 
biogeochemical tracers from the ratio of these depths. 
Afterwards the advection of biogeochemical tracer fields 
is computed in the same fashion as for salinity and temperature
by a call to ocean subroutine {\tt ocadpo.f90}. In case of 
numerical undershoots (any of the biogeochemical tracer concentrations becomming smaller 
than zero) the tracer concentration at that point is reset to zero. 

Subgrid eddy effects are computed for the biogeochemical tracers in the same
way as for salinity and temperature (this is done in subroutine {\tt
ocjitr.f90}, part of the ocean model).  Finally, mixing of the tracers is computed 
by calling {\tt octdiff\_trf.f90}. It computes mixing identical to that for salinity and
temperature. Isopycnal mixing is expensive in terms of computing time.  
If performance is a concern, the subroutine {\tt octdiff\_bgc.f90}, which does not 
include isopycnal mixing, can de used instead of {\tt octdiff\_trf.f90}. \\
%Note that in contrast to mixing of salinity and temperature, for
%the biogeochemical tracers isopycnal mixing (which may be expensive in terms
%of computing time) is not included. \\
\\


%js 280605 no longer valid, commented out
%\subsection{Modules, header files and subroutines} 

%\subsubsection{ ADVECTION\_BGC.h}   Sets tracer values at dry cells to bottom
%cell value as done for temperature and salinity in ocean subroutine {\tt
%octher.f90}. Calls {\tt dilute\_bgc.f90} for dilution and {\tt ocadpo.f90} for
%advection of tracers. Resets tracers, if necessary. Inserted into {\tt
%mpiom.f90} during preprocessing.

%\subsubsection{ dilute\_bgc.f90}  
%Computes dilution (or concentration) of biogeochemical tracers due to changes in
%upper layer thickness caused by ice melting/freezing, evaporation, etc..
%Called by {\tt ADVECTION\_BGC.h}.

%\subsubsection{ MIXING\_BGC.h}
%Calls {\tt octdiff\_bgc.f90} for diffusion of tracers. Inserted into
% {\tt mpiom.f90}  during preprocessing.

%\subsubsection{ octdiff\_bgc.f90}
%Computes diffusion of biogeochemical tracers. Resets tracers if necessary.


\clearpage
\newpage
\section{\label{io}Input and output files}

Model input fields (see table \ref{table_io_files}) for initialization are read
by subroutine
%subroutines {\tt bgc\_lev.f90} (for  initialization from climatological
%fields of nutrients and oxygen from files {\tt INIPHO}, {\tt ININIT}, {\tt
%INISIL} and {\tt INIOX}), or 
{\tt aufr\_bgc.f90} (for initialization from
previous run from file {\tt restartr\_bgc}). 
%Both subroutines are called at the beginning of
The subroutine is called at the beginning of
each simulation by routine {\tt ini\_bgc.f90}.  Monthly mean dust deposition is
read from file {\tt dust\_grob} by routine {\tt get\_dust.f90}. The run script
provides the linking of files to the given names, and further creates a list of
runtime control parameters, {\tt NAMELIST\_BGC}, which is then read by {\tt
read\_namelist.f90}.

%table9
\begin{table}[hbt]
\caption{\label{table_io_files} Input and output files for \ham, the name of
the module that accesses the file, format and contents. File {\tt NAMELIST\_BGC}
is created by the job script, other files are supplied by the user. See section
\ref{sec:bgcmean} for specification of the time averaging of the bgcmean files.} 
\vspace{.2cm}
\begin{center}
\begin{tabular}{lllp{4.5cm}l} \hline
file name & accessed by   & format       & contains    \\ \hline
\multicolumn{4}{l}{\rule{0mm}{4mm}{\it input files}}\\ 
{\tt NAMELIST\_BGC} & {\tt read\_namelist.f90} & ASCII & runtime parameters\\ 
 {\tt INPDUST.nc} & {\tt get\_dust.f90} & NetCDF & dust input \\
% {\tt INIPHO}  & {\tt bgc\_lev.f90} & binary & initial phosphate\\
% {\tt ININIT}  & {\tt bgc\_lev.f90} & binary & initial nitrate\\
% {\tt INISIL}  & {\tt bgc\_lev.f90} & binary & initial silicate\\
% {\tt INIOX}   & {\tt bgc\_lev.f90} & binary & initial oxygen\\
 {\tt restartr\_bgc}& {\tt aufr\_bgc.f90} & binary & initial tracers (default)\\
\\ 
 {\tt restartr\_bgc.nc}& {\tt aufr\_bgc.f90} & netCDF & initial tracers ({\tt
 -DPNETCDF})\\
\\ \hline
\multicolumn{4}{l}{\rule{0mm}{4mm}{\it output files}}\\
{\tt restartw\_bgc}& {\tt aufw\_bgc.f90} & binary & final tracers (default)\\
{\tt restartw\_bgc.nc}& {\tt aufw\_bgc.f90} & netCDF & final tracers ({\tt -DPNETCDF})\\

{\tt bgcmean\_2d.nc }& {\tt open\_bgcmean\_2d.f90}  & netCDF & global time averaged    \\
                     & {\tt write\_bgcmean\_2d.f90} &        & 2-D tracer fields and   \\
                     & {\tt close\_bgcmean\_2d.f90} &        & fluxes ({\tt -DPNETCDF})\\


{\tt bgcmean\_3d.nc }& {\tt open\_bgcmean\_3d.f90}  & netCDF & global time averaged    \\
                     & {\tt write\_bgcmean\_3d.f90} &        & 3-D tracer fields       \\
                     & {\tt close\_bgcmean\_3d.f90} &        & ({\tt -DPNETCDF})       \\

{\tt bgcmean\_bioz.nc }& {\tt open\_bgcmean\_bioz.f90}    & netCDF & global time averaged \\
                       & {\tt write\_bgcmean\_bioz.f90}   &        & 3-D tracer fields in the\\
                       & {\tt close\_bgcmean\_bioz.f90}   &        & euphotic zone ({\tt -DPNETCDF})  \\

{\tt bgcmean\_sed.nc }& {\tt open\_bgcmean\_sed.f90}    & netCDF & global time averaged \\
                      & {\tt write\_bgcmean\_sed.f90}   &        & 3-D tracer fields in the\\
                      & {\tt close\_bgcmean\_sed.f90}   &        & sediment ({\tt -DPNETCDF})  \\

{\tt timeser\_bgc}& {\tt save\_timeser\_bgc.f90} & ASCII & time series of tracers and fluxes at defined stations\\
{\tt timeser\_bgc.nc}& {\tt save\_timeser\_bgc.f90} & netCDF & time series of tracers and fluxes at defined stations ({\tt -DPNETCDF})\\
{\tt bgcout}& various & ASCII & diagnostic output during simulation; budgets\\
\hline
\end{tabular}
\end{center}
\end{table}

Model output is written by subroutines {\tt save\_timeseries\_bgc.f90} 
(time series with high temporal resolution for global averages or at specific
stations), {\tt aufr\_bgc.f90} (to be used for initialization of next run), and
{\tt write\_bgcmean.f90} (for global monthly means of specific biogeochemical
tracers). The former two are called at the end of each simulation by routine
{\tt end\_bgc.f90}, the latter one is called directly by the ocean main
routine. Further, some diagnostic output (mainly for debugging purpose) is
written to file {\tt bgcout} during simulation by various subroutines. 



\subsection{\label{input_files}Input files} 

\paragraph{\tt restartr\_bgc.nc} Created with compile option {\tt -DPNETCDF}.
File to restart the model from previous run. Contains fields of all
biogeochemical tracers for water column and sediment, plus latitude, longitude,
water depth, layer depth and the chemical constant fields {\tt chemc(ie,je,7)},
and {\tt aksp(je,ke)}. Further contains storage fields for permanently buried
sediment, {\tt burial(ie,je,4)}. NetCDF format. Read by {\tt aufr\_bgc.f90}, if
model simulation is based on previous one. Written by {\tt aufw\_bgc.f90}.

\paragraph{\tt restartr\_bgc} As {\tt restartr\_bgc.nc}, but binary format.

%\paragraph{\tt INIPHO} Contains initial field of phosphate with dimensions {\tt ie}, {\tt
%je} and {\tt ke}, from World Ocean Atlas (WOA) data set \cite{conkright:1994}.
%Binary format. Read by routine {\tt bgc\_lev.f90}.  File supplied with code is
%{\tt inipo4\_grob.L20}, linked to {\tt INIPHO} via run script. 

%\paragraph{\tt ININIT} Contains initial field of nitrate with dimensions {\tt ie}, {\tt
%je} and {\tt ke}, from World Ocean Atlas (WOA) data set \cite{conkright:1994}.
%Binary format. Read by routine {\tt bgc\_lev.f90}.   File supplied with code is
%{\tt inino3\_grob.L20}, linked to {\tt ININIT} via run script. 

%\paragraph{\tt INISIL} Contains initial field of silicate with dimensions {\tt ie}, {\tt
%je} and {\tt ke}, from World Ocean Atlas (WOA) data set \cite{conkright:1994}.
%Binary format. Read by routine {\tt bgc\_lev.f90}.   File supplied with code is
%{\tt inisil\_grob.L20}, linked to {\tt INISIL} via run script. 

%\paragraph{\tt INIOX} Contains initial field of oxygen with dimensions {\tt ie}, {\tt je}
%and {\tt ke}, from World Ocean Atlas (WOA) data set \cite{conkright:1994}.
%Binary format. Read by routine {\tt bgc\_lev.f90}.   File supplied with code is
%{\tt inio2\_grob.L20}, linked to {\tt INIOX} via run script. 

\paragraph{\tt INPDUST.nc} Contains monthly mean dust input interpolated onto the model grid. 
Available are dust fields from Timmreck and Schulz (2004)\nocite{timmreck:2004}
and from ECHAM5/HAM simulations. NetCDF format. Read by routine {\tt get\_dust.f90}.
Files are interpolated from a 12 month climatology of dust fields from an atmospheric simulation 
("dustdep\_monthl.nc") to the model grid with the following shell-script:
\begin{footnotesize}
\begin{verbatim}
#! /bin/sh
cdo remapcon,/pf/m/m212047/grid/GR15s.nc -setgrid,t63grid dustdep_monthl.nc dustdep_GR15.nc
#
###########################################################################
#
cat > make_dust_GR15.jnl  << EOF
use dustdep_GR15.nc

define axis/modulo/x=1:254:1 xlon
define axis/y=1:220:1 ylat
define axis/T=1:12:1 mytime
define grid/x=xlon/y=ylat/t=mytime mygrid

let/title="dust"/units="kg/m^2/yr" dust = DEP_DUST[g=mygrid@asn]*365*24*60*60

save/clobber/file=GR15_INPDUST.nc dust[i=254:509]
EOF

set -e
ferret << STOP
go make_dust_GR15.jnl
exit
STOP
set +e
\end{verbatim}
\end{footnotesize}


%File supplied with code is {\tt dust\_grob}, linked to
%{\tt INPDUST.nc} via run script. 

\paragraph{\tt NAMELIST\_BGC} List of runtime parameters, to be read during model
simulation. This list sets values for checking/debugging options, model setup
and output options. Its name is {\tt NAMELIST\_BGC}, with constants defined in
{\tt NAMELIST\_BGC.h}, and it is read by {\tt read\_namelist.f90}. 
\begin{itemize}
\item{\tt io\_stdo\_bgc}  Logical number for bgc output unit (set to 8).
\item{\tt isac} Sediment acceleration factor. If set to 1 there is no
sediment acceleration. 
\item{\tt kchck}  Switch to check max/min of bgc arrays on wet/dry
cells. The check is carried out when {\tt kchck} is set to 1. This creates
a very large {\tt bgcout} file and is fairly cpu-time intensive, 
so it is not recommended for longer test runs or production runs.
\item{\tt  nfreqts1} Sampling frequency for time series. Setting {\tt  nfreqts1} to 10 means
time series are sampled every tenth time step.
\item{\tt  nts1}  Number of stations at which time series are sampled.
\item{\tt  rlonts1} Longitude array of biogeochemical time series  stations.  
\item{\tt  rlatts1} Latitude array of biogeochemical time series  stations.  
\item{\tt  rdep1ts1} First sampling depth for export at biogeochemical
time series  stations.  
\item{\tt  rdep2ts1} Second sampling depth for export at biogeochemical
time series  stations.  
\item{\tt  rdep3ts1} Third sampling depth for export at biogeochemical time series 
stations.  
\item{\tt  emission} Only for option {\tt -DPANTHROPOCO2}. Global annual emission
of carbon dioxide, in Pg C.  
\end{itemize}


\subsection{\label{output_files}Output files}

\subsubsection{Standard output: {\tt bgcout}} Contains model diagnostics
written by various subroutines during the course of a simulation. After initial
checks, the model reports the progress of the simulation (day/time step) to this
file. Global fluxes and inventories are written to this file at the end of each
year. ASCII format.

\subsubsection{Time averaged output: {\tt bgcmean}} 
\label{sec:bgcmean}
Only for compile option {\tt -DPNETCDF}. 
Time mean output of biogeochemical tracer fields. The output is written to 
three files: {\tt bgcmean\_2d} contains time,lon,lat fields of surface parameters,
{\tt bgcmean\_3d}  time,lon,lat,depth fields, and {\tt bgcmean\_bioz} time,lon,lat,depth
fields for the euphotic layer. The time dimension depends on the setting of
{\tt mean\_2D\_day} (1=daily, 2=monthly) and {\tt mean\_3D\_month} (1=monthly, 2=length of run).
{\tt mean\_2D\_day} is used for {\tt \_2d} and {\tt \_bioz}, {\tt mean\_3D\_month} for {\tt bgcmean\_3d}.
A list of the output variables can be obtained by typing {\tt ncdump -h bgcmean\_2d.nc} etc.

Values for monthly means are summed-up in the
biogeochemical subroutines (currently {\tt atmotr.f90}, {\tt carchm.f90}, {\tt
dipowa.f90} and {\tt ocprod.f90}). They are averaged at the end of each month for the
monthly mean fields {\tt avrg\_bgcmean\_2d.f90} (\_2d, \_bioz) and 
{\tt avrg\_bgcmean\_3d.f90} (\_3d, \_sed). The output is written to file in 
{\tt write\_bgcmean\_2d.f90}, {\tt write\_bgcmean\_bioz.f90}, {\tt write\_bgcmean\_3d.f90} and {\tt write\_bgcmean\_sed.f90}. 

Field indices for the biogeochemical tracers to be stored
are set in {\tt mo\_bgcmean.f90}. There are four types of mean tracer fields:

%-old-
%\begin{itemize}
%\item{\tt bgct2d(ie,je,nbgct2d)} Three dimensional field over 
%%longitude and latitude,
%$x$ and $y$, 
%containing annual integrals of various {\tt nbgct2d} variables. This field
%contains, e.g., annual means of fluxes to the sediment or air-sea gas-exchange.
%\item{\tt bgcm2d(ie,je,12,nbgcm2d)} Four dimensional field over 
%longitude and latitude,
%$x$ and $y$, 
%containing 12 monthly means of various {\tt nbgcm2d} variables. This field
%contains, e.g., the monthly air-to-sea fluxes of CO$_2$, nitrogen fixation at the
%sea surface, or sedimentation of organic matter at certain depths.
%\item{\tt bgcm3d(ie,je,kwrbioz,12,nbgcm3d)} Five dimensional field over 
%%longitude and latitude,
%$x$ and $y$, 
%and the upper {\tt kwrbioz} depths, containing 12 monthly means of
%various {\tt nbgcm3d} variables. This field contains, e.g., the phytoplankton or
%photosynthesis in the upper four depths. 
%\item{\tt bgcmtot(ie,je,kwrbioz,12,nbgcmtot)} Five dimensional field over 
%%longitude and latitude,
%$x$ and $y$, 
%and the all depths, containing 12 monthly means of various {\tt
%nbgcmtot} variables. 
%\end{itemize}

The dimensions of the monthly or annual fields are set in module {\tt mo\_bgcmean.f90}.
The fields are initialized in {\tt beleg\_bgc.f90}. The final output files {\tt
bgcmean\_2d[\_3d, \_bioz].nc} are in netCDF format and contain not only the fields themselves,
but also information about longitude, latitude, layer depths and grid cell
dimensions. Integral fluxes stored in {\tt bgct2d} are used in {\tt
inventory\_bgc.f90} for calculation of annually integrated fluxes.

\subsubsection{Time series: {\tt timeser\_bgc}} Values of the biogeochemical
tracers at selected locations are written at every {\tt nfreqts1} time step. The field for the
time series is {\tt ts1(nvarts1,nelets1,lents1}), where {\tt nvarts1} 
is the number of variables (e.g., phytoplankton,  photosynthesis,
export ...), {\tt lents1} is the number of times the time series are sampled
(e.g. 360, if the model is run for one year, using a time step of 0.1 day
and a sampling frequency {\tt nfreqts1} of 10) and {\tt nelets1} is the number
of positions where the time series are sampled.  {\tt lents1} is determined
during runtime in file {\tt ini\_timeser\_bgc.f90} from time step, sampling
interval and length of the integration. The variables and the total number of variables ({\tt nvarts1})
are defined in {\tt mo\_timeser\_bgc.f90}.  Content is provided  
in the subroutines where it are computed (e.g., {\tt ocprod.f90}).
Sampling positions are determined in the following way:  
\begin{enumerate} 
\item An initial value for the number of locations {\tt nts} is set in {\tt mo\_timeser\_bgc.f90}.
\item {\tt read\_namelist.f90} reads the locations (latitude, longitude, depth) from {\tt NAMELIST\_BGC}.
The corresponding grid cell indices are determined in subroutine {\tt ini\_timeser\_bgc.f90}.
\item The total number of positions {\tt nelets1} is {\tt nts} + 1, including 
the additional globally averaged time series, which has the index 1.
\end{enumerate}

Output format is ASCII or netCDF ({\tt -DPNETCDF}).
NetCDF output defines one field for each variable (e.g., phytoplankton, ...) 
with dimensions {\tt lents1, nelets1}. The timestamp corresponds to the simulated time-span in seconds.
Files can be concatenated with {\tt nccat}.
For ASCII output the file is structured as follows: first, there are
a few lines (currently 18) of time series information. After
that follows the output for the
time series station. Here in {\tt nvarts1} columns are the
different variables for output (e.g., phytoplankton, phosphate, etc.). Each
time series has {\tt lents1} lines of output (e.g, one line for each day).  The
{\tt nelets1} different time series stations are written one after the other.
The output starts with global inventories (or averages), followed by the
discrete stations. I.e., after the first header files, {\tt timeser\_bgc}
contains {\tt nelets1}$\times${\tt lents1} lines of output with {\tt nvarts1}
columns.

\begin{footnotesize}
\begin{verbatim}
<360 = lents1 lines of time series station no.1:>
ts1(1,1,1)            ts1(2,1,1)...             ..ts1(nvarts1,1,1)
ts1(1,1,2)            ts1(2,1,2)...             ..ts1(nvarts1,1,2)
ts1(1,1,3)            ts1(2,1,3)...             ..ts1(nvarts1,1,3)
.......               .......                     .......
ts1(1,1,lents1)       ts1(2,1,lents1)...        ..ts1(nvarts1,1,lents)
<360 = lents1 lines of time series station no.2:>
ts1(1,2,1)            ts1(2,2,1)...             ..ts1(nvarts1,2,1)
ts1(1,2,2)            ts1(2,2,2)...             ..ts1(nvarts1,2,2)
ts1(1,2,3)            ts1(2,2,3)...             ..ts1(nvarts1,2,3)
.......               .......                     .......
ts1(1,2,lents1)       ts1(2,2,lents1)...        ..ts1(nvarts1,2,lents)
<360 = lents1 lines of time series station no.3:>
......                .......                     .......
<360 = lents1 lines of time series station no.4:>
......                .......                     .......
......                .......                     .......
......                .......                     .......
<360 = lents1 lines of time series station no.nelets1:>
ts1(1,nelets1,1)      ts1(2,nelets1,1)...       ..ts1(nvarts1,nelets1,1) 
ts1(1,nelets1,2)      ts1(2,nelets1,2)...       ..ts1(nvarts1,nelets1,2) 
ts1(1,nelets1,3)      ts1(2,nelets1,3)...       ..ts1(nvarts1,nelets1,3) 
.......               .......                     .......
ts1(1,nelets1,lents1) ts1(2,nelets1,lents1)...  ..ts1(nvarts1,nelets1,lents) 
\end{verbatim}
\end{footnotesize}

\subsubsection{Restart file: {\tt restartw\_bgc.nc}} Created with compile
option {\tt -DPNETCDF}. Identical to input file {\tt restartr\_bgc.nc}
(section \ref{input_files}). The run
script copies {\tt restartw\_bgc.nc} to  {\tt restartr\_bgc.nc} before the
start of the next integration.

\subsubsection{Restart file: {\tt restartw\_bgc}} As {\tt restartw\_bgc.nc},
but binary format.

\clearpage
\newpage
\begin{appendix}

\section{\label{commulication_between_modules}Communication between the
modules}

The following table lists the subroutines of \ham and the subroutines from
which they are called. \\

\footnotesize
\begin{tabular}{l|l} 
Subroutine name          &called by \\ \hline
{\tt atmotr.f90      }&  {\tt bgc.f90}    \\
{\tt aufr\_bgc.f90         }&  {\tt ini\_bgc.f90}    \\
{\tt aufw\_bgc.f90         }&  {\tt mpiom.f90} {\em (MPI-OM)}   \\
{\tt  avrg\_bgcmean\_2d.f90 }&   {\tt bgc.f90, end\_bgc.f90}   \\ 
{\tt  avrg\_bgcmean\_3d.f90 }&   {\tt bgc.f90, end\_bgc.f90}   \\ 
{\tt  avrg\_bgcmean\_bioz.f90 }&   {\tt bgc.f90, end\_bgc.f90}   \\ 
{\tt  avrg\_bgcmean\_sed.f90 }&   {\tt bgc.f90, end\_bgc.f90}   \\ 
{\tt avrg\_timeser\_bgc.f90   }&  {\tt bgc.f90}  \\ 
{\tt beleg\_bgc.f90        }&  {\tt ini\_bgc.f90}   \\
{\tt bgc.f90              }&  {\tt mpiom.f90} {\em (MPI-OM)}   \\
%{\tt bgc\_lev.f90          }&  {\tt beleg\_bgc.f90}   \\
{\tt  bodensed.f90        }&  {\tt ini\_bgc.f90}    \\ 
{\tt  carchm.f90          }&  {\tt bgc.f90}   \\
{\tt  chck\_bgc.f90        }&  {\tt aufw\_bgc.f90, carchm.f90, ini\_bgc.f90, ocprod.f90 } \\
{\tt  chemcon.f90          }&  {\tt ini\_bgc.f90}, {\tt bgc.f90}    \\
{\tt  cyano.f90           }&  {\tt bgc.f90}   \\ 
{\tt  dilute\_bgc.f90      }&  {\tt mpiom.f90} {\em (MPI-OM)}     \\
{\tt  dipowa.f90          }&  {\tt powach.f90}    \\ 
{\tt  end\_bgc.f90         }&  {\tt mpiom.f90} {\em (MPI-OM)}   \\ 
{\tt  get\_dust.f90      }&  {\tt beleg\_bgc.f90}    \\ 
{\tt  ini\_bgc.f90         }&  {\tt mpiom.f90} {\em (MPI-OM)}   \\
{\tt  ini\_timeser\_bgc.f90 }&  {\tt ini\_bgc.f90}    \\ 
{\tt  inventory\_bgc.f90 }&  {\tt bgc.f90, end\_bgc.f90}    \\ 
{\tt  ocdiff\_bgc.f90     }&   {\tt mpiom.f90} {\em (MPI-OM)}   \\ 
{\tt  ocprod.f90          }&  {\tt bgc.f90}    \\
{\tt  open\_bgcmean\_2d.f90 }&   {\tt ini\_bgc.f90}   \\ 
{\tt  open\_bgcmean\_3d.f90 }&   {\tt ini\_bgc.f90}   \\ 
{\tt  open\_bgcmean\_bioz.f90 }& {\tt ini\_bgc.f90}   \\ 
{\tt  open\_bgcmean\_sed.f90 }&  {\tt ini\_bgc.f90}   \\ 
{\tt  powach.f90          }&  {\tt bgc.f90}    \\
{\tt  powadi.f90          }&  {\tt powach.f90}    \\
{\tt  read\_namelist.f90   }&  {\tt ini\_bgc.f90}   \\ 
{\tt  save\_timeser\_bgc.f90}&  {\tt end\_bgc.f90}   \\
{\tt  sedshi.f90         }&   {\tt bgc.f90}    \\ 
{\tt  write\_bgcmean\_2d.f90 }&   {\tt bgc.f90, end\_bgc.f90}   \\ 
{\tt  write\_bgcmean\_3d.f90 }&   {\tt bgc.f90, end\_bgc.f90}   \\ 
{\tt  write\_bgcmean\_bioz.f90 }&   {\tt bgc.f90, end\_bgc.f90}   \\ 
{\tt  write\_bgcmean\_sed.f90 }&   {\tt bgc.f90, end\_bgc.f90}   \\ 
{\tt mo\_biomod.f90      }&  almost all   \\
{\tt mo\_bgcmean.f90      }&  almost all   \\
{\tt mo\_carbch.f90      }&  almost all   \\
{\tt mo\_control\_bgc.f90 }&  almost all   \\
{\tt mo\_sedmnt.f90      }&  almost all   \\
{\tt mo\_timeser\_bgc.f90 }&  {\tt ini\_timeser\_bgc.f90, ocprod.f90, }  \\
&{\tt read\_namelist.f90, save\_timeser\_bgc.f90 }  \\ 
{\tt  NAMELIST\_BGC.h    }&  {\tt read\_namelist.f90}   \\
{\tt  mo\_param1\_bgc.f90      }&  almost all   \\ 
\end{tabular}
\normalsize

\newpage

\section{\label{effect_of_preprocessor_options}Impact of pre-processor keys on the various subroutines of \ham}

The following tables lists the subroutines of \ham. Subroutines are modified (x) or activated (i) by setting the
respective preprocessor keys.\\

\footnotesize
\begin{tabular}{|l|c|c|c|c|c|}\hline
Subroutine name                 &{\tt -DAGG}&{\tt -DDIFFAT}&{\tt -DFB\_BGC\_OCE}&{\tt -DPNETCDF}&{\tt -DPANTHROPOCO2}\\ \hline
{\tt atmotr.f90      }          &           &    i         &                    &               &        i          \\     
{\tt aufr\_bgc.f90         }    &     x     &    x         &                    &       x       &        x          \\       
{\tt aufw\_bgc.f90         }    &     x     &    x         &                    &       x       &        x          \\       
{\tt avrg\_bgcmean\_2d.f90 }    &           &              &                    &               &                   \\        
{\tt avrg\_bgcmean\_3d.f90 }    &           &              &                    &               &                   \\      
{\tt avrg\_timeser\_bgc.f90}    &           &              &                    &               &                   \\  \hline
{\tt beleg\_bgc.f90        }    &     x     &              &         x          &               &        x          \\
{\tt bgc.f90              }     &           &    x         &                    &               &        x          \\
%{\tt bgc\_lev.f90          }    &           &              &                    &               &                   \\
{\tt  bodensed.f90        }     &           &              &                    &               &                   \\ \hline
{\tt  carchm.f90          }     &           &    x         &                    &               &                   \\
{\tt  chck\_bgc.f90        }    &           &              &                    &               &                   \\
{\tt  chemcon.f90          }    &           &              &                    &               &                   \\
{\tt  cyano.f90           }     &           &              &                    &               &                   \\ \hline
{\tt  dilute\_bgc.f90      }    &           &              &                    &               &                   \\
{\tt  dipowa.f90          }     &           &              &                    &               &        x          \\ \hline
{\tt  end\_bgc.f90         }    &           &              &                    &               &                   \\ \hline
{\tt  get\_dust.f90      }      &           &              &                    &               &                   \\ \hline
{\tt  ini\_bgc.f90         }    &           &    x         &                    &               &                   \\
{\tt  ini\_timeser\_bgc.f90 }   &           &              &                    &               &                   \\
{\tt  inventory\_bgc.f90 }      &           &    x         &                    &               &        x          \\ \hline
{\tt  ocprod.f90          }     &     x     &              &         x          &               &        x          \\ \hline
{\tt  powach.f90          }     &           &              &                    &               &                   \\
{\tt  powadi.f90          }     &           &              &                    &               &                   \\ \hline
{\tt  read\_namelist.f90   }    &           &              &                    &               &        x          \\ \hline
{\tt  save\_timeser\_bgc.f90}   &           &              &                    &       x       &                   \\
{\tt  sedshi.f90          }     &           &              &                    &               &                   \\ \hline
{\tt  open\_bgcmean\_2d.f90 }   &     x     &    x         &                    &       x       &        x          \\ 
{\tt  open\_bgcmean\_3d.f90 }   &           &              &                    &       x       &        x          \\
{\tt  open\_bgcmean\_bioz.f90 } &           &              &                    &       x       &        x          \\
{\tt  open\_bgcmean\_sed.f90 }  &           &              &                    &       x       &                   \\
{\tt  close\_bgcmean\_2d.f90 }  &           &              &                    &       x       &                   \\ 
{\tt  close\_bgcmean\_3d.f90 }  &           &              &                    &       x       &                   \\
{\tt  close\_bgcmean\_bioz.f90 }&           &              &                    &       x       &                   \\
{\tt  close\_bgcmean\_sed.f90 } &           &              &                    &       x       &                   \\
{\tt  write\_bgcmean\_2d.f90 }  &     x     &    x         &                    &       x       &        x          \\ 
{\tt  write\_bgcmean\_3d.f90 }  &           &              &                    &       x       &        x          \\
{\tt  write\_bgcmean\_bioz.f90 }&           &              &                    &       x       &        x          \\
{\tt  write\_bgcmean\_sed.f90 } &           &              &                    &       x       &                   \\\hline
{\tt mo\_biomod.f90      }      &     x     &              &         x          &               &                   \\
{\tt mo\_bgcmean.f90      }     &     x     &              &                    &               &        x          \\
{\tt mo\_carbch.f90      }      &           &              &                    &               &        x          \\
{\tt mo\_control\_bgc.f90 }     &           &              &                    &               &                   \\
{\tt mo\_sedmnt.f90      }      &           &              &                    &               &                   \\
{\tt mo\_timeser\_bgc.f90 }     &           &              &                    &               &                   \\
{\tt mo\_param1\_bgc.f90    }   &     x     &              &                    &               &                   \\ \hline
{\tt mpiom.f90    }             &           &              &         x          &               &                   \\ \hline
\end{tabular}
\normalsize


\vspace{.2cm}
Further, there are three options that affect only one or two files and are only
related to initilization (at the start of a simulation) or to debugging:

\footnotesize
{\tt -DPANTHROPOCO2\_START}: Affects {\tt aufr\_bgc.f90}.\\
{\tt -DPBGC\_CK\_TIMESTEP}: Affects {\tt bgc.f90}.\\
%{\tt -DBGCLEVI} Affects {\tt beleg\_bgc.f90}. Inserts {\tt bgc\_lev.f90}.\\
\normalsize



\end{appendix}

\clearpage

\newpage 
\begin{thebibliography}{}




%\bibitem[\protect\citename{Conkright {\em et~al.}, }1994]{conkright:1994}
%Conkright, M.E., Levitus, S. and Boyer, T. (1994).
%\newblock World Ocean Atlas 1994, Vol. 1: Nutrients.
%\newblock {\em Page  150 pp. of:} {\em NOAA Atlas NESDIS 1}.
%\newblock U.S. Gov. Printing Office, Wash., D.C.

\bibitem[\protect\citename{Dickson, }1994]{Dickson:1994}
Dickson, A.G. and C. Goyet (1994).
\newblock {DOE Handbook of methods for the analysis of the various parameters
of the carbon dioxide system in sea water; Version 2. A.G. Dickson \& C. Goyet, eds. 
ORNL/CDIAC-74}

\bibitem[\protect\citename{Evans and Gar\c{c}on, }1997]{jgofs:1997}
Evans, G.T. and Gar\c{c}on, V. (1997).
\newblock {\em One--dimensional models of water column biogeochemistry}.
\newblock JGOFS Report~23. Scientific Committee on Oceanic Research, Bergen,
  Norway, 85 pp.

\bibitem[\protect\citename{Heinze and Maier-Reimer, }1999]{heinze:1999b}
Heinze, C. and Maier-Reimer, E. (1999).
\newblock {\em The Hamburg Oceanic Carbon Cycle Circulation Model version
  ``HAMOCC2s'' for long time integrations}.
\newblock Technical Report~20. Deutsches Klimarechenzentrum,
  Modellberatungsgruppe, Hamburg.

\bibitem[\protect\citename{Heinze {\em et~al.}, }1999]{heinze:1999a}
Heinze, C., Maier-Reimer, E., Winguth, A.M.E. and Archer, D. (1999).
\newblock A global oceanic sediment model for long-term climate studies.
\newblock {\em {G}lobal {B}iogeochem. {C}ycles}, {\bf 13}(1), 221--250.

\bibitem[\protect\citename{Johnson {\em et~al.}, }1997]{johnson:1997}
Johnson, K.S., Gordon, R.M. and Coale, K.H. (1997).
\newblock What controls dissolved iron concentrations in the world ocean?
\newblock {\em Marine Chemistry}, {\bf 57}, 137--161.

\bibitem[\protect\citename{Keeling {\em et~al.}, }1998]{keeling:1998}
Keeling, R.F., Stephens, B.B., Najjar, R.G., Doney, S.C., Archer, D. and
  Heimann, M. (1998).
\newblock Seasonal variations in the atmospheric O$_2$/N$_2$ ratio in relation
  to the air-sea exchange of O$_2$.
\newblock {\em {G}lobal {B}iogeochem. {C}ycles}, {\bf 12}, 141--164.

\bibitem[\protect\citename{Kriest, }2002]{kriest:2002}
Kriest, I. (2002).
\newblock Different parameterizations of marine snow in a 1D-model and their
  influence on representation of marine snow, nitrogen budget and
  sedimentation.
\newblock {\em {D}eep-{S}ea {R}es. {I}}, {\bf 49}, 2133--2162.

\bibitem[\protect\citename{Kriest and Evans, }2000]{kriest:2000}
Kriest, I. and Evans, G.T. (2000).
\newblock A vertically resolved model for phytoplankton aggregation.
\newblock {\em {P}roc. {I}ndian {A}cad. {S}ci. ({E}arth {P}lanet. {S}ci.)},
  {\bf 109}(4), 453--469.

\bibitem[\protect\citename{Maier-Reimer and Hasselmann,
  }1987]{maier-reimer:1987}
Maier-Reimer, E. and Hasselmann, K. (1987).
\newblock Transport and storage of $CO_2$ in the ocean - an inorganic
  ocean-circulation carbon cycle model.
\newblock {\em Climate Dynamics}, {\bf 2}, 63--90.

\bibitem[\protect\citename{Marsland {\em et~al.}, }2003]{marsland:2003}
Marsland, S.J., Haak, H., Jungclaus, J.H., Latif, M. and R\"oske, F. (2003).
\newblock The {Max-Planck-Institute} global ocean/sea ice model with orthogonal
  curvilinear coordinates.
\newblock {\em Ocean Modelling}, {\bf 5}(2), 91--127.

\bibitem[\protect\citename{Saltzman {\em et~al.}, }1993]{saltzman:1993}
Saltzman, E.S., King, D.B., Holmen, K. and Leck, C. (1993).
\newblock Experimental determination of the diffusion coefficient of
  dimethylsulfide in water.
\newblock {\em {J}. {G}eophys. {R}es.}, {\bf 98}(C9), 16481--16486.

\bibitem[\protect\citename{Six and Maier-Reimer, }1996]{six:1996}
Six, K.~D. and Maier-Reimer, E. (1996).
\newblock Effects of plankton dynamics on seasonal carbon fluxes in an ocean
  general circulation model.
\newblock {\em {G}lobal {B}iogeochem. {C}ycles}, {\bf 10}(4), 559--583.

\bibitem[\protect\citename{Timmreck and Schulz, }2004]{timmreck:2004}
Timmreck, C. and Schulz, M. (2004).
\newblock Significant dust simulation differences in nudged and climatological
  operation mode of the {AGCM ECHAM}.
\newblock {\em {J}. {G}eophys. {R}es.}, {\bf 109}(D13), 10.1029/2003JD004381.

\bibitem[\protect\citename{Wanninkhof, }1992]{wanninkhof:1992}
Wanninkhof, R. (1992).
\newblock Relationship between wind speed and gas exchange over the ocean.
\newblock {\em {J}. {G}eophys. {R}es.}, {\bf 97}, 7373--7382.

\bibitem[\protect\citename{Weiss, }1970]{weiss:1970}
Weiss, R.F. (1970).
\newblock The solubility of nitrogen, oxygen and argon in water and sea water.
\newblock {\em {D}eep-{S}ea {R}es.}, {\bf 17}, 721--735.

\bibitem[\protect\citename{Weiss, }1974]{weiss:1974}
Weiss, R.F. (1974).
\newblock Carbon dioxide in water and sea water: The solubility of a non-ideal
  gas.
\newblock {\em {M}arine {C}hem.}, {\bf 2}, 203--215.

\bibitem[\protect\citename{Zeebe, }2001]{Zeebe:2001}
Zeebe, R.E. and D. Wolf-Gladrow (2001).
\newblock CO$_2$ in seawater: Equilibrium, 
kinetics, isotopes.
\newblock {Elsevier Oceanography series}

\end{thebibliography}
\end{document}

